%%%%%%%%%%%%%%%%%%%%%%%%%%%%%%%%%%%%%%%%%%%%%%%%%%%%%%%%%%%%%%%%%%%%%%%%%%%%%%%
% template.tex - version 0.9.1.1 (5/26/2011)
%
% This is a template file for the osudiss-2 class. See
% osudiss-2.pdf for documentation, and the GS material 
% for the requirements.
%
% Copy the following osudiss-2 files your latex path (or just the folder containing this file):
% osudiss-2.cls (v0.9.1)
% sa-draftwater.sty
%
% Then, to compile this file:
% latex template
% bibtex template
% latex template
% latex template
%
% (You can also use pdflatex if you prefer.)
%
%%%%%%%%%%%%%%%%%%%%%%%%%%%%%%%%%%%%%%%%%%%%%%%%%%%%%%%%%%%%%%%%%%%%%%%%%%%%%%%
\documentclass[11pt, onehalf, phd]{osudiss-2}
% The `11pt' option is unnecessary since it is the default

% `onehalf' sets the line spacing to one-and-a-half spacing instead of
% double spacing. 'double' sets the line spacing to double.

% The `phd' option is unnecessary since it is the default

% Remove `draft' option for final draft

%%%%%%%%%%%%%%%%%%%%%%%%% Packages %%%%%%%%%%%%%%%%%%%%%%%%%
% Load your favorite packages here
\usepackage{graphicx} % for importing images in figures - you definitely want this!
\usepackage{lipsum} % for fake latin text---you probably don't want this
\usepackage{verbatimbox}
\usepackage{amsmath}
\usepackage{braket} % bra and ket notation
\usepackage{xcolor} % change font color in blocks
\usepackage[super]{nth} % superscripted 1st, 2nd, 3rd, .... call using \nth{3} to write 3rd.
%\usepackage{gensymb} % degree symbol called by \degree

% For instance... see osudiss-2.pdf for some suggestions, if you don't
% have a clue
\usepackage{bm} % for bold math---useful
\usepackage{booktabs} % for more professional tables
\usepackage[titletoc]{appendix}

%hyperref packages and options
\usepackage{bookmark} % helps booksmarks look better in PDF
%hypersetup option 'breaklinks' is reguired for line wrapping in the table of contents during latex compilation, and can be removed if you use pdflatex
%\hypersetup{colorlinks=true,linkcolor=blue, breaklinks} %internal links in blue, citations in green
\hypersetup{colorlinks=true,linkcolor=blue, citecolor=blue, breaklinks} %all links in blue
\usepackage[all]{hypcap}

%Use of natbib is STRONGLY recommended to sort and compress your references within each citation
%With these options, natbib will convert i.e. [5,3,9,4] to [3-5, 9]
\usepackage[sort&compress]{natbib}
%\usepackage{bibunits} % to get a 2nd bibliography in the Vita (my publications)

\usepackage[capitalise]{cleveref} % to combine multiple equation references
\usepackage{breqn} % automatically break long equations over several lines
\usepackage{physics} % for partial derivatives

%required to have latex automatically generate subfigures (i.e. (a), (b) etc)
\usepackage{subfig}
\usepackage[export]{adjustbox}
\setcounter{lofdepth}{2}
\PassOptionsToPackage{obeyspaces}{url}
%\usepackage{hyperref}% http://ctan.org/pkg/hyperref

% rotated figures and captions
\usepackage{rotating}
\usepackage{tikz}

%load glossaries packages
\usepackage[acronym, section=chapter]{glossaries}
%\usepackage[xindy,acronym, section=chapter]{glossaries} - recommended if supported by your OS
\makeglossaries %required to actually make a glossary
%A list of common acronyms
%Only those used will be displayed, so you can just add to this list


\newacronym{hhg}{HHG}{High Harmonic Generation}











 %load list of acronyms contained in acronyms.tex

%The following commands can be used to help deal with "overfull hbox" issues
%See, for example, http://www.tex.ac.uk/cgi-bin/texfaq2html?label=overfull for details
%\pretolerance 1000
\setlength{\emergencystretch}{3em}
%\tolerance 1000

% packages and commands for tables
\usepackage{makecell} % split-line cells
\usepackage{booktabs} % booktabs style tables
\usepackage{diagbox} % putting a diagonal line in a cell
\usepackage[normalem]{ulem}
\useunder{\uline}{\ul}{}
\renewcommand\theadalign{bc}
\renewcommand\theadfont{\bfseries}
\renewcommand\theadgape{\Gape[4pt]}
\renewcommand\cellgape{\Gape[4pt]}
%%%%%%%%%%%%%%%%%%%%%%%%% Custom Commands/Environments %%%%%%%%%%%%%%%%%%%%%%%%%
% Put your favorite custom commands here

%Print list of abbreviations - use same font as List of Figures and List of Tables for the title, and same formatting in the table of contents.
% Argument #1 - title for list of abbreviations (i.e. List of Abbreviations)

\newcommand\PrintListofAbbreviations[1]{
\phantomsection
\addcontentsline{toc}{front}{\typesetColumnHeading{#1}}
\printglossary[type=\acronymtype,title={\protect {\typesetLevelTwo{#1}}}]
}


\newenvironment{chapabstract}{%
    \begin{center}%
      \bfseries Abstract
    \end{center}}%

% OMRON font command
\newcommand*{\myOMRONfont}{\fontfamily{qag}\selectfont}
\DeclareTextFontCommand{\textOMRON}{\myOMRONfont}
%\newcommand*{\OMRON}{\textOMRON{omRon}}
\newcommand*{\OMRON}{\textOMRON{\resizebox{!}{1ex}{om}\resizebox{!}{1ex}{R}\resizebox{!}{1ex}{on}}}

% Below is an example of customizing the style of headings in your
% dissertation. See osudiss-2.pdf for more information.
%
% For example, if you simply must have uppercase titles:
%\renewcommand\typesetLevelOne[1]{{\Large\textbf{\MakeUppercase{#1}}\par}}
%\renewcommand\typesetLevelTwo[1]{{\Large\textbf{\MakeUppercase{#1}}}}
% Note the \par for \typesetLevelOne
%
% If you want the title to be bold and |\Large| instead of |\Huge|:
%\renewcommand\titleFont{\normalfont\Large\bfseries}

% Add words that TeX may not know how to hyphenate below. This can
% help prevent overfull hboxes. For example,
\hyphenation{eigen-state space-time} 

%%%%%%%%%%%%%%%%%%%%%%%%% Document Metadata %%%%%%%%%%%%%%%%%%%%%%%%%
\title{Application of Attosecond Techniques to Condensed Matter Systems}
\author{Gregory J. Smith}
\advisorname{Louis F. DiMauro}
\degree{Doctor of Philosophy} % Default value
\member{L. Robert Baker}
\member{Jay A. Gupta}
\member{Yuri V. Kovchegov}
\authordegrees{B.Sc., M.Sc.}
\graduationyear{2020}
\unit{Graduate Program in Physics} 

%%%%%%%%%%%%%%%%%%%%%%%%% Begin Document %%%%%%%%%%%%%%%%%%%%%%%%%
\begin{document}

\frontmatter

%\include{template-abstract}
%\dedication{Dedicated to my family.} % Optional, and seriously not this lame
%\begin{acknowledgments}

\lipsum[3-4]

\end{acknowledgments}
%\begin{vita}
\dateitem{Oct 17, 1990}{Born---Kolkata, India}
\dateitem{May, 2013}{B.S., NC State University, Raleigh, NC}
\dateitem{Dec, 2015}{M.S., Ohio State University, Columbus, OH}
% Insert other relevant items here (GTA, etc.)

\begin{publist}

\pubitem{``Constraints on the Diffuse High-Energy Neutrino Flux from the Third Flight of ANITA", 
P. W. Gorham, P. Allison, {\bf O. Banerjee} {\it et al.}, Physical Review D. 
I am a lead author and contributor of the new binned analysis presented, 
which is one of the three complementary analyses in the paper.
\href{https://arxiv.org/abs/1803.02719}{Link to electronic version.}}

\pubitem{``Dynamic tunable notch filters for the Antarctic Impulsive Transient Antenna (ANITA)'',
P. Allison, {\bf O. Banerjee} {\it et al.}, Nuclear Instruments and Methods A. 
I led this paper and served as {\bf corresponding author}. 
This paper is on the filters that I played a lead role in commissioning for ANITA-4, that helped to triple the livetime of the experiment.
%This is the first paper to describe the trigger systems of ANITA-3 and ANITA-4. 
\href{https://www.sciencedirect.com/science/article/pii/S016890021830411X}{Link to electronic version.}} 

I am also a co-author on all ANITA publications (6 total) since Jan 2016.

%\item ``Secret paper", A. Connolly, P. Allison, {\bf O. Banerjee}, Submitting to Astroparticle Physics. 
%This is a theoretical paper, and I worked closely with Connolly on literature review and simulation. 

%\item "GRB paper", O. Banerjee {\it et al.}, In progress for Astroparticle Physics. 

%\item ``Characteristics of Four Upward-pointing Cosmic-ray-like Events Observed with ANITA'',
%P. W. Gorham {\it et al.}, Phys.Rev.Lett. 117 (2016) no.7, 071101. I wrote a preliminary Monte Carlo simulation for energy loss of the tau lepton in different media. 

%\item ``Antarctic Surface Reflectivity Measurements from the ANITA-3 and HiCal-1 Experiments",
%P. W. Gorham {\it et al.}, J. Astron. Instrum. 06, 1740002 (2017). I gave comments. 

\end{publist}


\begin{fieldsstudy}
\majorfield{Physics}
\onestudy{Particle Astrophysics}{Connolly group} % optional
% Alternatively you can do:
\end{fieldsstudy}

\end{vita}

\tableofcontents

% list of figures (comment out if you don't have any figures)
\clearpage %remove if you don't want a page break before list of figures
\listoffigures

% list of tables (comment out if you don't have any tables)
\clearpage  %remove if you don't want a page break before list of tables
\listoftables 

%print glossary - comment out if you don't want this.  Make sure you also add \glsdisablehyper if you don't want to print a glossary, but do use the %glossaries package to keep track of acronyms
\clearpage %remove if you don't want a page break before list of abbreviations
%\PrintListofAbbreviations{List of Abbreviations} %Title is in { } - change if desired
%\printglossary[type=\acronymtype]

\mainmatter
\chapter{Introduction}

\section{Ultrafast Dynamics in Condensed Matter Systems}

timescales and processes in solids

\section{Attosecond Transient Absorption Spectroscopy (ATAS)}

\subsubsection{why are you doing it with HHG?}

include figure of pulse duration vs photon energy, showing different light sources (synchrotrons, HHG sources, XFEL, etc.) tie this into the timescales neccessary to probe condensed matter physics.

\subsection{overview of the technique}

references \cite{ramaseshaRealTimeProbingElectron2016}

The basic concept of an \textit{attosecond transient absorption spectroscopy} (ATAS) experiment is shown in \cref{fig:ATAS_Cartoon_Si_Leone}. In this experiment, a sample is placed at the combined XUV/IR focus in a transmission (normal) geometry. An XUV photon spectrometer is placed behind the sample and the transmitted XUV spectrum $S$ is measured as a function of XUV-IR delay. The IR light is not measured by the spectrometer.

Fundamentally, changes in photoabsorption correspond to electron and phonon dynamics in the sample. In condensed matter materials, these processes occur on the picosecond ($10^{-12}$ s), femtosecond ($10^{-15}$ s) and even attosecond ($10^{-18}$ s) time scales \cite{schultzeAttosecondBandgapDynamics2014,cushingDifferentiatingPhotoexcitedCarrier2019,zurchDirectSimultaneousObservation2017,volkovAttosecondScreeningDynamics2019}. At XUV energies, photons drive electronic transitions from a core-level state to one near the Fermi level, which requires electron population in the initial state and a vacancy in the final state. Because the initial state is tens or even hundreds of eV below the bandgap, it is shielded from the external IR field. The final states, being closer to the Fermi level, enjoy no such shielding. Therefore they can be distorted by the external IR field, and the electron population can be transferred between these states in response to the IR field. After an initial IR excitation, electrons relax via different scattering channels, including with other electrons or phonon modes with longer lifetimes. Provided the dipole selection rules allow it, the photoabsorption spectrum is sensitive to all of these dynamics. Thus by measuring the XUV spectrum as a function of XUV-IR delay, we can track the electronic and phononic response of a sample to an ultrafast IR excitation.


\subsubsection{induced dipole interpretation}

\subsubsection{population transfer and probing interpretation}

\subsubsection{comparison of absorptive and reflective measurements}


\begin{figure}
	\centering
	\includegraphics[width=0.75\textwidth]{figures/chap1/Fresnel_Geometry.pdf}
	\caption{Normal and non-normal incident geometries. \textbf{a)} Normal incidence geometry showing Fresnel coefficients $R_F$, $T_F$ for interfaces and total transmission $T$ and reflectance $R$ for a slab of thickness $L$. Figure recreated from \cite{nichelattiComplexRefractiveIndex2002}. \textbf{b)} Non-normal geometry showing definitions of angles $\theta_i, \theta_r$ and $\theta_t$ with respect to each interface.}
	\label{fig:Fresnel_Geometry}
\end{figure}

\begin{figure}
	\centering
	\includegraphics[width=0.75\textwidth]{figures/chap1/Si_transmission_Fresnel.pdf}
	\caption{Consequences of ignoring the real part of $\tilde{n}$ when calculating the transmission $T$ of a thin sample. Top panel: complex refractive index of silicon using the notation from \cref{eqn:complex_index}. The Si $L$-edge absorption feature is visible near 100 eV. Data from \cite{gulliksonCXROXRayInteractions}. Bottom panel: relative error in $T$, as defined in \cref{eqn:Fresnel_rel_err}, introduced by ignoring the contribution of $\Re(\tilde{n})$. An infinite number of bounces (e.g., \cref{eqn:Fresnel_coefs_inf_bounce}) is assumed.}
	\label{fig:Si_transmission_Fresnel}
	% plotted using \Python Scripts\CXRO\test\real_imag_index_plotting.py
\end{figure}

In a transient absorption experiment, we measure the transmission $T$ of a sample in response to excitation by an external field. Generally speaking, $T$ depends on both parts of the complex refractive index: $\tilde{n} = n + i k$. However, in a normal transmission geometry it turns out that the contribution of $\Im(\tilde{n})$ dominates the measured signal, and to a good approximation the role of $\Re(\tilde{n})$ can be ignored. Note that in a non-normal reflection geometry, both parts of $\tilde{n}$ make significant contributitions to the measured signal. In the following discussion we will analyze the Fresnel equations to see why this is the case. This section will draw from arguments made in reference \cite{nichelattiComplexRefractiveIndex2002}.

First, we consider the normal geometry shown in the left panel of \cref{fig:Fresnel_Geometry}. We write the complex index of refraction in the following form:
\begin{equation}
\begin{aligned}
\tilde{n} &= n - i k \\
&= (1-\delta) - i \beta
\end{aligned}
\label{eqn:complex_index}
\end{equation}
The Fresnel coefficients $R_F$ and $T_F$ describe the interface reflectance and transmittance and depend on both parts of the complex index $\tilde{n}$. For normal incidence, they are:
\begin{equation}
\begin{aligned}
R_F &= \left| \frac{n-ik-1}{n-ik+1}   \right|^2 \\
T_F &=  \frac{4n}{\left|n-ik+1\right|^2}
\end{aligned}
\label{eqn:fresnel_normal}
\end{equation}
Absorption in the bulk is described via the absorption length $\alpha$:
\begin{equation}
\alpha = 4 \pi k / \lambda
\end{equation}
Ignoring interface effects, the transmisison through the bulk is:
\begin{equation}
T_{\text{bulk}} = \exp( - \alpha L)
\end{equation}
Note that $\alpha$ and $T_{\text{bulk}}$ only depend on $k$.

The total reflectance $R$ and transmission $T$ are the result of interface effects plus bulk effects. We must consider the case where the detected light is the result of multiple reflections within the sample. Neglecting interference, we consider the case of $2N$ bounces where the laser's coherence length is less than the thickness of the bulk. In this case, the sum is incoherent with the expressions for $T$ and $R$ given by:
\begin{equation}
\begin{aligned}
R &= R_F + R_F T_F^2 T_{\text{bulk}}^2 \sum_{m=0}^{N} \left[ R_F T_{\text{bulk}} \right]^{2m} \\
T &= T_F^2 T_{\text{bulk}} \sum_{m=0}^{N} \left[ R_F T_{\text{bulk}} \right]^{2m}
\end{aligned}
\label{eqn:Fresnel_coefs_N_bounce}
\end{equation}
For the case of an infinite number of bounces, \cref{eqn:Fresnel_coefs_N_bounce} simplifies to:
\begin{equation}
\begin{aligned}
R &= R_F + \frac{R_F T_F^2 T_{\text{bulk}}^2}{1-R_F^2 T_{\text{bulk}}^2} \\
T &= \frac{T_F^2 T_{\text{bulk}}}{1-R_F^2 T_{\text{bulk}}^2},
\end{aligned}
\label{eqn:Fresnel_coefs_inf_bounce}
\end{equation}
whereas if only a single bounce occurs, \cref{eqn:Fresnel_coefs_N_bounce} reduces to:
\begin{equation}
\begin{aligned}
R &= R_F + R_F T_F^2 T_{\text{bulk}}^2 \\
T &= T_F^2 T_{\text{bulk}}
\end{aligned}
\label{eqn:Fresnel_coefs_1_bounce}
\end{equation}

We now consider the fractional error introduced by ignoring the interface effects described by $T_F$ and $R_F$. That is, what would happen if we assume that the interfaces have no effect on the transmitted intensity? We introduce the relative error $\epsilon$ made by ignoring the Fresnel coefficients of \cref{eqn:Fresnel_coefs_inf_bounce}:
\begin{equation}
\epsilon \equiv \frac{T_{\text{bulk}}}{T} - 1
\label{eqn:Fresnel_rel_err}
\end{equation}

As an example, consider a 100 nm thick Si sample measured in transmission near the Si $L$-edge (about 100 eV), as shown in \cref{fig:Si_transmission_Fresnel}. The relative error is in the range of one part in $10^4$ to $10^5$, well below our experimental detection limit. Silicon was chosen due to its data availability above and below the absorption edge, but this behavior should hold for all materials in normal transmission.

The real part of the complex index becomes important when the sample isn't normal to the beam, as shown in the right panel of \cref{fig:Fresnel_Geometry}. In this case, the Fresnel equations are a bit messier:
\begin{equation}
\begin{aligned}
R_s &= \left| \frac{\tilde{n}_1 \cos \theta_i - \tilde{n}_2 \cos \theta_t}{\tilde{n}_1 \cos \theta_i + \tilde{n}_2 \cos \theta_t}  \right|^2 \\
R_p &= \left| \frac{\tilde{n}_1 \cos \theta_t - \tilde{n}_2 \cos \theta_i}{\tilde{n}_1 \cos \theta_t + \tilde{n}_2 \cos \theta_i}  \right|^2 \\
T_s &= 1 - R_s \\
T_p &= 1 - R_p \\
%\theta_t &= \sqrt{1- \left( \frac{n_1}{n_2} \sin \theta_i \right)^2}
\end{aligned}
\label{eqn:Fresnel_nonnormal}
\end{equation}

Here, the subscripts $s$ and $p$ denote the polarization relative to the surface normal. For a sample in vacuum, $\tilde{n}_1=1$ and $\tilde{n}_2$ is the index of the sample. We can extract the relevant physics without any additional manipulation of \cref{eqn:Fresnel_nonnormal}. Right away, we can see that unlike \cref{eqn:fresnel_normal}, \cref{eqn:Fresnel_nonnormal} is symmetric in the real and imaginary parts of the sample's complex index, $\tilde{n}_2$. In the limit of a thick slab, ($L \gg \alpha$), the light is attenuated before it can reflect off the back surface and we have $T \rightarrow 0$ and $R \rightarrow R_{s,p}$. That is, the only contributions to the reflected intensity are from the interface and possibly the sample volume within $z \approx 1/\alpha$ of the interface. As a result, both parts of $\tilde{n}_2$ will make significant contributions to the reflected intensity. This geometry is common in transient reflection-absorption experiments \cite{cirriAchievingSurfaceSensitivity2017,kaplanFemtosecondTrackingCarrier2018}.



complex refractive index

sample requirements and preparation

pointing stability (in reflection, sample is an XUV optic)

\subsection{previous work}
what is state of the art?

previous work in condensed matter (Si, Ge, Si-Ge, etc)

motivation for long-wavelength studies in condensed matter

\subsection{physical observables in ATAS}
limited k-space information (requires single crystal)

transmission geometry measures imaginary and not the real part of n

\subsection{interpretation of experimental data}

\section{High Harmonic Generation}

High harmonic generation (HHG) is the extremely nonlinear process in which a strong infrared field produces light with frequencies that are integer multiples of the fundamental field after interacting with a medium. Rather than providing a first principles discussion of HHG, the main objective of this section is to understand how we can produce bright attosecond XUV light pulses with a sufficient spectral coverage for use in an ATAS experiment.


We consider the case of an atom in strong laser field, where the electric field of the laser is comparable to the Coulomb field of the parent atom. Under these conditions, there is an appreciable chance of ionization. To determine which physical process is responsible for the ionization, we must consider two energy scales: the ionization potential of the atom $I_p$ and the ponderomotive energy $U_p$:
\begin{equation}
U_p = \frac{q_e^2 F_0^2}{4 m_e \omega^2} \propto I_0 \lambda^2
\label{eqn:Up}
\end{equation}
where $m_e$ is the electron mass, $q_e$ is the electron charge, $\omega$ is the frequency, $F_0$ is the electric field strength, $I_0$ is the intensity, and $\lambda$ is the wavelength of the laser. A more useful form of \cref{eqn:Up} is given below:
\begin{equation}
U_p \textrm{ [eV]} = \left( 9.33738 \times 10^{-5} \right) \times I_0 \textrm{[PW/cm\textsuperscript{2}]} \times \lambda\textrm{[nm]}^2
\end{equation}

The Keldysh parameter $\gamma$ compares the magnitudes of these two energy scales \cite{keldyshIonizationFieldStrong1965}:
\begin{equation}
\gamma = \sqrt{\frac{I_p}{2 U_p}}
\end{equation}
The value of $\gamma$ determines the physical mechanism responsible for ionization. When $\gamma \gg 1$, we are in the multi-photon region; $\gamma \le 1/2$ corresponds to tunnel ionization, and $\gamma \ll 1$ corresponds to above the barrier ionization. We will restrict our discussion to the tunnelling regime, where HHG occurs.

- spectral coverage of harmonics

- pulse duration of harmonic light

Gas phase HHG 

symmetry leads to odd harmonics.

gas and solid HHG has been studied

\subsection{Single Atom Response}

\begin{figure}
	\centering
	\includegraphics[width=0.75\textwidth]{figures/chap1/ThreeStepModel.png}
	\caption{The three step model of HHG. Figure adapted from \cite{schounAttosecondHighHarmonicSpectroscopy2015}.}
	\label{fig:ThreeStepModel}
\end{figure}

\begin{figure}
	\centering
	\subfloat[]{
		\includegraphics[width=0.4\textwidth]{figures/chap1/eich_APT_a.pdf}
		\label{fig:APT_time_domain}}
	\qquad
	\subfloat[]{
		\includegraphics[width=0.4\textwidth]{figures/chap1/eich_APT_b.pdf}
		\label{fig:APT_freq_domain}}
	\caption{Time and frequency domain pictures of HHG. Figure adapted from \cite{eichTimeAngleresolvedPhotoemission2014}.}
	\label{fig:APT_IR_field}
\end{figure}

\begin{table}[]
	\centering
	\begin{tabular}{l|l|l|l|l|l}
		atom & $I_p$ (eV) & $I_p$ (at. u.) & $l$ & $m$ & $F_0$ (at. u.) \\ \hline
		He & 24.5874 & 0.90357 & 0 & 0 & 2.42946 \\
		Ne & 21.5645 & 0.792481 & 1 & 0 & 1.99547 \\
		Ar & 15.7596 & 0.579155 & 1 & 0 & 1.24665 \\
		Kr & 13.9996 & 0.514476 & 1 & 0 & 1.04375 \\
		Xe & 12.1298 & 0.445762 & 1 & 0 & 0.84187
	\end{tabular}
	\caption{ADK parameters.}
	\label{tab:ADK-params}
\end{table}

\begin{figure}
	\centering
	\includegraphics[width=0.75\textwidth]{figures/chap1/ADK_ion_frac.pdf}
	\caption{ADK ionization rate.}
	\label{fig:ADK_ion_frac}
\end{figure}

We start with a microscopic picture of harmonic generation, focusing on the interaction between a single atom and the laser field. In the early 1990s, a semi-classical model was developed to describe the process of high harmonic generation in three discrete steps: ionization, classical propagation in the vacuum, and recombination \cite{schaferThresholdIonizationHigh1993,corkumPlasmaPerspectiveStrong1993}. This model accurately predicts many of the fundamental features of HHG.

In the three step model, the electric field strength is on the order of the atomic potential that binds the electron to its parent atom. The valence electron's wavepacket evolves subject to the sum of the shielded Coloumb field and the spatially varying laser field. The electron can tunnel out of the distorted Coloumb field, as shown in the left panel of \cref{fig:ThreeStepModel}. This step is most likely to occur at the peak of the field, which occurs every half-cycle of the laser period.


============================


\textbf{ADK rate formula}
The rate of ionization from the ground state in the tunnelling regime is best described by the ADK formula \cite{ammosovTunnelIonizationComplex1986} (double check these equations):
\begin{equation}
\omega(t) = \omega_p \left|C_{n*}\right|^2 \left(\frac{4 \omega_p}{\omega_t}\right)^{2 n*1-1} \exp \left(- \frac{4 \omega_p}{3 \omega_t}\right)
\end{equation}
with:
\begin{align}
\omega_p &= \frac{I_p}{\hbar} \\
\omega_t &= \frac{e E(t)}{\sqrt{2 m I_p}} \\
n^* &= Z \sqrt{\frac{I_H}{I_p}} \\
\left|C_{n*}\right|^2 &= \frac{2^{2n*}}{n^* \Gamma (n^*+1) \Gamma (n^*)}
\end{align}
$I_p$ is the ionization potential of the atom, $E(t)$ is the electric field of the laser, $m$ is the electron mass, $Z$ is the ion charge after ionization, $I_H$ is the ionization potential of atomic hydrogen, and $\Gamma(x)$ is the gamma function.

\textbf{from Zenghu's book (p.183-4)} \cite{changFundamentalsAttosecondOptics2011}:

$n^*$ is the effective prinicple quantum number, $l^* = n^* -1 $ is the effective orbital quantum number, and $w_{ADK}$ is the instantaneous ADK rate:
\begin{equation}
w_{ADK} = \left|C_{n^*l^*}\right|^2 G_{lm} I_p \left( \frac{2 F_0}{F} \right)^{2n^*-|m|-1} \exp \left(-\frac{2 F_0}{3 F}\right)
\end{equation}
with $F_0 = (2 I_p)^{3/2}$.
the cycle averaged rate is $\bar{w}_{ADK}$:
\begin{equation}
\bar{w}_{ADK} = \sqrt{\frac{2}{\pi}} \sqrt{\frac{3 F_a}{2 F_0}} w_{ADK} (F_a)
\end{equation}

\textbf{from marco's paper} \cite{laiExperimentalInvestigationStrongfieldionization2017}:
\begin{equation}
w_{ADK}(F) = c^2_{n^*l^*} f(l,m) I_p \left( \frac{2}{F (n^*)^3} \right)^{2n^*-|m|-1} \exp \left( - \frac{2 (2 I_p)^{3/2}}{3F} \right)
\end{equation}

with
\begin{align}
c_{n*l^*} &= \frac{2^{2n^*}}{n^* \Gamma(n^* + l^* + 1) \Gamma(n^* - l^*)}  \\
f(l,m) &= \frac{(2l+1)(l+|m|)!}{2^{|m|} (|m|)! (l-|m|)!}\\
n^* &= \frac{1}{\sqrt{2I_p}}\\
l^* &= n^* - 1 
\end{align}

$\Gamma$ is the Gamma function
the fraction of atoms ionized by time $t$ is given by:

\begin{equation}
\eta(t) = \exp \left( - \int_{-\infty}^{t} \omega(t') \dd{t'} \right)
\end{equation}

=================


The recently liberated electron is assumed to be born with zero initial kinetic energy. It accelerates in the oscillating laser field, gaining kinetic energy along the way, as shown in the central panel of \cref{fig:ThreeStepModel}. Its kinetic energy is proportional to the cycle-averaged quiver energy:
\begin{equation}
U_p = \frac{q_e^2 F_0^2}{4 m_e \omega^2} \propto I_0 \lambda^2
\label{eqn:Up}
\end{equation}
where $m_e$ is the electron mass, $q_e$ is the electron charge, $\omega$ is the frequency, $F_0$ is the electric field strength, $I_0$ is the intensity, and $\lambda$ is the wavelength of the laser. A more useful form of \cref{eqn:Up} is given below:
\begin{equation}
U_p \textrm{ [eV]} = \left( 9.33738 \times 10^{-5} \right) \times I_0 \textrm{[PW/cm\textsuperscript{2}]} \times \lambda\textrm{[nm]}^2
\end{equation}

The birth phase of the electron (relative to the laser period) determines its classical trajectory. Some electrons will be driven away from the parent ion, never to return; some will be driven back to the birth location, where they can scatter off of, miss, or recombine with the parent ion. We will only concern ourselves with those electrons that recombine (right panel of \cref{fig:ThreeStepModel}). Upon recombination, the electron will emit a photon of energy $I_p + KE$, where $I_p$ is the ionization potential of the atom and $KE$ is the kinetic energy acquired during the propagation step. A classical analysis of the electron propagation reveals that the maximum kinetic energy such an electron can gain is $3.17 U_p$, and therefore the maximum photon energy is: 
\begin{equation}
\hbar \omega_{cutoff} = I_p + 3.17 U_p
\label{eqn:cutoff_energy}
\end{equation}
This quantity is often called the cutoff energy, and it is proportional to $I_0 \lambda^2$. Thus, we can extend the maximum photon energy of the harmonics by increasing the fundamental wavelength of the laser.

Unfortunately, the brightness of an individual harmonic order will decrease strongly with increasing wavelength, with the intensity scaling between $\lambda^{-5}$ and $\lambda^{-6}$ \cite{tateScalingWavePacketDynamics2007,shinerWavelengthScalingHigh2009}. We can conceptually understand this as the compounding of two separate problems \cite{lewensteinTheoryHighharmonicGeneration1994}. First, longer wavelengths extend the cutoff energy, spreading a fixed harmonic conversion efficiency across more harmonics and lowering the brightness of each individual harmonic. This accounts for a factor of $\lambda^{-2}$. Secondly, the recombination probability scales inversely with square of the electron wavepacket spread that occurs during the propagation step. Longer wavelengths mean longer excursion times $\tau$, and the wavepacket spreads out as $\tau^{3/2}$. Since we are concerned with the harmonic intensity, we square this value to get $\tau^3 \propto \lambda^{-3}$. With this simple argument, we can see why the harmonic brightness should decrease as $\lambda^{-5}$.

So far, it appears that XUV spectrum is continuous in energy, ranging from $I_p$ to $\hbar \omega_{cutoff}$. This is because we have been considering the effects of a single-cycle laser pulse. In a multi-cycle pulse, the ionization-propagation-recombination steps will happen twice per pulse (every $T_0/2$ seconds), and each event results in a brief burst of light, as shown in \cref{fig:APT_time_domain}. If we Fourier transform this comb of attosecond pulses, we will get a comb in the frequency domain with separation $2 \omega_0$, as shown in \cref{fig:APT_freq_domain}. Thus, we expect to see only odd harmonics of the laser frequency $\omega_0$.

\subsection{Macroscopic Picture}

We now zoom out to the macroscopic picture, which encompasses the entire gas-laser interaction volume. In the far field, the radiation from individual atoms will be coherently summed to form an bright XUV light source. The overall efficiency of the HHG process depends on the phase mismatch $\Delta k$ of the individual dipoles across the interaction volume. We will see how the phase matching determines optimal interaction pressures for a given driving wavelength. Additionally, we will see the effect of XUV reabsorption by the gas on the overall XUV brightness.

\subsubsection{Phase Matching}

\begin{figure}
	\centering
	\includegraphics[width=0.75\textwidth]{figures/chap1/crit_ion_frac.pdf}
	\caption{The critical ionization fraction, $\eta_c$ for helium and argon at different fundamental wavelengths. Refractive index information from \cite{gulliksonCXROXRayInteractions,peckDispersionArgon1964,mansfieldDispersionHelium1969}.}
	\label{fig:crit_ion_frac}
\end{figure}

HHG will be an efficient process if the wave vector mismatch $\Delta k$ of the independent dipoles is zero. The phase mismatch term can be expressed as four separate factors, each arising from distinct physical phenomena \cite{rothhardtAbsorptionlimitedPhasematchedHigh2014}:
\begin{equation}
\Delta k \equiv q k_{\omega} - k_{q \omega} = \Delta k_{\textrm{neutral}} + \Delta k_{\textrm{plasma}} + \Delta k_{\textrm{Gouy}} + \Delta k_{\textrm{dipole}}
\label{eqn:phase_mismatch}
\end{equation}
The first two term represents the dispersion from the generating neutral atoms and the electron plasma. In the discussion that follows below, we will use the refractive index $n$ at STP (pressure $P_0$, number density $\rho_0$, temperature $T_0$), where literature values are readily available. We assume that the index scales linearly with the interaction pressure (density). That is, we assume $n(\rho) = (1-\eta)(\rho / \rho_0) n(\rho_0)$, where $\rho$ is the interaction density before any atoms are ionized and $\eta$ is the ionization fraction of the gas medium. For brevity, we will write the index of refraction at STP as $n = n(\rho_0)$. Using this notation, the neutral dispersion mismatch term for an interaction density $\rho$ is:
\begin{align}
\Delta k_{\textrm{neutral}} &= q k_{\textrm{neutral}}(\lambda_1) - k_{\textrm{neutral}}(\lambda_{q}) \nonumber \\
%&= \frac{q}{2 \pi \lambda_1} (1 - \eta) \frac{\rho}{\rho_0}\Delta n
&= \frac{2 \pi q}{\lambda_1} (1-\eta) \frac{\rho}{\rho_0}\Delta n
\label{eqn:deltak_neutral}
\end{align}
where we have used $k(\lambda)= \omega n /c = n / (2 \pi \lambda)$ and defined $\Delta n$ as:
\begin{equation}
\Delta n \equiv n_{\textrm{neutral}}(\lambda_1) - n_{\textrm{neutral}}(\lambda_q)
\end{equation}
using the notation $\lambda_q \equiv \lambda_1 / q$ for the wavelength of the $q^{th}$ harmonic with frequency $\omega_q \equiv q \omega$ and $\lambda_1$ ($\omega_1)$ for the laser wavelength (frequency). To compute the plasma mismatch term, we start with the refractive index of a plasma:
\begin{align}
n_{\textrm{plasma}}(\omega) &= \sqrt{1 - \frac{\omega_p^2}{\omega^2}} \approx 1 - \frac{\omega_p^2}{2 \omega^2} \\
\omega_p &= \sqrt{\frac{e^2 \rho_e}{\epsilon_0 m_e}}
\end{align}
where $\omega_p$ is the plasma frequency and $\rho_e = \eta \rho$ is the plasma density. Then, the phase mismatch of the plasma is:
\begin{align}
\Delta k_{\textrm{plasma}} &= q k_{\omega_1} - k_{q \omega_1} \nonumber \\
&= \frac{q}{2 \pi \lambda_1} (n_{\textrm{plasma}}(\omega_1) -n_{\textrm{plasma}}(\omega_q) ) \nonumber \\
&= - \frac{1}{4 \pi \lambda_1} \frac{q^2-1}{q} \frac{\omega_p^2}{\omega_1^2} \nonumber \\
&\approx - \frac{q}{4 \pi \lambda_1} \frac{\omega_p^2}{\omega_1^2} \quad \textrm{(for large $q$)} \nonumber \\
&= - q \eta \rho  r_e \lambda_1
% &= - 4 \pi^2 q r_e \rho_e \lambda_1, \quad \textrm{using } r_e = \frac{1}{4 \pi \epsilon_0} \frac{e^2}{m_e c^2} \nonumber \\
% &= - 4 \pi^2 q r_e \eta \rho \lambda_1, \quad \textrm{using } \rho_e = \eta \rho
\end{align}
Where $r_e$ is the classical electron radius:
\begin{equation}
r_e = \frac{1}{4 \pi \epsilon_0} \frac{e^2}{m_e c^2}
\end{equation}
Note that since $\Delta n > 0$ for the wavelengths of interest, we have $\Delta k_{\textrm{plasma}} \le 0$. Adding the two dispersive terms together, we get an expression for the dispersive phase mismatch terms:
\begin{equation}
% \Delta k_{\textrm{neutral}} + \Delta k_{\textrm{plasma}} = \frac{q}{2 \pi \lambda_1} \left( (1-\eta) \frac{\rho}{\rho_0} \Delta n - 8 \pi ^3 r_e \eta \rho \lambda_1^2 \right)
\Delta k_{\textrm{neutral}} + \Delta k_{\textrm{plasma}} = \frac{2 \pi q}{\lambda_1} (1-\eta) \frac{\rho}{\rho_0}\Delta n - q \eta \rho  r_e \lambda_1
\label{eqn:deltak_dispersive}
\end{equation}
Inspection of \cref{eqn:deltak_dispersive} reveals that there exists a \textit{critical ionization fraction} $\eta_c$ for which the dispersive phase terms sum to zero:
\begin{equation}
% \eta_c \equiv \left( 1+ \frac{8 \pi^3 \rho_0 r_e \lambda_1^2}{\Delta n} \right)^{-1}
\eta_c \equiv \left( 1+ \frac{\rho_0 r_e \lambda_1^2}{2 \pi \Delta n} \right)^{-1}
\end{equation}
We can rewrite \cref{eqn:deltak_dispersive} using $\eta_c$:
\begin{equation}
% \Delta k_{\textrm{neutral}} + \Delta k_{\textrm{plasma}} = \frac{q \Delta n}{2 \pi \lambda_1} \frac{\rho}{\rho_0} \left(1 - \frac{\eta}{\eta_c}\right)
\Delta k_{\textrm{neutral}} + \Delta k_{\textrm{plasma}} = \frac{2 \pi q \Delta n}{\lambda_1} \frac{\rho}{\rho_0} \left(1 - \frac{\eta}{\eta_c}\right)
\end{equation}
The critical ionization fraction for helium and argon is shown in \cref{fig:crit_ion_frac}. Note that for $\eta < \eta_c$, the dispersive mismatch term is positive. We will see below that at the IR focus, phase matching ($\Delta k = 0$) is impossible for $\eta > \eta_c$.

The third term of \cref{eqn:phase_mismatch} is the geometrical phase mismatch caused by focusing:
\begin{align}
\Delta k_{\textrm{Gouy}} &= \pdv{z} \left[ q \arctan\left( \frac{2 z}{b_1} \right) - \arctan \left( \frac{2 z}{b_q} \right) \right] \nonumber \\
&= q \frac{2 b_1}{b_1^2 + 4 z^2} - \frac{2 b_q}{b_q^2 + 4 z^2} \nonumber \\
&\approx -(q-1) \frac{2 b_1}{b_1^2 + 4 z^2}
\end{align}
where $b_1 = 2 z_R$ is the confocal parameter and $z_R$ is the Rayleigh range. For the $q^{th}$ harmonic, we assume the nonlinear process obeys a power law $p$ and $b_q = b_1 p /q$ \cite{schounAttosecondHighHarmonicSpectroscopy2015}. Note that $\Delta k_{Gouy}$ is negative for all values of $z$.

The fourth term arises from the intensity-dependent dipole phase acquired during the electron excursion \cite{lewensteinTheoryHighharmonicGeneration1994,balcouGeneralizedPhasematchingConditions1997,salieresCoherenceControlHighOrder1995}:
\begin{equation}
\Delta k_{\textrm{dipole}} = - \alpha_q \pdv{I}{z}
\label{eqn:deltak_atomic}
\end{equation}
The value of $\alpha_q$ depends on the quantum trajectory the electron takes during its excursion. For short trajectories, {$\alpha_q = 2 \times 10^{-14}$ cm\textsuperscript{2}/W} and for long trajectories, {$\alpha_q = 22 \times 10^{-14}$ cm\textsuperscript{2}/W} \cite{kazamiasPressureinducedPhaseMatching2011,balcouQuantumpathAnalysisPhase1999}. The sign of $\Delta k_{dipole}$ is positive (negative) if the gas source is located upstream (downstream) of the focus.

Experimentally, the phase matching can be adjusted by tuning the laser parameters (wavelength $\lambda$, intensity $I$, pulse duration, focal spot size $w_0$), the gas species ($I_p, n$ and $k$) and interaction pressure $P$, and the gas location relative to the laser focus ($z$). Additionally, a variable aperture (iris) located just before the generation lens effectively tunes multiple laser parameters simultaneously, and is known colloquially as ``the magic iris trick" \cite{kazamiasHighOrderHarmonic2002}. Because $\Delta k$ is dependent on the harmonic order $q$, it is impossible to perfectly phase match the entire harmonic spectrum simultaneously. As a result, we adjust the phase matching parameters to optimize the useful part of the harmonic spectrum, usually at the expense of the rest of the spectrum.

Also note that the dispersive terms phase can be controlled by tuning the interaction pressure $p$, while the other terms are (to first order) pressure independent. If we place the gas medium at the focus, $\Delta k_{\textrm{dipole}} = 0$, and $\Delta k_{\textrm{Gouy}} = -(q-1) \lambda / (\pi w_0^2)$. In this case, the condition $\Delta k = 0$ can be met by setting to the density to the \textit{optimal phase matching density} $\rho_{\textrm{opt}}$ \cite{rothhardtAbsorptionlimitedPhasematchedHigh2014}:
\begin{equation}
\frac{\rho_{\textrm{opt}}}{\rho_0} = \left( \frac{q-1}{q} \right) \frac{2 \lambda^2}{\Delta n w_0^2 \left(1 - \frac{\eta}{\eta_c}\right)}
%\frac{2 \pi^2}{\Delta n} \left( \frac{q-1}{q} \right) \left( \frac{1}{1-\frac{\eta}{\eta_c}} \right) \frac{\lambda^2}{\pi^2 w_0^2 + \left(\frac{z \lambda}{w_0}\right)^2}
\label{eqn:phase_matching_density}
% this equation follows from my notation.
\end{equation}
%\begin{equation}
%p_{opt} = p_0 \frac{\lambda^2}{2 \pi^2 w_0^2 \Delta \delta \left( 1 - \frac{\eta}{\eta_c} \right)}
%\label{eqn:phase_matching_pressure}
%% this equation is from rothhardtAbsorptionlimitedPhasematchedHigh2014
%\end{equation}



We therefore arrive at the conclusion that the optimal phase matching density scales with the square of the fundamental wavelength. Furthermore, tighter focusing (${w_0 \rightarrow 0}$) and higher ionization fractions (${\eta \rightarrow \eta_c}$) require higher densities. Therefore, creating bright harmonics from a long wavelength, relatively weak pulse requires significantly higher interaction pressures than a loosely focused 800 nm pulse. This is the motivation for designing a vacuum system and gas source that can deliver high interaction pressures.

\subsubsection{XUV Reabsorption}
\label{sec:XUV_reabsorption}

\begin{figure}
	\centering
	\includegraphics[width=0.75\textwidth]{figures/chap1/Constant1999_fig1.pdf}
	\caption{1D absorption model from \cref{eqn:HHG_Nout_simple}.}
	\label{fig:Constant1999_fig1}
	% created with \Python Scripts\HHG_Phasematching-master\test\Constant_fig1.py
\end{figure}

We now consider the effects of XUV absorption on the phase matching process using the 1-dimension model introduced by Constant et. al. \cite{constantOptimizingHighHarmonic1999}. In doing so, we restrict ourselves to the on-axis emission of the $q$\textsuperscript{th} harmonic. That is, we consider only harmonics with a wave vector $k_q$ that is collinear to the fundamental ($k_0$). In this case, the number of photons emitted per unit time and area is:
\begin{equation}
N_{out} = \frac{\omega_q}{4 c \epsilon_0 \hbar} \left| \left[ \int_{0}^{L_{med}} \dd{z} \rho(z) A_q(z) \exp \left( - \frac{L_{med} - z}{2 L_{abs}}  \right) \exp \left( i \varphi_q(z) \right)  \right] \right|^2
\label{eqn:HHG_Nout}
\end{equation}
Here, $\rho(z)$ is the gas medium density, $A_q(z)$ is the amplitude of the harmonic response at frequency $\omega_q$ and $\varphi_q$ is its phase at the exit of the medium, which has length $L_{med}$. If we are using a loose focusing geometry, then the gas density and harmonic response amplitude are constant along the interaction volume: $\rho(z) = \rho$ and $A_q(z)=A_q$. With this restriction, \cref{eqn:HHG_Nout} evaluates to:
\begin{equation}
\begin{aligned}
N_{out} = & \\ \rho^2 A_q^2 & \frac{4L_{abs}^2}{1+4\pi^2(L_{abs}^2 / L_{coh}^2)} \left[ 1 + \exp\left(-\frac{L_{med}}{L_{abs}}\right) - 2 \exp\left(\frac{\pi L_{med}}{L_{coh}}\right) \exp\left(-\frac{L_{med}}{2L_{abs}}\right) \right]
\label{eqn:HHG_Nout_simple}
\end{aligned}
\end{equation}
Here, we use the notation $L_{coh} = \pi/\Delta k$ for the coherence length ($\Delta k = k_q - q k_0$) and $L_{abs} = 1/{\sigma \rho}$ for the absorption length.


\cref{eqn:HHG_Nout_simple} is plotted in \cref{fig:Constant1999_fig1}. In the case of no absorption ($L_{abs} \rightarrow \infty$), the harmonic yield grows as $L_{med}^2$. Otherwise, the optimized conditions are $L_{med} > 3 L_{abs}$ and $L_{coh} > 5 L_{abs}$.

Experimentally, $L_{med}$ is fixed by the geometry of the gas source, $L_{abs}$ is directly controlled by adjusting the backing pressure, and $L_{coh}$ is indirectly controlled by other parameters (gas source position relative to focus, focusing conditions, iris diameter, etc.). In \cref{sec:HHG_gas_sources}, we will apply this simple model to the gas sources available in our lab to maximize our HHG yield.

\textbf{to do:}

now, find out where you are on this plot for the different gas cell geometries. that is, free jet, LPC and HPC have set medium lengths. given their pressure performance, you can calculate the range of interaction pressures achievable by each HHG source, and therefore you can calculate the Labs for a specific generating gas (Ar, for example). having done that, you know the Labs and the Lmed, so you know the x-axis position. you still don't know the coherence length, but it's a start.

motivation: in the LPC, we can't see pressure rollover. this plot helps show why. (assuming the LPC and the free jet are still on the rising edge of the curves). this plot explains why.

talk about choice of generating gas (He vs Ar) in terms of critical ionization fraction and Ip, as well as cross section $A_q$. He, with its low reabsorption, is great for showing interaction pressure scaling, and you can blast it with lots of 800 nm intensity. but overall it has a lower yield than argon b/c of the cross section. Ar performs poorly at 800 nm b/c of the cooper minimum; at longer wavelengths the cutoff extends past the cooper minimum and Ar is a good choice.
\chapter{ANITA instrument}
\label{anita}

\section{ANITA payload}
\label{payload}

The {\bf AN}tarctic {\bf I}mpulsive {\bf T}ransient {\bf A}ntenna (\gls{anita}) is a NASA-sponsored long-duration balloon experiment with the primary goal of detecting \gls{uhe} neutrinos as broadband radio signals in the frequency range $200 - 1200\,\mbox{MHz}$. The \gls{anita}-4 payload at the NASA \gls{ldb} Facility is shown in Figure~\ref{my_anita}. The \gls{anita}-4 payload just prior to launch and after launch at its float altitude through a telescope is shown in Figures~\ref{balloon_filled} and ~\ref{launch_float}.
The most important parts of the \gls{anita} payload are its \gls{rf} antennas and its signal processing units (most of which are inside the Instrument Box). A peer-reviewed description of these can be found in a recent publication that I led as the corresponding author~\cite{tuff}. \href{https://www.sciencedirect.com/science/article/pii/S016890021830411X}{Click here to find the electronic version of this paper.} 

\begin{figure}
\centering
\includegraphics[width=1.0\textwidth]{figures/anita_thesis.jpg}
\caption{Linda Cremonesi and I during testing the GPS systems on ANITA-4 before its launch from near McMurdo Station, Antarctica. This shows the relative size of the instrument compared to humans. Picture credit: Steven Prohira.}
\label{my_anita}
\end{figure}

\begin{figure}
\centering
\includegraphics[width=1.0\textwidth]
{figures/launch_balloon.jpg}
\caption{Here NASA's balloon for the launch of ANITA-4 is being filled with Helium. When NASA takes the balloon out, one knows there will be a launch for real.}
\label{balloon_filled}
\end{figure}

\begin{figure}
\centering
\subfloat[ANITA-4 at launch]{
	\includegraphics[width=0.24\textwidth]
	{figures/launch_thesis.jpg}
	\label{launch}
}
\subfloat[ANITA-4 at float altitude]{
	\includegraphics[width=0.7\textwidth]
	{figures/anita_float_big.jpg}
	\label{anita_float_big}
}
\caption{Left: The ANITA-4 payload attached to its balloon just before launch. Right: When ANITA gets close to its float altitude of about $40$ km, one cannot see it from the ground with the naked eye. This is ANITA-4 through a telescope. Telescope picture credit: Steven Prohira.}
\label{launch_float}
\end{figure}

\subsection{Flight path and payload weight}

The \gls{anita} neutrino observatory is a NASA long-duration balloon-borne payload.
After its launch from the NASA \gls{ldb} Facility near McMurdo Station,
the Summer polar vortex winds keep the ANITA payload flying in roughly circular loops above the continent of Antarctica. 
A lighter payload is able to reach higher altitudes where the polar vortex is spatially tighter. 
This leads to a more favorable flight path which, in turn, increases the chances of a longer flight and increased livetime. 
Most importantly, this 
keeps the payload from 
venturing out over the ocean and becoming unrecoverable. 

There are strict weight restrictions on a balloon payload. 
This is why
the \gls{anita} gondola is made of hollow aluminum tubes connected by joints. 
The aluminum beams can be seen in Figure~\ref{hanging}. 
A need for a light payload informed the design of the \gls{tuff} boards as detailed
in Section~\ref{tuff}. 
This helps to keep the payload weight under $5000\,\mbox{lb}$. 
The ANITA-4 payload weighed $4526\,\mbox{lb}$. 
For the first time, the \gls{anita}-4 payload was able to reach above $40\,\mbox{km}$ altitude for part of its flight. 
Furthermore, the ANITA-4 payload was able to maintain a more favorable flight path compared to the \gls{anita}-3 payload, resulting in a longer flight of 27 days compared to 22 days in \gls{anita}-3.

\begin{figure}
\centering
\includegraphics[width=1.0\textwidth]{figures/anita_antenna_me.jpg}
\caption{Here I am standing next to one of the ANITA horn antennas before the hang test of the ANITA-4 mission at Columbia Scientific Balloon Facility in Palestine, TX. Picture credit: Jacob Gordon.}
\label{antenna}
\end{figure}

\subsection{Radio antennas and Phi Sectors}

\gls{anita} looks for \gls{uhe} neutrinos with \gls{rf} antennas.
Figure~\ref{antenna} shows myself standing next to one of these antennas during the integration and testing of \gls{anita}-4 at the Columbia Scientific Balloon Facility in Palestine, TX in July of 2016. Custom-built by Seavey Engineering, they
are $0.8\,\mbox{m}$ long on a side, quad-ridged and horn-shaped.
The antennas are broadband and highly-directional. 
They have an on-axis gain of $\sim10\,\mbox{dB with respect to isotropic gain}$.
The $3\,\mbox{dB}$ point of these antennas is $\sim30^{\circ}$.


There are 48 antennas on the \gls{anita}-4 payload.
They are mounted on the ANITA gondola covering $360^{\circ}$ in azimuth. 
%Each antenna has two perpendicular feeds allowing detection of horizontally polarized and vertically polarized signals. 
%There have been 32, 40, 48, and 48 antennas on the \gls{anita}-1, 2, 3, and 4 flights, respectively.
The antennas are arranged in three aligned rings of 16 antennas, termed the top, middle, and bottom rings. 
The top ring consists of two staggered sub-rings each having eight antennas. 
%The antennas are mounted on the payload's gondola which is made of aluminum for its light-weight. 
The \gls{fwhm} beamwidth of the antennas is approximately $45^{\circ}$. 
The antennas in the top ring are evenly spaced by $45^{\circ}$ in azimuth. 
The two sub-rings in the top ring are offset by $22.5^{\circ}$ for uniform coverage.
The antennas in the middle ring are evenly spaced by $22.5^{\circ}$.
The antennas in the bottom ring are evenly spaced by $22.5^{\circ}$.
All the antennas are angled downward by $10^{\circ}$ to preferentially observe signals coming from the ice as opposed to from the sky. 
Each group of three antennas in a vertical column, taking one antenna from each ring, forms a phi sector, viewing a $22.5^{\circ}$ region in azimuth.

The antennas are dually-polarized with a feed each for \gls{hpol} and \gls{vpol} signal.
\gls{rf} signal through each channel goes through the 
\gls{ampa} unit before entering the Instrument Box. 
There is an \gls{ampa} unit connected directly to the \gls{hpol} and \gls{vpol} outputs of each antenna. 
The \gls{ampa} contains a $200 - 1200\,\mbox{MHz}$ bandpass filter, followed by an approximately $35\,\mathrm{dB}$ Low Noise Amplifier (LNA). The \gls{ampa} performs the first-stage amplification of the incoming \gls{rf} signal. I led a camp (John Russell christened it ``Campa") at the ANITA headquarters, University of Hawaii, in September of 2016, to finish assembling these \gls{ampa} units for the \gls{anita}-4 flight. I show one of the 100 that I worked on assembling in Figure~\ref{ampa}. These even involved a bit of careful soldering and using the heat gun!

\begin{figure}
\centering
\includegraphics[width=0.52\textwidth]{figures/beams.jpg}
\caption{Andrew Ludwig and I working on integration and cabling of ANITA-4 at the NASA LDB facility near McMurdo Station, Antarctica. The aluminum beams that form the underlying structure of the payload can be seen here. Light aluminum beams help to abide by weight restrictions. Image credit: Nan Wang.}
\label{hanging}
\end{figure}

\begin{figure}
\centering
\includegraphics[width=1.0\textwidth]{figures/ampa.jpg}
\caption{Rare picture of the inside of an AMPA from ANITA-4.}
\label{ampa}
\end{figure}

\subsection{Instrument Box and Science Instrument Package}
\label{box_sip}

The Instrument Box of ANITA sits on the payload's deck. 
Most of the signal processing in \gls{anita} takes place inside the Instrument Box.  
Following the \gls{ampa} unit, the \gls{rf} signal travels through $12\,\mbox{m}$ of
LMR240 coaxial cable to the Instrument Box. 
Inside the Instrument Box, the signal first goes through second-stage amplification and notch filtering both performed by the \gls{tuff} boards in ANITA-4.
Then it passes through another set of bandpass filters before
being split into digitization and triggering paths. 
The triggering and digitization processes are detailed in~\cite{tuff}. 

The \gls{sip} also sits on the payload's deck. 
The SIP is powered and controlled by NASA.
It is used for flight control such as ballast release and flight termination. 
The SIP also provides a connection to the ANITA payload during flight through line-of-sight transmission, the Iridium satellite, and the Tracking and Data Satellite System (TDRSS). 
This allows us to monitor the payload continuously during the flight.
A small fraction of data (less than 1\%) is transferred from the payload through telemetry.
Commands to perform different functions, such as tuning a \gls{tuff} notch filter, 
%or altering a trigger threshold, 
can be sent to the payload in real time using the SIP connection. 

\subsection{Power}
\label{power}

There are unusual constraints on the total power budget of ANITA as it is a balloon payload. 
The ANITA-3 and ANITA-4 payloads operated on $\sim500\,\mathrm{W}$ and $\sim600\,\mathrm{W}$ respectively.
The payload is solar-powered by photovoltaic (PV) cells. 
One set of PV cells are on top of the gondola. 
These are managed by NASA and used to power the SIP. 
The other set is termed the ``drop-down PV array" and the PV cells in this set are arranged in eight 90-cell strings, laid out in an octagon around the bottom of
the payload. 
The drop-down PV array powers the Instrument Box. 
Before launch, they partially cover up the antennas in the bottom ring, as seen in Figure~\ref{anita}. 
After launch, the eight strings are remotely instructed to drop down by the SIP, which fires a servo to deploy them below the bottom ring of antennas. 

A charge controller distributes the output from the drop-down PV array to the payload as $24\,\mbox{V}$, using DC-DC converters 
to provide $12\,\mbox{V}$, $-12\,\mbox{V}$, $3.3\,\mbox{V}$ and $5\,\mbox{V}$ to various systems. 
The charge controller is also connected to a battery farm of $12\,\mbox{V}$ lead-acid batteries. 
Although there is daylight 24/7 in the Antarctic summer, the amount of power the PV array produces changes with the Sun's elevation during each 24 hour period. Thus, a battery farm is needed as backup. 
When the PV array is able to power the payload by itself, the charge controller charges the battery farm. 
This is in the Battery Box which is also placed on the deck. 

\subsection{GPS Systems and Heat dissipation} 
\label{gps_heat}

\gls{anita} data analysis relies on location and orientation information of the payload. 
\gls{anita} uses three GPS systems during flight: ADU5A, ADU5B and G12. 
These GPS antennas are located on top of the payload.
The ADU5 systems provide heading, pitch and roll information.
The G12 system updates absolute time on the flight computer through its Network Time Protocol (NTP) server. 
As backup to the GPS systems, there are four sun-sensor instruments, a magnetometer and accelerometer located on the deck. 

%\subsection{Heat dissipation}
At high altitudes of $35\,\mbox{km}$ and above, the primary method of heat loss is radiation. 
Thus, many parts of the payload are painted white to reflect sunlight and regulate temperature. 
Components producing large amounts of heat are connected to a teflon-coated, silver-tape-lined radiator plate on the Instrument Box.

\section{ANITA Signal Processing}

In this section we describe the signal processing chain for ANITA-4, and in particular
the steps that are relevant to understanding the role of the \gls{tuff} boards.
We will note when and where the ANITA-3 signal processing differed.
Note that much of the contents of this section is also covered in~\cite{tuff}. 
The \gls{rf} signal processing chain for ANITA-4 is illustrated in Figure~\ref{system}. 

\begin{figure}
\centering
\includegraphics[width=1.0\textwidth]{figures/A4_simple_system.png}
\caption{Simplified form of the signal processing chain in ANITA-4. A more detailed diagram can be found in \cite{tuff}. The blue solid arrow shows where the TUFF notch filters are in the chain. More details on the TUFF boards and notch filters are presented later in this chapter.}
\label{system}
\end{figure}

\subsection{Triggering}
\label{trigger}

In the triggering path, the \gls{rf} signals from both the \gls{vpol} and \gls{hpol} channels of a single antenna 
are passed through a $90^{\circ}$ hybrid (hybrids were absent in ANITA-3). 
The outputs from the $90^{\circ}$ hybrid are the left- and right- circular polarized
(LCP and RCP) components of the combined \gls{vpol} and \gls{hpol} signals from an antenna. 
The hybrid outputs are input to the SURF (Sampling Unit for \gls{rf}) high-occupancy \gls{rf} Trigger (SHORT) unit before being passed to the SURF board. 
Each SHORT takes four channels as its input. 
In a SHORT
channel, the \gls{rf} signal passes through a tunnel diode and an amplifier. 
The output of the SHORT is
approximately proportional to the square of the voltages
of the input \gls{rf} signal integrated over approximately $5\,\mbox{ns}$.
It is a measure of the power of the incoming signal and is typically a negative voltage.
The SHORT output is routed to a SURF trigger input where 
it enters a discriminator that compares this negative voltage in Digital-to-Analog Converter (DAC) counts to the output
of a software-controlled DAC threshold on the SURF, henceforth referred to as the SURF DAC threshold. 
The SURF DAC threshold is expressed in arbitrary units of DAC counts corresponding to voltages. Lower thresholds 
correspond to higher voltages and therefore, higher power of the incoming signal. 
The SURF DAC threshold can be changed during flight. Thresholds for the \gls{anita}-3 and -4 flights are shown in Figure~\ref{thresholds_fig}. 
During the ANITA-3 flight, CW interference overwhelmed the digitization system, forcing us to impose
frequent and large changes in the SURF DAC thresholds. 
%A comparison of SURF DAC thresholds between ANITA-3 and ANITA-4 is presented in Figure~\ref{thresholds}.
Note that the lower overall threshold for ANITA-4 is primarily due to the modified triggering scheme, 
which requires more overall coincidences between channels. 
The increased stability of the ANITA-4 thresholds, due to the CW mitigation schemes presented later, is clearly apparent.

\begin{figure}
\centering
\includegraphics[width=1.0\textwidth]{figures/anita3_threshold_time20_2.pdf}
\includegraphics[width=1.0\textwidth]{figures/anita4_threshold_time_2.pdf}
\caption{Thresholds during the ANITA-3 (top) and ANITA-4 (bottom) flights.}
\label{thresholds_fig}
\end{figure}


\paragraph{Trigger logic:}
Due to power and bandwidth limitations, ANITA is not able to constantly record data. 
Digitization of data only occurs when the trigger conditions are satisfied.
The ANITA-4 trigger consists of three triggering levels: Level~1, Level~2 and Level~3. 
The trigger requirements at each of these three levels is described below.

\paragraph{Level~1 trigger:}

The Sampling Unit for \gls{rf} (SURF) board issues the Level~1 trigger.
To form a Level~1 trigger, the SHORT outputs of the LCP and RCP channels from the same antenna 
are required to exceed the SURF DAC threshold within $4\,\mbox{ns}$. This LCP/RCP coincidence requirement 
was added to the ANITA-4 trigger to mitigate anthropogenic and thermal backgrounds.
The signals of
interest are known to be linearly polarized, whereas satellite emission is often circularly polarized
and thermal noise is unpolarized. In the presence of a continuous source of CW signal such as satellites,
the LCP/RCP coincidence may still allow a combination of circularly polarized satellite noise and the circularly polarized component of
thermal noise to satisfy the Level~1 trigger requirement. Therefore, the LCP/RCP coincidence aids in
reducing triggers induced by satellites but does not completely mitigate their effect.

\paragraph{Level~2 trigger:} 

The SURF board issues the Level~2 trigger. 
A Level~1 trigger opens up a time window.
If there are two Level~1 triggers in the same phi sector within the allowable time window, then a Level~2 trigger is issued. 
The allowable time window depends on which antenna had the first Level~1 trigger. 
Time windows of
$16\,\mbox{ns}$, $12\,\mbox{ns}$ and $4\,\mbox{ns}$ in duration are
opened up when a Level~1 trigger is issued in the bottom, middle and top ring respectively.
These time windows were chosen to preferentially select signals coming up from the ice. 
The Level~2 trigger decisions are passed from the SURF boards to a dedicated triggering board called the Triggering Unit for \gls{rf} (TURF).
The Level~2 trigger timing in ANITA-4 differed from that used in ANITA-3 as changes were made to further restrict
the allowed timing of the antenna coincidences to 
better match timing expected from an incoming plane wave.

\paragraph{Level~3 trigger:}

The TURF board issues the Level~3 trigger.
A field programmable gate array (FPGA) on the TURF board monitors Level~2 triggers.
A Level~3 trigger is issued by the TURF board when there are Level~2 triggers in two adjacent phi sectors within $10\,\mbox{ns}$. 
When there is a Level~3 trigger, the TURF board instructs 
the SURF board to begin digitization.

\subsection{Digitization}

The digitization of the signal is performed by the SURF board. 
There are twelve SURF boards, each containing four custom-built Application Specific Integrated
Circuits called \gls{lab}. 

\paragraph{\gls{lab} chip and digitization deadtime:}
ANITA-4 uses the third generation
of \gls{lab} chips that are described by Varner \textit{et al}. \cite{labrador}. 
Each \gls{lab} chip has a 260-element switched capacitor array (SCA) for each of its 9 input channels, with one channel used for timing synchronization.
The \gls{rf} signal entering a SURF gets split and fed into four parallel \gls{lab} chips (forming four ``buffers" for digitization). 
The SCAs sample waveform data at the rate of $2.6\,\mbox{GSa/s}$. 
At any moment, the charge stored in an SCA is a $100\,\mbox{ns}$ record of the signal voltage. 
This $100\,\mbox{ns}$ snapshot of the incoming plane wave is known as an ``event."
When a Level~3 trigger occurs, a single \gls{lab} chip stops sampling and is ``held.'' It then digitizes
the stored data, which is then read out by the flight computer, taking approximately $5-10\,\mbox{ms}$.
If all four \gls{lab} chips are held, the trigger is ``dead'' and the accumulated time when the trigger is dead is recorded as
digitization deadtime by the TURF board. 

\paragraph{Masking:}

During ANITA-3, digitization deadtime due to high levels of anthropogenic noise was reduced
by excluding
certain phi sectors from participating in the Level~3 trigger. 
This is called phi-masking. 
Alternatively, specific channels (each antenna has two channels) were excluded from participating in the Level~1 trigger. 
This is called channel-masking.
Together these are referred to as masking.
Because of CW interference by military communications
satellites, over half of the payload had to be masked
during most of the ANITA-3 flight. 
This strongly motivated the creation of the \gls{tuff}
boards with tunable, switchable notch filters. 
%A comparison of masking between \gls{anita}-3 and \gls{anita}-4 is presented in Figure~\ref{phimasking}. 

\paragraph{Data storage:}
All \gls{anita} data is stored on-board with less than 1\% of it transmitted to the ground during flight by telemetry. This is why payload recovery is critical. 
The primary storage devices are two
HGST UltraStar He6 disks, each with $6\,\mathrm{TB}$ capacity. 
These two Helium
drives contain identical copies of the data for redundancy in case of a drive failure.
Additionally, there are six $1\,\mathrm{TB}$ Solid State Drives for secondary data storage. 

\section{Tunable Universal Filter Frontend}
\label{tuff}

For ANITA-4, we built and deployed 16 \gls{tuff} boards (not counting spares) with 
six channels each for the 96 total full-band \gls{rf} channels of ANITA. 
Figure~\ref{system} shows, for a single \gls{rf} channel in ANITA-4, where 
the \gls{tuff} boards are in the signal processing chain. Details on these boards, their function and performance, as well as a portion of the contents of this section, are presented in~\cite{tuff}. 
%However, most of the images that I include in this thesis are not already shown in the publication.

\subsection{The problem: modulated continuous wave interference} 

The principal challenge of the \gls{anita} experiment is to distinguish neutrino signals from \gls{rf} noise. 
The two main sources of noise are thermal radiation by the Antarctic ice and anthropogenic noise, much of which is modulated \gls{cw} interference. 

While Antarctica itself is relatively free of \gls{cw} transmissions, except for bases of human activity, transmissions from geosynchronous satellites are continuously in view.
The average \gls{fwhm} beamwidth of the \gls{anita} antennas is approximately $45^{\circ}$.
Although the \gls{anita} antennas are canted downward by $10^{\circ}$, the beam of the antennas extends to horizontal from the perspective of the payload and
above.  
The Antarctic science bases, the most prominent being McMurdo and South Pole Station, are more radio-loud than the rest of the continent, producing \gls{cw} interference, for example, in the $430-460\,\mbox{MHz}$ band. 

\begin{figure}
\centering
\includegraphics[width=1.0\textwidth]{figures/anita3spectra_wais_replot.pdf}
\caption{This averaged (over 2 mins) power spectrum shows the two CW peaks caused by military satellites that greatly reduced the instrument livetime (instrument livetime was only 31.6\%) of the ANITA-3 flight.}
\label{cw_peaks}
\end{figure}

\gls{cw} interference due to military satellites has affected all \gls{anita} flights.
\gls{anita}-1 (Dec. 2006 - Jan. 2007) and \gls{anita}-2 (Dec. 2008 - Jan. 2009) observed \gls{cw} interference primarily in the $240-270\,\mbox{MHz}$ band, peaking
at $260\,\mbox{MHz}$. This frequency range is predominantly used by the aging Fleet Satellite (FLTSAT) Communications System 
and the Ultra High Frequency Follow-On (UFO) System, both serving the 
United States Department of Defense since year 1978 and 1993 respectively.
In addition to \gls{cw} interference at $260\,\mbox{MHz}$, ANITA-III (Dec. 2014 - Jan. 2015) observed \gls{cw} interference at $375\,\mbox{MHz}$
which is thought to be due to
the newer Mobile User Objective System (MUOS) satellites that were launched during the period from Feb. 2012 - June 2016 \cite{milsat}.
The \gls{cw} signals generate events with excess power in left circular polarization (shown for the first time in Stafford's thesis~\cite{samStaffordThesis}) above the horizon, in approximately stationary positions.

The \gls{anita}-3 experiment was most affected by \gls{cw} interference due to military satellites. The first and second peaks in the power spectrum shown in Figure~\ref{cw_peaks} were present during all and about half, respectively, of the \gls{anita}-3 flight.
Due to the design
of the \gls{anita}-1 and \gls{anita}-2 trigger, which required coincidences among different frequency bands, the \gls{cw} interference did not overwhelm
the acquisition system. 
However, \gls{anita}-3 was redesigned for improved sensitivity and based its trigger decisions on full-bandwidth ($200 - 1200\,\mbox{MHz}$) signals. 
The modulation present in the \gls{cw} interference produced trigger rates far in excess of the digitization system's readout capabilities ($\sim50\,\mbox{Hz}$) for thresholds comparable to those used in previous flights. 
Thus, the \gls{anita}-3 experiment was susceptible to digitization deadtime throughout the flight. 

The lesson learned from the \gls{anita}-3 flight was
that a new method of mitigation of \gls{cw} signal was critical for the \gls{anita}-4 flight. 
Before \gls{anita}-4, the available methods to reduce digitization deadtime were masking
and decreasing thresholds when in the presence of higher levels of noise. 
A decrease in thresholds corresponds to higher power of the incoming signal. 
Masking and decreasing thresholds
come at the cost of instrument livetime~\cite{tuff} and sensitivity to neutrinos, respectively. 
%Both of these methods come at a high price.
For about 90\% of the time during the \gls{anita}-3 flight,
masking was used 
%during noisy periods 
to veto triggers from
over 
half of the payload field-of-view to keep the
trigger rate at or below $50\,\mbox{Hz}$. 
This significantly lowered the total instrument livetime. 
%Decreasing thresholds led to reduction of sensitivity to neutrinos during noisy periods. 
For \gls{anita}-4, the \gls{tuff} boards were built with
tunable notch filters to restore triggering efficiencies 
in the presence of CW interference. 
Additionally, the $90^{\circ}$ hybrids, previously deployed in \gls{anita}-1 as described in our design paper \cite{instrPaper}, were added to the \gls{anita}-4 trigger system by requiring a coincidence between left- and right- circularly polarized signals.
 
\begin{figure}
\centering
\includegraphics[width=1.0\textwidth]{figures/tuff_construction.png}
\caption{During the construction of the TUFF boards at OSU (May - July of 2016). The picture of myself holding one of the boards gives an idea for their size and shape. Clearly, building these boards made me very happy.}
\label{tuff_construction}
\end{figure}

\subsection{Design and construction}

In April of 2016, NASA gave the \gls{anita} collaboration the go ahead to attempt a launch of the \gls{anita}-4 mission at the end of that same year. From May - July of 2016, I worked on constructing and testing the \gls{tuff} boards. Constructing them involved soldering several thousand parts on to the boards. This was done by a small team at OSU, including myself, Jacob Gordon and Michael Kovacevich. Patrick Allison designed the boards and supervised our work. Testing of the boards was done in different stages and involved frequent measurement of the \gls{tuff} response using the network analyzer, making measurements of the board's current, capacitance, etc. with the multimeter, and performing experiments using the thermal and vacuum chambers. Figure~\ref{tuff_construction} shows myself holding a \gls{tuff} board and standing next to a freshly soldered batch of \gls{tuff} boards.

\begin{figure}
\centering
\includegraphics[width=1.0\textwidth]{figures/tuff_case_screws.jpg}
\caption{Pairs of TUFF boards were enclosed within aluminum cases with RF padding on the inside. The enclosures were held shut with the help of the screws shown here. Even a slight problem with the case design could make it very difficult to put the screws in or take them out. In fact, these screws became the bane of our existence during integration and testing of ANITA-4, and demonstrated how important it was to get the design of the cases right. Thanks to Christian Miki for designing the case.}
\label{case_screws}
\end{figure}

The design of the \gls{tuff} board was affected by the low power budget of \gls{anita} as well as the weight and size restrictions of a balloon mission, as described in Section~\ref{payload}. 
The \gls{tuff} boards needed to be low-power, compact and light. %Figure~\ref{tuff_channel} shows a single \gls{tuff} channel next to a quarter USD coin for size comparison. 
A single channel is about twice the size of a quarter dollar coin. 
Each printed circuit board has four layers of copper with an FR-4 dielectric material. 
%This is a composite made of woven fiberglass cloth with an epoxy resin binder that is flame resistant.
The \gls{tuff} boards operate on $3.3\,\mathrm{V}$ and $4.7\,\mathrm{V}$ power sources 
provided by
a MIC5504 from Microchip Technologies Inc. and a ADM7171 from Analog Devices Inc. Both
voltage regulators draw from a $5\,\mathrm{V}$ source 
supplied by the DC/DC unit in the ANITA
Instrument Box. 
A single \gls{tuff} channel consumes only $330\,\mathrm{mW}$ of power. 
The total power consumed by the ANITA payload is approximately $800\,\mathrm{W}$. 

Two \gls{tuff} boards were assembled into a final 
12-channel aluminum housing as shown in Figure~\ref{case_screws}. This provides heat-sinking, structural support, and \gls{rf} isolation. 
Two of these 12-channel modules were placed inside an 
\gls{irfcm} inside the Instrument Box of ANITA. Figure~\ref{irfcm} shows the inside of an IRFCM. Each \gls{tuff} channel has four main components which are described in the following subsections. 

\begin{figure}
\centering
\includegraphics[width=1.0\textwidth]{figures/irfcm_thesis.jpg}
\caption{Rare picture of the inside of an Internal Radio Frequency Conditioning Module (IRFCM) holding two TUFF modules and a TUFF master. There are four total IRFCMs.}
\label{irfcm}
\end{figure}

\subsection{Amplifiers and bias tee} 

There are two amplifiers connected in series that together 
produce second-stage \gls{rf} power amplification of approximately $45\,\mathrm{dB}$. 
%The gain of a \gls{tuff} channel, as measured in the lab, is shown in Figure~\ref{s21}. 
AMP~1 is a BGA2851 by NXP Semiconductors and AMP~2 is an ADL5545 by Analog Devices. 
There is an attenuator producing $1\,\mathrm{dB}$ of 
attenuation to the \gls{rf} signal as it leaves
AMP~1 and before it enters AMP~2.
The BGA2851 provides a gain of $24.8\,\mathrm{dB}$ at $950\,\mbox{MHz}$. 
It has a noise figure~of $3.2\,\mathrm{dB}$ at $950\,\mbox{MHz}$. 
It consumes $7\,\mathrm{mA}$ of current at a supply voltage of $5\,\mathrm{V}$, 
or $35\,\mathrm{mW}$ of power.
The ADL5545 provides a gain of $24.1\,\mathrm{dB}$ with broadband operation from $30-6000\,\mbox{MHz}$.
Out-of-band power at frequencies above $2\,\mbox{GHz}$ is suppressed by a filter on each \gls{tuff} channel. 
Additionally, there are band-pass 
filters immediately after the \gls{tuff} boards in the signal processing chain allowing power only in the frequency range $200 - 1200\,\mbox{MHz}$. 
The ADL5545 has a noise figure~of $2.9\,\mathrm{dB}$ at $900\,\mbox{MHz}$ 
and a $1\,\mathrm{dB}$ compression point (P1dB) of $18.1\,\mathrm{dBm}$ at $900\,\mbox{MHz}$. 
It consumes $56\,\mathrm{mA}$ of current at a supply voltage of $5\,\mathrm{V}$, or $300\,\mathrm{mW}$ of power. 
Thus, this amplifier consumes the majority of the power required by a single \gls{tuff} channel. 

There is a bias tee on each \gls{tuff} channel that 
remotely powers the \gls{ampa} unit at the other end of the coaxial cable connecting an AMPA and that channel. 
%A bias tee is composed of an inductor and a capacitor in series. 
It consists of a 4310LC inductor by Coilcraft in series with a $0.1\,\mathrm{\mu F}$ capacitor. 
The inductor delivers DC to the AMPA unit while the capacitor prevents DC from passing through to the signal path of the \gls{tuff} channel. 
The bias tee allows \gls{rf} signal traveling from the AMPA unit through the coaxial cable to pass 
through to the rest of the signal path of the \gls{tuff} channel. 

\subsubsection{Notch filters} 

There are three tunable, switchable notch filters 
for mitigation of CW noise at the default frequencies of $260\,\mbox{MHz}$ (Notch~1), $375\,\mbox{MHz}$ (Notch~2) 
and $460\,\mbox{MHz}$ (Notch~3). 
The measured as well as simulated gain, phase and group delay of a \gls{tuff} channel, with the first two notch filters activated (most common configuration used during the \gls{anita}-4 flight) and all filters de-activated, is shown in Figure~\ref{tuff_measured_model}. 
The \gls{tuff} notches were able to achieve a maximum attenuation of approximately $13\, \mathrm{dB}$, and were implemented as a simple RLC trap, with the resistance $R$
originating from the parasitic on-resistance of a dual-pole, single-throw
\gls{rf} switch and the DC resistance of the remaining components. This is approximately $6 - 7\,\mathrm{\Omega}$. 
The inductance $L$ is fixed at $56\,\mathrm{nH}$. The capacitance $C$ is a
combination of a fixed capacitor and a PE64906 variable capacitor from Peregrine Semiconductor. Simulations using the device model of the variable capacitor also suggested that the mounting pads of the components contribute $\sim~0.6\,\mbox{pF}$ of parasitic capacitance.

With the 
tuning capability of the variable capacitor, the resonant frequency of the RLC circuit was 
modified during flight to dynamically mitigate CW interference. 
The variable capacitor in a notch can be 
tuned in 32 discrete steps of $119\,\mathrm{fF}$ in the range $0.9-4.6\,\mathrm{pF}$ and for 
each notch, is connected in series or parallel with a constant capacitance. 
For Notch~1, the variable capacitor is 
in parallel with a $1.8\,\mathrm{pF}$ capacitor. For Notches~2 and 3, the variable capacitor is in 
series with a $12.0\,\mathrm{pF}$ (Notch~2) and a $1.5\,\mathrm{pF}$ (Notch~3) capacitor for 
increased tuning capability. 
%Figure~\ref{circuit} shows a simplified circuit diagram.

%The \gls{tuff} notches were able to achieve a maximum attenuation of $\sim13\, \mathrm{dB}$. With the 
%tuning capability of the variable capacitor, the resonant frequency of the RLC circuit 
%could be modified during flight to dynamically mitigate CW interference. 

%\begin{figure}[ht]
%\centering
%\subfigure{
%	\includegraphics[width=0.8\textwidth]{notchesOff.pdf}
%	\label{s21off}
%}
%\subfigure{
%	\includegraphics[width=0.8\textwidth]{notchesOn.pdf}
%	\label{s21on}
%}
%\caption[]{Gain of a \gls{tuff} channel as measured in the lab with all notches de-activated (top) and all notches activated at their default frequencies (bottom).}
%\label{s21}
%\end{figure}

\begin{figure}
\centering
\includegraphics[width=1.0\textwidth]{figures/12measured_model_gain_phase_gd.pdf}
\caption{The gain, phase and group delay as measured and simulated for a TUFF channel with the first two notch filters activated (most common configuration used during the ANITA-4 flight) and all notch filters de-activated.}
\label{tuff_measured_model}
\end{figure}

\subsection{Microcontroller}

We use an ultra-low-power microcontroller, specifically a MSP430G2102 by Texas Instruments.
This features a powerful 16-bit Reduced Instruction Set Computing (RISC) central processing unit (CPU). 
There are five low-power modes optimized for extended battery life. 
The active mode consumes $220\,\mu\mbox{A}$ at $1\,\mbox{MHz}$ and $2.2\,\mathrm{V}$. 
The standby mode consumes only $0.5\,\mu\mbox{A}$ and the RAM retention-off mode consumes $0.1\,\mu\mbox{A}$.
The digitally-controlled oscillator allows wake-up from low-power modes to active mode in less than 
$1\,\mu\mbox{s}$. 

During the ANITA-4 flight, commands could be sent using the SIP connection to set the 
state of the variable capacitor of each \gls{tuff} notch filter via the microcontroller 
of that channel. 
This was done in real time if a re-tune of a notch filter was necessary to mitigate CW interference.
Commands could be sent to de-activate or activate a notch filter using the switch associated with each notch. 
Each microcontroller has the capability to communicate over universal serial communication interface.


\section{Impact of the TUFF boards}

The \gls{tuff} boards had a large impact on the livetime of \gls{anita}. 
There are two types of livetime in ANITA, which are described below.

\paragraph{Digitization livetime} 

In \gls{anita}, deadtime due to digitization by all four \gls{lab} chips of the SURF board is recorded by 
the TURF board, as illustrated in Figure~\ref{system}. 
This deadtime is recorded as a fraction of a second. Digitization livetime per second can be 
obtained by subtracting this from one. 
Increasing the digitization livetime increases the probability of receiving \gls{rf} signal due to an \gls{uhe} neutrino. 

\paragraph{Instrument livetime} 

At any given time, the digitization livetime multiplied by the fraction of unmasked phi sectors (after accounting for channel-masking) gives us the instrument livetime per second. 
In other words, instrument livetime accounts for the fraction of observable ice in azimuth after accounting for masking. 

\subsection{3x instrument livetime}

The most significant impact of the \gls{tuff} boards was the great reduction in the need for masking to mitigate noise during the \gls{anita}-4 flight as compared to \gls{anita}-3. This can be seen in Figure~\ref{masking}. The striking reduction in masking and increase in digitization livetime, as a result of implementing the \gls{tuff} notch filters, contributed to over 91.3\% instrument livetime in \gls{anita}-4 compared to the 31.6\% in \gls{anita}-3.
The performance and impact of the \gls{tuff} boards are described in detail in ~\cite{tuff}, along with visuals comparing the digitization and instrument livetime, thresholds and masking in \gls{anita}-3 and \gls{anita}-4. 

\begin{figure}
\centering
\includegraphics[width=1.0\textwidth]{figures/masking_compare.png}
\caption{Fractional masking implemented in the ANITA-4 and ANITA-3 (faded) flights as a function of time. The TUFF notch filters helped to reduce the need for masking and thereby, tripled the instrument livetime of the experiment.}
\label{masking}
\end{figure}

\begin{figure}
\centering
\includegraphics[width=1.0\textwidth]{figures/bag_for_palestine.jpg}
\caption{Bonus: This is the bag I packed for my trip to Palestine, TX, for the hang test of ANITA-4. I packed my own power supply. TUFF boards needed to be tested in Palestine for the integration and hang test, and they needed power. I thought it pertinent to carry my own as other folks' power supplies simply cannot be trusted, especially in challenging situations. This is from Jim Beatty's stash of lab equipment that he may let you borrow for such occasions. The highlight is I got this through airport security by telling the officers all about ANITA!}
\label{palestine_bag}
\end{figure}
\chapter{XUV Light Source Design and Apparatus Performance}

\section{Introduction}

Compared to RABBITT measurements, condensed matter transient absorption experiments require a high XUV photon flux. First, the sample thickness is usually chosen such that the XUV transmission is roughly 50\% near the spectral feature of interest. This optical density represents a compromise between the incompatible goals of having a strong ground state absorption (enabling the detection of small changes in the optical density) while simultaneously avoiding the noise floor of the detector (which is required for good statistics). Second, a high XUV flux will reduce the number of laser shots required for a given data point, which in turn reduces the total IR flux on the sample and minimizes sample heating. Finally, a high flux reduces the overall time required to complete an experiment. This increases data fidelity by reducing the effects of unavoidable experimental noise sources such as long-term laser drift (either pointing or energy) and environmental changes caused by the building's HVAC system.

This chapter will detail the development of bright XUV sources which were required for ATAS experiments. It will also quantify the performance of the available XUV sources and the TABLe beamline as a whole.

\section{HHG Gas Sources}
\label{sec:HHG_gas_sources}

This section will discuss the HHG gas sources used to generate harmonics for ATAS experiments. For each gas source, we will describe the device, model the gas flow and discuss their XUV output in the context of the physical principles discussed in \cref{sec:HHG}.

\subsection{Introduction}

Recalling the arguments of \cref{sec:phase-matching}, there exists an optimal phase matching pressure $P_{\textrm{opt}}$ for which the XUV flux is maximized. According to \cref{eqn:phase_matching_density}, $P_{\textrm{opt}}$ is proportional to the square of the fundamental wavelength. From \cref{eqn:HHG_Nout_2}, we expect the harmonic yield of a perfectly phase matched HHG process to scale as the square of the pressure-length product, $(PL)^2$. Additionally, we expect the harmonic yield to scale with the fundamental wavelength as $\lambda^{-(5-6)}$, as explained in \cref{sec:single-atom-response}. These factors indicate that a high interaction pressure is critical for a successful ATAS experiment.

Experimentally, the interaction pressure is a consequence of the gas flow dynamics and the design of the gas delivery device. The interaction pressure can be increased by increasing the backing pressure into the device, but this is limited by the finite pumping speed of the vacuum system. Significant improvements to both the interaction pressure $P$ and the interaction length $L$ can be made by modifying the design of the gas delivery device. The simplest and most common gas delivery device (a free expansion nozzle) cannot reach $P_{\textrm{opt}}$ at $\lambda_1 = 800 \ \textrm{nm}$ before the vacuum system is overwhelmed. Since most of our experiments are performed at longer wavelengths using the signal output of the TOPAS, and because ATAS experiments require a high XUV flux, more advanced gas delivery systems were required.

%Depending on the energy of the spectral feature, obtaining a high photon flux can range from trivial to challenging. There are many (usually interdependent) experimental parameters (gas type, interaction pressure and length, wavelength, intensity, confocal parameter, focal position relative to gas source, etc.) that can be tuned to optimize photon flux. Physically, these parameters can change the microscopic single atom response, the macroscopic coherent addition of dipole emitters (via phase matching), or both. Each experiment will usually require a unique combination of experimental settings to achieve a usable light source. For example, optimizing the harmonic yield at 100 eV for a Si L-edge measurement will usually come at the expense of harmonics yield in the 30-50 eV range, which are used to measure the transition metal M-edges.

%In general, an experimentalist has neither perfect knowledge nor control over all the variables that contribute towards phase matching. Setting aside the complicated topic of phase matching, the one dimensional on-axis phase matching model\cite{constantOptimizingHighHarmonic1999} shows that the photon flux is proportional to the square of the pressure-length product of the interaction gas. That is, so long as we can remain phase matched and below the critical phase matching pressure\cite{popmintchevPhaseMatchingHigh2009}, we can universally increase the harmonic flux of our experiments by increasing the pressure-length product.

%Unfortunately, one cannot ignore phase matching. Oftentimes, the spectral feature of interest lies beyond the harmonic cutoff when using the more convenient shorter wavelengths. In this case, the fundamental wavelength is increased to extend the cutoff (which scales as $\lambda^2$). However, the critical phase matching pressure also scales as $\lambda^2$ \cite{popmintchevPhaseMatchingHigh2009}, and the single atom response scales as $\lambda^{-(5-6)}$ \cite{tateScalingWavePacketDynamics2007}. These two combined effects result in a dramatically decreased photon flux if intensity and pressure are kept constant with increasing wavelength, often to the point that the resulting flux is insufficient for a transient absorption experiment, even though your cutoff has been extended to the proper energy. While some of the flux can be recovered by increasing the backing pressure of the continuous free expansion nozzle, the generation chamber's finite pumping speed limits the efficacy of pressure tuning at the longer wavelengths. Even at 800 nm, the maximum backing pressure of the continuous free expansion nozzle results in an interaction pressure below the critical phase matching pressure. Practically speaking, the continuous free expansion gas nozzle is not suitable for transient absorption experiments using the signal wavelengths ($\lambda > 1.6 \ \mu m$) or with spectral features greater than the aluminum edge at 72 eV.

Providing the lab with a brighter harmonic source was the ultimate goal of the high pressure cell, and for the most part this goal was achieved. Below, we will review the basic design considerations, drawbacks and advantages of the four main types of gas sources used in this thesis: the free expansion nozzle, the low pressure cell (LPC), the high pressure cell (HPC) and the Amsterdam pulsed piezovalve. A primer on how to install and use the high pressure cell can be found in Appendix \ref{appendix:TABLe_manual}.

\subsection{Free Expansion Nozzle}
\label{sec:free-expansion-nozzle}
%outline of gas jet physics:
%- why supersonic? basic physics argument
%- outline of derivation to get to the T/P/rho relationships
%- outline of derivation to get to the centerline equations
%- mach disk location, description, thickness
%- derivation of gas nozzle throughput T
%
%\begin{figure}
%	\centering
%	\includegraphics[width=0.5\textwidth]{figures/chap3/gas_expansion.PNG}
%	\caption{The structure of the supersonic gas plume after leaving a gas nozzle. This figure was taken from Ref \cite{millerFreeJetSources1988}.}
%	\label{fig:gas_expansion}
%\end{figure}

%\begin{figure}
%	\centering
%	\includegraphics[width=0.75\textwidth]{figures/chap3/off_axis_density.pdf}
%	\caption{Off-axis mass density $\rho(x,y)$ for various on-axis distances $x$. For an aperture size of $d = 200 \ \mu$m, the FWHM of the plume density is $120 \ \mu$m at $x = 100 \ \mu$m.}
%	\label{fig:off_axis_density}
%	% \Python Scripts\HPC\HPCvsLPC.py
%\end{figure}

%\begin{figure}
%	\centering
%	\includegraphics[width=0.5\textwidth]{figures/chap3/Scoles_Fig25.pdf}
%	\caption{Centerline Mach number versus distance in nozzle diameters for 2D (planar) and 3D (axisymmetric) geometries, calculated using \cref{eqn:Scoles_centerline2.2}.}
%	\label{fig:scoles_mach}
%\end{figure}
%
%\begin{figure}
%	\centering
%	\includegraphics[width=0.5\textwidth]{figures/chap3/Scoles_Fig23.pdf}
%	\caption{Free jet centerline properties versus distance in nozzle diameters for helium gas ($\gamma$=5/3, W=4). Mach number is calculated using \cref{eqn:Scoles_centerline2.2}, and the centerline properties are calculated using \cref{eqn:mach_properties}. Velocity $V$ is scaled by terminal velocity $V_{\infty}$; temperature $T$, number density $n$ and pressure $P$ are normalized by source stagnation values $T_0$, $n_0$, $P_0$.}
%	\label{fig:scoles_centerline}
%\end{figure}

We use an \textit{in vacuo} gas nozzle to deliver a localized plume of gas near the IR focus. Generally, when gas flows from a high pressure region ($P_0$) to a low pressure region ($P_b$) through a small aperture of diameter $d$, a supersonic plume may form in the low pressure region. If the pressure ratio $P_0/P_b$ exceeds a critical value $G$, given by
\begin{equation}
G \equiv ((\gamma+1)/2)^{\gamma/(\gamma-1)} \le 2.1,
\label{eqn:G_factor}
\end{equation}
then the gas flow at the aperture will be equal to the speed of sound, and the pressure will be equal to $P_0 / G \approx P_0/2$. The highest chamber pressures in our experiments are on the order of $P_b \approx 10$ mTorr, and typical backing pressures for harmonic generation generally exceed 50 Torr, so we are always operating with a supersonic jet. The on-axis spatial extent of the supersonic gas plume is estimated by the Mach disk location, $x_M$:
\begin{equation}
x_M / d = 0.67 \sqrt{P_0/P_b}
\label{eqn:Mach-disk}
\end{equation}
For a chamber pressure of 3 mTorr and a backing pressure of 450 Torr, ${x_M = 260d = 51.9 \ \textrm{mm}}$ for a 200 $\mu$m diameter aperture. As will be shown below, our laser-gas interaction region is well within the structure of the gas jet.

\begin{table}[]
	\centering
	\begin{tabular}{lllllllll}
		\hline
		\multicolumn{1}{c}{Source} & \multicolumn{1}{c}{$j$} & \multicolumn{1}{c}{$\gamma$} & \multicolumn{1}{c}{$C_1$} & \multicolumn{1}{c}{$C_2$} & \multicolumn{1}{c}{$C_3$} & \multicolumn{1}{c}{$C_4$} & \multicolumn{1}{c}{$A$} & \multicolumn{1}{c}{$B$} \\ \hline
		3D                         & 1                     & 5/3                          & 3.232                     & -0.7563                   & 0.3937                    & -0.0729                   & 3.337                & -1.541                \\
		3D                         & 1                     & 7/5                          & 3.606                     & -1.742                    & 0.9226                    & -0.2069                   & 3.190                 & -1.610                \\
		3D                         & 1                     & 9/7                          & 3.971                     & -2.327                    & 1.326                     & -0.311                    & 3.609                 & -1.950                \\
		2D                         & 2                     & 5/3                          & 3.038                     & -1.629                    & 0.9587                    & -0.2229                   & 2.339                 & -1.194                \\
		2D                         & 2                     & 7/5                          & 3.185                     & -2.195                    & 1.391                     & -0.3436                   & 2.261                 & -1.224                \\
		2D                         & 2                     & 9/7                          & 3.252                     & -2.473                    & 1.616                     & -0.4068                   & 2.219                 & -1.231               
	\end{tabular}
	\caption{Gas parameters used in \cref{eqn:Scoles_centerline2.2}. Table recreated from Ref \cite{millerFreeJetSources1988}.}
	\label{tbl:Scoles_gas_params2.2}
\end{table}

The physics of supersonic gas flow have been discussed at length in the literature, so we will only go over the revelant highlights \cite{millerFreeJetSources1988}. The ratio of the velocity of the gas $V$ to the speed of sound $a$ is called the \textit{Mach number} $M=V/a$. It can be shown that all thermodynamic parameters within the supersonic structure (density, pressure, velocity and temperature) can be expressed in terms of the Mach number and the heat capacity ratio $\gamma$. For harmonic generation, we are primarily concerned with the on-axis ($y=0$) mass density $\rho$ and pressure $P$:
\begin{subequations}
	\label{eqn:mach_properties}
	\begin{align}
	\frac{\rho}{\rho_0} = \frac{n}{n_0} = \left(\frac{T}{T_0}\right)^{1/(\gamma-1)} &= \left(  1 + \frac{\gamma-1}{2} M^2 \right)^{-1/(\gamma-1)} \label{eqn:mach_rho} \\
	\frac{P}{P_0} = \left(\frac{T}{T_0}\right)^{\gamma/(\gamma-1)} &= \left(  1 + \frac{\gamma-1}{2} M^2 \right)^{-\gamma/(\gamma-1)} \label{eqn:mach_pressure}
	\end{align}
\end{subequations}
%\begin{align}
%\frac{\rho}{\rho_0} = \frac{n}{n_0} = \left(\frac{T}{T_0}\right)^{1/(\gamma-1)} &= \left(  1 + \frac{\gamma-1}{2} M^2 \right)^{-1/(\gamma-1)} \\
%\frac{P}{P_0} = \left(\frac{T}{T_0}\right)^{\gamma/(\gamma-1)} &= \left(  1 + \frac{\gamma-1}{2} M^2 \right)^{-\gamma/(\gamma-1)}
%\label{eqn:mach_properties}
%\end{align}
Here, $\rho_0$ is the mass density at the nozzle aperture ($x=0$), and $n$ is the number density. The Mach number is found by solving the fluid mechanics equations dealing with the conversation of mass, momentum and energy for a given nozzle geometry. For a complete discussion, see \cite{millerFreeJetSources1988}. Below we present the on-axis result, which is an analytic fit to a numerical solution of the thermodynamic equations:
\begin{subequations}
	\label{eqn:Scoles_centerline2.2}
	\begin{align}
	% eqns from table 2.2 of scoles, page 23
	\frac{x}{d} > 0.5&: &&M = \left( \frac{x}{d} \right)^{(\gamma-1)/j} \left[ C_1 + \frac{C_2}{\left(\frac{x}{d}\right)} + \frac{C_3}{\left(\frac{x}{d}\right)^2} + \frac{C_4}{\left(\frac{x}{d}\right)^3} \right] \label{eqn:Scoles_centerline1} \\
	0 < \frac{x}{d} < 1.0&: &&M = 1.0 + A \left( \frac{x}{d} \right)^2 + B \left( \frac{x}{d} \right)^3 \label{eqn:Scoles_centerline2}
	\end{align}
\end{subequations}
The fitting coefficients for \cref{eqn:Scoles_centerline2.2} are listed in \cref{tbl:Scoles_gas_params2.2}. We can see that $M$ scales with powers of $x/d$, the number of nozzle diameters away from the nozzle aperture.
%Likewise, the off-axis density $\rho(x,y)$ is given by:
%\begin{align}
%\frac{\rho(x,y)}{\rho(x,0)} &= \cos^2 \theta \cos^2 \left( \frac{\pi \theta}{2 \phi} \right) \\
%\textrm{with} \quad \tan \theta &\equiv \frac{y}{x}
%\label{eqn:off-axis-density}
%%note: this eqn is only valid for x > (x/d)_{min}. it isn't valid for HHG experiments.
%\end{align}
%where $y$ is the distance from the centerline axis, and $\phi$ is a gas constant with values ${\phi = 1.365, 1.662}$, and $1.888$ for ${\gamma = 5/3, 7/5}$ and $9/7$, respectively.

The gas nozzle throughput $\hat{T}$ is proportional to the area of the aperture and the backing pressure:
\begin{equation}
\hat{T} \ (\text{Torr} \cdot \text{l}/\text{s}) = c \left(\frac{T_C}{T_0} \right)\sqrt{\frac{300}{T_0}} P_0 d^2
\label{eqn:nozzle_thruput}
\end{equation}
where $C$ is a gas constant\footnote{Values of $C$ for common species are listed here: 45 [He], 20 [Ne], 14 [Ar], 16 [N\textsubscript{2}] in l/cm\textsuperscript{2}/s. For a full table of values, see Table 2.5 in \cite{millerFreeJetSources1988}.}, $T_C$ and $T_0$ are the vacuum chamber and backing temperatures, respectively, and $d$ is the nozzle diameter in cm. Ignoring the effect of the generation chamber's vacuum aperture, we can estimate the operating pressure of the generation chamber using the following equation \cite{hablanianHighvacuumTechnologyPractical1997}:
\begin{equation}
P_b \ (\text{Torr}) = \frac{\hat{T}}{S}
\end{equation}
where $S$ is the pumping speed of the turbo pump in liters per second. To avoid overloading our turbo pumps, we are typically limited to operating pressures below 5 - 10 mTorr.

\begin{figure}
	\centering
	\includegraphics[width=0.5\textwidth]{figures/chap3/gas_nozzle.png}
	\caption{Rendering of the continuous free expansion nozzle. Gas flows from the base of the nozzle (bottom) and out of the 200 $\mu$m aperture (top). The top surface is beveled so the nozzle can be brought closer to the IR focus without clipping the beam.}
	\label{fig:gas_nozzle}
\end{figure}

The basic design of our continuous free expansion nozzle is shown in \cref{fig:gas_nozzle}. The nozzle is an aluminum cylinder with a small diameter hole drilled into the top surface. Gas is delivered to the aperture via a universal gas receiver (not shown), which attaches to base of the nozzle. To reduce the gas load on the pumps, we used a $200 \ \mu m$ diameter aperture, which was the smallest size hole the machine shop could readily drill into aluminum. A 200 $\mu$m diameter aperture backed with 180 Torr of argon will deliver a gas throughput of approximately {1 Torr $\cdot$ l/s}. With a pumping speed of $S = 1000$ L/s, the generation chamber pressure will be around $P_b = 1$ mTorr. For a monatomic gas, $G = 2.05$, and the pressure at the nozzle aperture is $P_0/G \approx 87 \textrm{ Torr}$.

\begin{figure}
	\centering
	\includegraphics[width=0.75\textwidth]{figures/chap3/M_rho_P_vs_x.pdf}
	\caption{On-axis Mach number $M$, mass density $\rho$ and pressure $P$ for a free expansion nozzle.}
	\label{fig:M_rho_vs_x}
	% \Python Scripts\HPC\HPCvsLPC.py
\end{figure}


The on-axis Mach number, density and pressure for a monoatomic gas are shown in \cref{fig:M_rho_vs_x}. We can see the on-axis gas density drops off precipitously with increasing distance from the nozzle aperture $x$. Recalling \cref{fig:Constant1999_fig1}, we want to bring the nozzle as close to the optical axis to maximize the interaction density. However, if nozzle face enters the focal volume it will be drilled by the high intensity light and the resulting metallic plume will coat the generation chamber's vacuum window. Under normal operating conditions, we estimate the optical axis is located at $x=100 \ \mu$m.
%Using the numbers from our previous example, this gives us an argon interaction pressure of about 45 Torr and a number density of $2.67 \times 10^{24} \textrm{ m}^{-3}$.
% number density is calculated in \Python Scripts\HHG_Phasematching-master\test\Constant_fig1.py

%The normalized off-axis gas density is shown in \cref{fig:off_axis_density}. At $x/d=0.5$, the FWHM of the density is $L_{med} = 120 \ \mu$m, which is smaller than the width of the laser spot size ($w_0 \sim 30 \ \mu$m), and the Rayleigh range ($z_R \sim 300 \ \mu$m).

\begin{figure}
	\centering
	\includegraphics[width=1.0\textwidth]{figures/chap3/on-axis-pressure.pdf}
	\caption{On-axis interaction pressure and absorption length for the free expansion nozzle. Left panel: On-axis pressure as a function of nozzle backing pressure for various on-axis distances $x$. Vertical lines indicate pressures at which the vacuum system is overwhelmed (5 mTorr). Right panel: Required average interaction pressure as a function of absorption length $L_{abs}$ for both gases at 30 and 100 eV.}
	\label{fig:on-axis-pressure}
	% \Python Scripts\HPC\HPCvsLPC.py
\end{figure}

The left panel of \cref{fig:on-axis-pressure} shows the on-axis interaction pressure for He and Ar ($\gamma = 5/3$) as a function of the nozzle backing pressure at various on-axis distances. Vertical dashed lines correspond to the highest sustainable backing pressure for each gas. For $x/d = 1$, the maximum interaction pressure for argon is $\sim 120$ Torr and for helium it is $\sim 35$ Torr; for $x/d = 2$, $P_{int}(Ar) \sim 360$ Torr and $P_{int}(He) \sim 110$ Torr.

Recalling the results of the 1D reabsorption model discussed in \cref{sec:XUV_reabsorption}, ideal phase matching occurs if $L_{\textrm{med}} > 3 L_{\textrm{abs}}$. We can estimate the absorption length of the gas plume by assuming a boxcar function density profile of width $d$. Note that this approximation overestimates the average density in the interaction region, as the off-axis density profile at large values of $x/d$ is proportional to $\cos^4 \theta$, where $\tan \theta \equiv y/x$ \cite{millerFreeJetSources1988}. Therefore $d/3$ represents a hard upper bound for $L_{\textrm{abs}}$, which is 67 $\mu$m for a 200 $\mu$m nozzle. The right panel of \cref{fig:on-axis-pressure} shows the absorption length for both gasses at 30 and 100 eV. We can see that the free expansion nozzle is only capable of phase matching argon at lower photon energies, where the XUV absorption cross section is larger. The free expansion nozzle is incapable of properly phase matching helium at any photon energy.

%We can now calculate the absorption length $L_{abs} = 1 / \rho \sigma$ for this jet. The photoabsorption cross section is relatively constant for argon in the range 45 - 150 eV \cite{gulliksonCXROXRayInteractions}. For argon at 100 eV, $\sigma = 2 r_0 \lambda f_2 = 1.3 \times 10^{-4} \textrm{ nm\textsuperscript{2}}$. Therefore the absorption length of the gas jet backed by 180 Torr of argon is $L_{abs} = 2.9 \textrm{ mm}$, and $L_{med}/L_{abs} = 0.04$. Referring back to \cref{fig:Constant1999_fig1,eqn:HHG_Nout_simple}, we can see that we are well within the quadratic regime of the HHG reabsorption model, regardless of the coherence length $L_{coh}$. Increasing the backing pressure by a factor of 5 will yield only $L_{med}/L_{abs} = 0.20$, still within the quadratic regime. This leaves significant room for HHG yield improvement.
% absorption length is calculated in \Python Scripts\HHG_Phasematching-master\test\Constant_fig1.py

%The structure of the resulting supersonic plume is shown in \cref{fig:gas_expansion}. The physics of supersonic gas flow has been extensively studied in the literature and will not be discussed at length here. Below is a brief overview of the relevant physics required to understand the gas nozzles used for HHG in our lab. For a more detailed review of the field, see Ref \cite{millerFreeJetSources1988}.

% derivation of \cref{eqn:gas_dens}
%energy equation. $V$ is velocity, $h$ is enthalpy per unit mass.
%\begin{equation}
%h + V^2/2 = h_0
%\end{equation}
%for ideal gases, $dh = \hat{C}_p \ dt$, and we have

%\begin{equation}
%V^2 = 2(h_0 -h) = 2 \int_{T}^{T_0} \hat{C}_p \ dT
%\label{eqn:Scoles_gas_jet_energy}
%\end{equation}

%For an ideal gas, $\hat{C}_p = \gamma / (\gamma-1) (R/W)$, where $\gamma = C_p/C_V$ is the ratio of the specific heats, $R$ is the gas constant, $W$ is the molecular weight. if the gas is cooled substantially in the expansion ($T \ll T_0$), then we have:

%\begin{equation}
%V_{\infty} = \sqrt{ \frac{2R}{W} \left( \frac{\gamma}{\gamma-1} \right) T_0 }
%\end{equation}

%For an ideal gas, the speed of sound is $a = \sqrt{\gamma R T/W}$ and the Mach number is $M = V/a$. Assuming $\hat{C}_p$ is constant, we can recast \cref{eqn:Scoles_gas_jet_energy} in terms of $\gamma$ and $M$.  Using these assumptions, one can obtain the following relationships for the temperature $T$, velocity $V$, pressure $P$, mass density $\rho$ and number density $n$ in the gas jet scaled to those parameters at the stagnation point $(T_0, P_0, \rho_0, n_0)$:

%\begin{subequations}
%	\label{eqn:mach_properties}
%	\begin{align}
%	% eqn 2.3 - 2.6 in scoles, page 18
%	(T/T_0) &= \left(  1 + \frac{\gamma-1}{2} M^2 \right)^{-1} \label{eqn:gas_temp} \\
%	V &= M \sqrt{ \frac{\gamma R T_0}{W} } \left( 1 + \frac{\gamma-1}{2} M^2 \right)^{-1/2} \label{eqn:gas_velo} \\
%	(P/P_0) &= (T/T_0)^{\gamma/(\gamma-1)} = \left(  1 + \frac{\gamma-1}{2} M^2 \right)^{-\gamma/(\gamma-1)} \label{eqn:gas_pres} \\
%	(\rho/\rho_0) &= (n/n_0) = (T/T_0)^{1/(\gamma-1)} = \left(  1 + \frac{\gamma-1}{2} M^2 \right)^{-1/(\gamma-1)} \label{eqn:gas_dens}
%	\end{align}
%\end{subequations}

%Therefore, once we know the Mach number $M$, we can calculate the above properties for the gas jet. The Mach number is found by solving the fluid mechanics equations dealing with the conversation of mass, momentum and energy:

%\begin{subequations}
%	\label{eqn:scoles_continuum}
%	\begin{flalign}
%	% eqn 2.7 of scoles, page 19
%	\text{mass:} && \nabla \cdot (\rho \mathbf{V}) &= 0 && \label{eqn:scoles_mass} \\
%	\text{momentum:} && \rho \mathbf{V} \cdot \nabla \mathbf{V} &= - \nabla P  && \label{eqn:scoles_momentum} \\
%	\text{energy:} && \mathbf{V} \cdot \nabla h_0 &= 0 \textrm{ or } h_0 = \textrm{constant along streamlines} \label{eqn:scoles_energy} && \\
%	\text{equation of state:} && P &= \rho \frac{R}{W} T  && \label{eqn:scoles_eqn-state} \\
%	\text{thermal equation of state:} && dh &= \hat{C}_P \ dT \label{eqn:scoles_thermal-eqn-state} && 
%	\end{flalign}
%\end{subequations}

%The above equations are valid for an isentropic, compressible flow of a single component ideal gas molecular weight $W$ and constant specific heat ratio $\gamma$. A steady state is assumed and viscosity and heat conduction are neglected. These equations have been numerically solved in the literature for two source geometries: a ``slit" nozzle (2D, planar) and a circular aperture (3D, axisymmetric). The numerical solutions to each geometry scale with the nozzle diameter $d$, and have been fit to the following analytical functions:

%\begin{subequations}
%	\label{eqn:Scoles_centerline2.2}
%	\begin{align}
%	% eqns from table 2.2 of scoles, page 23
%	\frac{x}{d} > 0.5&: &&M = \left( \frac{x}{d} \right)^{(\gamma-1)/j} \left[ C_1 + \frac{C_2}{\left(\frac{x}{d}\right)} + \frac{C_3}{\left(\frac{x}{d}\right)^2} + \frac{C_4}{\left(\frac{x}{d}\right)^3} \right] \label{eqn:Scoles_centerline1} \\
%	0 < \frac{x}{d} < 1.0&: &&M = 1.0 + A \left( \frac{x}{d} \right)^2 + B \left( \frac{x}{d} \right)^3 \label{eqn:Scoles_centerline2}
%	\end{align}
%\end{subequations}

%\textbf{question: why does M increase without bound with increasing x, while V is limited to a finite value? scoles has a discussion, you should address it here.}

%The fitting coefficients for \cref{eqn:Scoles_centerline2.2} are listed in \cref{tbl:Scoles_gas_params2.2}. A plot of the results for different source geometries and gases are shown in \cref{fig:scoles_mach}.





%\cref{tbl:Scoles_mach_params} shows the centerline Mach numbers used in the following equations:

%\begin{subequations}
%	\label{eqn:Scoles_centerline2.1}
%	% eqns from table 2.1 of scoles, page 22
%	\begin{align}
%	M &= A \left( \frac{x-x_0}{d}\right)^{\gamma-1} - \frac{\frac{1}{2} \left( \frac{\gamma+1}{\gamma-1} \right)}{A \left(\frac{x-x_0}{d} \right)^{\gamma-1}} \label{eqn:gas_mach} \\
%	\frac{\rho(y,x)}{\rho(0,x)} &= \cos^2(\theta) \cos^2\left(\frac{\pi\theta}{2\phi}\right) \\
%	\frac{\rho(R,\theta)}{\rho(R,0)} &= \cos^2\left(\frac{\pi\theta}{2\phi}\right) \\
%	\left(\frac{x}{d} \right) &> \left( \frac{x}{d} \right)_{\text{min}} \label{eqn:mach_cond}
%	\end{align}
%\end{subequations}
%The gas nozzle throughput $\hat{T}$ is calculated from:
%\begin{equation}
%\hat{T} \ (\text{torr} \cdot \text{l}/\text{s}) = \hat{S} \cdot P_b = C \left(\frac{T_C}{T_0} \right)\sqrt{\frac{300}{T_0}}(P_0 d) d
%\label{eqn:nozzle_thruput}
%\end{equation}

%where $C$ is the gas constant from \cref{tbl:Scoles_gas_params}, $P_0$ is the nozzle's backing pressure in Torr, $T_C$ and $T_0$ are the vacuum chamber and backing temperatures, respectively, in Kelvin, $P_0$ is the backing pressure in Torr, and $d$ is the nozzle's diameter in cm.

%
%\begin{table}[]
%	\centering
%	\begin{tabular}{llllll}
%		Gas    & $\epsilon / k$ (K) & $\sigma$ (angstrom) & $C_6 / k$ ($10^{-43}$ K $\cdot$ cm$^6$) & $Z_r$     & \begin{tabular}[c]{@{}l@{}}C (l/cm$^2$/s);\\ \cref{eqn:nozzle_thruput}\end{tabular} \\ \hline
%		He     & 10.9               & 2.66                & 0.154                                   & -         & 45                                                               \\
%		Ne     & 43.8               & 2.75                & 0.758                                   & -         & 20                                                               \\
%		Ar     & 144.4              & 3.33                & 7.88                                    & -         & 14                                                               \\
%		Kr     & 190                & 3.59                & 16.3                                    & -         & 9.8                                                              \\
%		Xe     & 163                & 4.3                 & 41.2                                    & -         & 7.9                                                              \\
%		H$_2$  & 39.6               & 2.76                & 0.7                                     & $\sim$300 & 60-63                                                            \\
%		D$_2$  & 35.2               & 2.95                & 0.93                                    & $\sim$200 & 42                                                               \\
%		N$_2$  & 47.6               & 3.85                & 6.2                                     & $\sim$2.5 & 16                                                               \\
%		CO     & 32.8               & 3.92                & 4.76                                    & $\sim$4.5 & 16                                                               \\
%		CO$_2$ & 190                & 4.0                 & 31.1                                    & $\sim$2.5 & 12-13                                                            \\
%		CH$_4$ & 148                & 3.81                & 18.1                                    & $\sim$15  & 21                                                               \\
%		O$_2$  & 115                & 3.49                & 8.31                                    & $\sim$2   & 15                                                               \\
%		F$_2$  & 121                & 3.6                 & 10.5                                    & $\sim$3.5 & 14                                                               \\
%		I$_2$  & 550                & 4.98                & 336                                     & $\sim$1   & 5.2                                                              \\ \hline
%	\end{tabular}
%	\caption{Gas parameters used in free expansion calculations. Table recreated from Ref \cite{millerFreeJetSources1988}.}
%	\label{tbl:Scoles_gas_params2.1}
%\end{table}
%
%\begin{table}[]
%	\centering
%	\begin{tabular}{lllll}
%		\hline
%		$\gamma$ & $x_0/d$ & $A$  & $\phi$ & $(x/d)_{\text{min}}$ \\ \hline
%		1.67     & 0.075   & 3.26 & 1.365  & 2.5                  \\
%		1.40     & 0.4     & 3.65 & 1.662  & 6                    \\
%		1.2857   & 0.85    & 3.96 & 1.888  & 4                    \\
%		1.20     & 1.00    & 4.29 & -      & -                    \\
%		1.10     & 1.60    & 5.25 & -      & -                    \\
%		1.05     & 1.80    & 6.44 & -      & -                    \\ \hline
%	\end{tabular}
%	\caption{Centerline Mach Number and Off-Axis Density Correlations for Axisymmetric Flow. Table recreated from Ref \cite{millerFreeJetSources1988}.}
%	\label{tbl:Scoles_mach_params}
%\end{table}

\subsection{Low Pressure Cell}
\label{sec:LPC}

\begin{figure}
	\centering
	\includegraphics[width=0.5\textwidth]{figures/chap3/LPC_diagram.png}
	\caption{Rendering of the low pressure cell (LPC). The interaction region is contained within the rectangular block. The circular disk is attached to the universal gas receiver (not shown) via the four through holes.}
	\label{fig:LPC_diagram}
	% figure created in Chap 3 figures powerpoint.
\end{figure}

We have shown that the free expansion nozzle cannot provide sufficiently high pressure-length product to phase match high energy photons. The low pressure cell (LPC), shown in \cref{fig:LPC_diagram}, was designed to improve the ratio $L_{\textrm{abs}} / L_{\textrm{med}}$ while maintaining a relatively simple nozzle geometry.\cite{wangMidinfraredStrongfieldLaser2018}.\footnote{Special thanks to Zhou Wang for designing the original LPC. The LPC used in this work has been slightly modified to work with our universal gas receiver.} The LPC consists of an aluminum disk with rectangular block at the center of the top surface. Gas flows from the universal gas receiver (which is mated to the bottom of the disk) through a thin capillary and into the rectangular block. A through hole drilled into the front face of the rectangular block intersects the capillary and serves as the gas-laser interaction volume. Below, we will model the gas density profile of the LPC and show that the thickness of the block ($W = 2.032$ mm) sets the gas-laser interaction length.

\subsubsection{Gas Flow in the LPC}

\begin{figure}
	\centering
	\includegraphics[width=0.75\textwidth]{figures/chap3/LPC_schematic.pdf}
	\caption{Conceptual gas flow schematic of the LPC. Arrows indicate direction of gas flow. Red shaded region indicates laser path.}
	\label{fig:LPC_schematic}
	% figure created in Chap 3 figures powerpoint.
\end{figure}

%\begin{figure}
%	\centering
%	\includegraphics[width=0.75\textwidth]{figures/chap3/LPC_interaction_p.pdf}
%	\caption{Calculated pressure in the LPC interaction region.}
%	\label{fig:LPC_interaction_p}
%	% figure created in \Python Scripts\HPC\HPCvsLPC.py
%\end{figure}

%\begin{figure}
%	\centering
%	\includegraphics[width=0.75\textwidth]{figures/chap3/LPC_chamber_p.pdf}
%	\caption{Calculated pressure in the target chamber using the LPC.}
%	\label{fig:LPC_chamber_p}
%	% figure created in \Python Scripts\HPC\HPCvsLPC.py
%\end{figure}

\cref{fig:LPC_schematic} shows a gas flow model of the LPC inside the generation vacuum chamber. The green arrows indicate the direction of gas flow, and the red shaded region indicates the laser focus. The gas receiver is considered to be an infinite reservoir of gas with pressure $P_1$. This region supplies the laser interaction region with gas via a thin capillary of diameter $R = 101.5 \ \mu \textrm{m}$, length $L = 5 \textrm{ mm}$ and volumetric flow rate $Q_2$, modelled as an ideal isothermal gas \cite{fryerTheoryGasFlow1966,venerusLaminarCapillaryFlow2006,landauFluidMechanics2011}. The interaction region with pressure $P_2$ acts as a pressure source for two diametrically opposed supersonic gas jets \cite{millerFreeJetSources1988}, each with diameter $d$ and throughput $\hat{T}_{n}$. The generation chamber has a turbopump with pumping speed $S_{t}$ and an equilibrium pressure $P_3$. We justify this treatment by noting that we are in the supersonic regime, as $P_2 \sim 100$ Torr and $P_3 \sim 10^{-3}$ Torr. Additionally, the mean free path $l$:
\begin{equation}
l = \frac{\mu}{P} \sqrt{\frac{\pi R T}{2 M}}
\end{equation}
is much shorter than either physical dimension ($d, W$) of the rectangular block. For helium at 100 Torr, $l \sim 1.5 \ \mu \textrm{m}$ and for argon  $l \sim 0.5 \ \mu \textrm{m}$. Therefore, we can think of the interaction region as a reservoir of gas for two diametrically opposed supersonic jets. This approach results in the following coupled equations (SI units):
\begin{align}
P_1^2 - P_2^2 &= \frac{16 \mu L Q_2 P_2}{\pi R^4} \\
Q_2 P_2 &= 2 \hat{T}_n \\
P_3 &= \frac{2 \hat{T}_n}{S_t} \\
\hat{T}_n &= C P_2 d^2
\label{eqn:LPC_coupled_equations}
\end{align}
where $C$ is the gas constant expressed in \si{m/s} (see \cref{eqn:nozzle_thruput}) and $\mu$ is the dynamic viscosity in \si{Pa.s}. Solving for the interaction pressure $P_2$ and chamber pressure $P_3$, we obtain:
\begin{align}
P_2 &= - \frac{16 C d^2 L \mu}{\pi R^4} + \sqrt{P_1^2 + \frac{256 C^2 d^4 L^2 \mu^2}{\pi^2 R^8}} \\
P_3 &= \frac{2 C d^2}{S_t} \left( - \frac{16 C d^2 L \mu}{\pi R^4} + \sqrt{P_1^2 + \frac{256 C^2 d^4 L^2 \mu^2}{\pi^2 R^8}}  \right)
\label{eqn:LPC_pressures}
\end{align}

\begin{figure}
	\centering
	\includegraphics[width=0.75\textwidth]{figures/chap3/LPC_on_axis.pdf}
	\caption{Calculated on-axis density and pressure for the low pressure cell.}
	\label{fig:LPC_on_axis}
	% figure created in \Python Scripts\HPC\HPCvsLPC.py
\end{figure}

\cref{eqn:mach_properties} can be used to calculate the on-axis density and pressure of the LPC, which are shown in \cref{fig:LPC_on_axis}. Our simplified model yields a constant pressure in the LPC assembly and a sharp drop off in the plume region. Note that due to the $G$ factor (see \cref{eqn:G_factor}), the FWHM of the pressure and density is simply the width of the LPC's rectangular block, $W$. When considering XUV reabsorption, we will neglect the existance of the gas plumes and treat the gas density pressure as a boxcar function with a thickness ${L_{\textrm{med}}=W}$.

\begin{figure}
	\centering
	\includegraphics[width=0.9\textwidth]{figures/chap3/LPC_pressure_vs_diameter.pdf}
	\caption{Effect of the laser hole diameter $d$ on the LPC gas flow. Left panel: chamber pressure. Dashed horizontal line the indicates maximum sustainable operating pressure. Right panel: interaction pressure. Circles indicate the interaction pressure at the maximum sustainable backing pressure.}
	\label{fig:LPC_pressure_vs_diameter}
	% figure created in \Python Scripts\HPC\HPCvsLPC.py
\end{figure}

\cref{fig:LPC_pressure_vs_diameter} shows the effect of the laser hole diameter $d$ on the chamber and interaction pressures when using argon gas. As $d$ increases relative to the capillary dimensions ($R, L$), the pressure of the interaction region decreases, and the overall conductance of the LPC assembly increases. As a result, gas is no longer efficiently trapped in the interaction region, but instead escapes out to the greater vacuum chamber. We therefore want to minimize the diameter of the laser hole as much as possible within manufacturing and optical constraints. Note that when misaligned, the laser will drill into the rectangular block, increasing the effective value of $d$ over time. For this reason, the LPC is considered a consumable part that should be replaced when its maximum operating pressure is insufficient for experimental needs.

\begin{figure}
	\centering
	\includegraphics[width=0.9\textwidth]{figures/chap3/LPC_IntPress_AbsLen.pdf}
	\caption{Interaction pressures and XUV absorption lengths for the low pressure cell (LPC). Left panel: interaction pressure of helium (blue) and argon (red) as a function of backing pressure. Bright lines correspond to $d = 400 \ \mu$m, faint lines are for $d = 200 \ \mu$m. Right panel: absorption length for the interaction pressures shown in the left panel.}
	\label{fig:LPC_IntPress_AbsLen}
	% figure created in \Python Scripts\HPC\HPCvsLPC.py
\end{figure}

\cref{fig:LPC_IntPress_AbsLen} shows the corresponding absorption length for a given backing and interaction pressure. In the left panel, we have selected pressure curves for $d = 400 \ \mu\textrm{m}$ (bright lines) and $d = 200 \ \mu\textrm{m}$ (faint lines) for both helium (blue) and argon (red). As in \cref{fig:LPC_pressure_vs_diameter}, the filled circles indicate the maximum sustained operating pressure before the vacuum system is overwhelmed. In the right panel, we show the corresponding absorption length $L_{\textrm{abs}}$ for low and high energy photons in both gas media. Applying the arguments of \cref{sec:XUV_reabsorption}, we compare the absorption length to the upper bound required for efficient phase matching, $L_{\textrm{abs}} = L_{\textrm{med}}/3$ (black vertical dashed line). We can see that a $200 \ \mu\textrm{m}$ LPC is capable of efficiently phase matching 100 eV photons in argon, whereas a $400 \ \mu\textrm{m}$ LPC cannot.

%Although the maximum backing pressure is similar to the free expansion jet, the longer interaction length yields a substantially higher value of $L_{\textrm{med}} / L_{\textrm{abs}}$.

%Experimentally, we are limited to backing pressures of about 400 Torr or less in the LPC, which yields interaction pressures below 100 Torr for argon or nitrogen, and below 30 Torr in helium. Somewhat counterintuitively, the interaction pressure is lower for helium than the other species for a given backing pressure. This is because helium has a larger value of $C$ that prevents it from being trapped in the interaction region. 

%Labs calculations done in \Python Scripts\HHG_Phasematching-master\test\Constant_fig1.py
%When operating at maximum backing pressure, the maximum absorption length for the LPC at 100 eV in argon is identical to that of the free expansion jet: ${L_{abs} = 0.6 \ \textrm{mm}}$. However, the longer interaction length of the LPC yields a much higher ratio ${L_{med}/L_{abs} = 3.5}$, which is about 17 times higher than that of the free expansion nozzle.

%reference the 1D model plot: \cref{fig:Constant1999_fig1,eqn:HHG_Nout_simple,sec:free-expansion-nozzle}


\subsubsection{HHG in LPC}

To test the feasibility of the LPC for transient absorption experiments, we performed high harmonic generation under various experimental conditions.

\begin{figure}
	\centering
	\includegraphics[width=0.75\textwidth]{figures/chap3/LPC_P_scaling_He800.pdf}
	\caption{Total harmonic yield of the LPC as a function of interaction pressure.}
	\label{fig:LPC_performance}
	% figure created in Python Scripts\HPC\LPC_800nm.py
\end{figure}

\cref{fig:LPC_performance} shows the pressure scaling of the LPC when using an 800 nm pulse and helium gas. The interaction pressure is calculated from the backing pressure and the geometry of the nozzle using \cref{eqn:LPC_pressures} using an assumed laser hole diameter of $d = 400 \ \mu$m. The pulse energy was controlled by closing an iris before the generation chamber; average power was measured with a power meter. From this figure, we can see that the harmonic yield increases with increasing interaction pressure. Using the reabsorption model of \cref{sec:XUV_reabsorption}, this trend indicates that we are operating with a sub-optimal interaction pressure, and increasing the pressure-length would improve our harmonic yield.

The red curve in \cref{fig:HHG-HPCvsLPCHPC} shows an optimized harmonic spectrum from the LPC (17 Torr interaction pressure, 1.85 mJ pulse energy). Recalling \cref{sec:XUV-spectral-calibration}, a harmonic spectrum with more than an octave of bandwidth will have both $m=1$ and $m=2$ diffraction orders present on the screen. The highest resolvable harmonic is at 107 eV, which we will call the cut-off energy. As such, the spectrum below 52 eV is contaminated with ${m=2}$ light and the shape of the lower energy harmonics should be ignored. From \cref{eqn:cutoff_energy}, we estimate $U_p = 26.0 \ \textrm{eV}$ and from \cref{eqn:Up-numbers}, we estimate the intensity to be ${I_0 = 3.5 \times 10^{14} \ \textrm{W/cm\textsuperscript{2}}}$. On the other hand, if we calculate the peak intensity from the input pulse energy, we would expect ${I_0 = 6 \times 10^{15} \textrm{ W/cm\textsuperscript{2}}}$, more than an order of magnitude higher that the HHG spectrum suggests.

\begin{figure}
	\centering
	\includegraphics[width=0.75\textwidth]{figures/chap3/eta_vs_t_He800_6e15Wcm2.pdf}
	\caption{On-axis ionization fraction vs. time for helium at the focus of an 800 nm $\tau = 65$ fs $6 \times 10^{15}$ W/cm$^2$ pulse. Times on the rising edge of the pulse where the ionization fraction is below the critical ionization fraction are shaded. Phase matching above 50 eV is possible in the blue region, where $\eta < 0.55\%$; phase matching below 50 eV is possible in both the blue region as well as the red region, where $\eta < 0.7\%$.}
	\label{fig:eta_vs_t_He800_6e15Wcm2}
	% figure created in Python Scripts\HPC\LPC_800nm.py
\end{figure}

\cref{fig:eta_vs_t_He800_6e15Wcm2} shows the on-axis ionization fraction within a single pulse for these experimental conditions. The peak intensity is sufficiently high to completely ionize the gas medium well before the peak of the field. The on-axis ionization fraction quickly exceeds the critical ionization fraction (see \cref{fig:crit_ion_frac}) a full 60 fs before the peak of the field, as shown in the shaded regions below the blue curve. This narrow on-axis phase matching window indicates that the majority of the light is coming from the larger off-axis volume, which is subjected to a lower peak intensity. Nevertheless, the interaction density is too low to accomodate ideal phase matching.

Relative to the continuous free expanion gas jet, the low pressure cell has an increased interaction length but cannot reach optimal phase matching pressures. While the flux is higher than the free expansion jet, it is insufficient for an ATAS experiment.

\subsection{High Pressure Cell}
\label{sec:HPC}

\subsubsection{Design of HPC}

The high pressure cell (HPC) was designed to be a drop-in upgrade to the previously available HHG gas sources. As such, we did not consider a semi-infinite gas cell design which would require disruptive chamber modifications. We also did not want to implement a waveguide solution, as its performance would be strongly effected by the coupling (and therefore the laser pointing) into the assembly \cite{popmintchevExtendedPhaseMatching2008,popmintchevPhaseMatchingHigh2009}. Finally, we wanted to avoid the complications of a servicing a pulsed solenoid valve \cite{evenEvenLavieValveSource2015}, so the HPC was designed to be user-servicable with low-cost replacement parts. As such, it consists of standard Swagelok and KF fittings with minimal modifications and a custom bellows assembly. The only consumable part is the stainless steel pipe housing the interaction region, and it only needs to be replaced when the HPC is installed or the focusing condition is changed.

\begin{figure}
	\centering
	\includegraphics[width=0.75\textwidth]{figures/chap3/HPC_cutaway2.png}
	\caption{Cutaway view of the HPC interaction region. From bottom left to top right: welded gas feedthrough, concentric inner \& outer pipes, connection to edge-welded bellows. The high pressure region is shaded blue. The green lines indicate the gas flow direction; the red line indicates the laser propagation direction.}
	\label{fig:HPC_cutaway2}
\end{figure}

\begin{figure}
	\centering
	\includegraphics[width=1.0\textwidth]{figures/chap3/HPC_cutaway_bellows.pdf}
	\caption{Cutaway view of the HPC assembly showing the flexible bellows connection and connection to the chamber wall.}
	\label{fig:HPC_cutaway_bellows}
\end{figure}

The design of the HPC is shown in \cref{fig:HPC_cutaway2,fig:HPC_cutaway_bellows}. It features two concentric cylinders: a stainless steel inner pipe which serves as the interaction region and gas source, and an outer shroud connected to an external rough pump which provides differential pumping. The inner pipe is connected to a gas line with continous flow. Laser-drilled diametrically opposed pinholes on the inner pipe wall allow for light propagation while minimizing gas flow to the outer shroud. A small portion of the gas within the outer shroud flows into the generation chamber via the machined holes, but most of the gas flows towards the exhaust and into a dedicated rough pump.

The relative positions of the inner pipe and the outer shroud are fixed by the KF hardware connections upon assembly. However, this positioning is not repeatable within the tolerances imposed by the laser transmission requirements. As a result, a new section of stainless steel pipe must be laser drilled every time the HPC is disassembled or removed from the generation chamber. The HPC assembly's position relative to the laser is adjustable via the same vacuum XYZ manipulator used for the free jet and LPC assemblies. A set of flexible bellows, visible in \cref{fig:HPC_cutaway_bellows}, allows for this movement while maintaining a vacuum-tight connection between the outer shroud and the chamber wall. A Baratron pressure gauge monitors the pressure of the KF tubing just outside the chamber wall.\footnote{Note: while the bellows can withstand an external pressure differential of 1 atm, they will become damaged if they are overpressured by 120 Torr. See \cref{app:HPC_instructions} before operating this system.} The flexible bellows has enough slack to allow the HPC to move below the optical axis, allowing the beam to pass over the top of the outer shroud. This is useful when aligning downstream optics.

The laser passes through the HPC assembly perpendicular to its symmetry axis; therefore the gas-laser interaction length is approximately equal to the diameter of the inner pipe. The outer shroud has two diametrically opposed machined 600 $\mu$m holes for the laser to pass through the assembly. During installation, the user aligns the two apertures in the outer shroud to the laser and fixes its position. Next, the inner pipe is installed and the unattentuated laser drills through the inner pipe walls. As a result, the four apertures are automatically colinear and aligned to the laser propagation axis.

\begin{figure}
	\centering
	\includegraphics[width=0.5\textwidth]{figures/chap3/HPC_laserhole_500x370.png}
	\caption{Photograph of the HPC's inner pipe showing the laser-drilled hole (bottom center of pipe). See text for details.}
	\label{fig:HPC_laserhole}
	% pictures of the HPC inner holes are located in \OneDrive - The Ohio State University\DiMauro\lab pics\HPC. The original TIFF picture was cropped with GIMP, then downsized and exported as a png.
\end{figure}

\cref{fig:HPC_laserhole} shows a photograph of the laser-drilled holes, taken with a 0.5x telecentric lens (Edmund Optics part number 62-911). Laser drift \&  misalignment, as well as daily harmonic optimization procedures over the course of several months have opened up these holes from their original diameter of approximately $100 \ \mu \textrm{m}$ to a final diameter of $430 \ \mu \textrm{m}$.

\subsubsection{Gas Flow in HPC}

\begin{figure}
	\centering
	\includegraphics[width=0.75\textwidth]{figures/chap3/HPC_pressure_schematic.pdf}
	\caption{Schematic used to calculate the pressures inside the HPC and generation chamber.}
	\label{fig:HPC_pressure_schematic}
	% chap3 powerpoint
\end{figure}

The differential pumping of the HPC assembly allows the user to use much higher interaction pressures than other continuous gas sources in the DiMauro lab. In this section we will model the gas flow through the HPC to see why this is the case. \cref{fig:HPC_pressure_schematic} shows a simplified gas flow of the HPC assembly. With the HPC cell installed, there are three distinct pressure regions within the generation chamber: the high pressure inner pipe (region $H$, with pressure $P_H$), the medium pressure outer shroud (region $M$, with pressure $P_M$), and the rest of the generation chamber remains at low pressure (region $L$, with pressure $P_L$). Two pairs of supersonic jets form at the boundaries of the three pressure regions. For each boundary, the gas jet serves as a gas sink for the higher pressure region, and as a gas source for the lower pressure region. Due to the large conductance of the tubing between the gas cylinder and the $H$ region, we treat $P_H$ as spatially constant and equal to the pressure reading on the regulator / inline pressure gauge. In the $M$ region, the majority of the gas flows orthogonal to the laser axis, down the roughing line to the floor pump; a small portion flows to the $L$ region via the machined apertures as supersonic jets. The large pressure differential (2-3 orders of magnitude) between each region justifies the assumption of supersonic flow \cite{millerFreeJetSources1988}.

In \cref{fig:HPC_pressure_schematic}, the dark blue region represents the high pressure region ($H$), the light blue region represents the medium pressure region ($M$), and the low pressure region is represented by the white region ($L$). Red arrows and text indicate gas sources, green arrows and text indicate flow towards the vacuum pumps; blue arrows and text indicate physical dimensions. $S_{\textrm{turbo}}$, $S_{\textrm{eff}}$ and $C_{\textrm{annular}}$ are the turbo pumping speed, effective rough pumping speed and annular conductance, respectively; $\hat{T}_H$ ($\hat{T}_M$) is the gas throughput from the $H$ ($M$) region into the $M$ ($L$) region from each supersonic jet.

By balancing the throughputs, we arrive at the following coupled equations:
\begin{equation}
\begin{aligned}
\hat{T}_H &= c P_H a_H^2 \\
P_M &= \frac{2(\hat{T}_H-\hat{T}_M)}{S_{\textrm{eff}}} \\
\hat{T}_M &= c P_M a_M^2 \\
P_L &= \frac{2 \hat{T}_M}{S_{\textrm{turbo}}}
\end{aligned}
\label{eqn:HPC-coupled-equations}
\end{equation}
In the above equations, $P_M$ and $P_L$ refer to the average background pressures in the $M$ and $L$ volumes, and the gas constant $c$ is the same as in \cref{eqn:nozzle_thruput}. That is, we ignore the structure of the plume in these calculations, which is justified because the size of each gas plume is smaller than the distance between the aperture and the next vacuum region.\footnote{For $P_H = 760 \ \textrm{Torr}$, $P_M = 0.7 \ \textrm{Torr}$ and $a_H = 100 \ \mu \textrm{m}$, $x_M = 22.1 a_H = 2.21 \ \textrm{mm}$, which is smaller than the distance between the laser drilled aperture and the outer shroud's machined aperture (6.2825 mm). For $P_L = 3 \times 10^{-4} \ \textrm{Torr}$ and $a_M = 600 \ \mu \textrm{m}$, we have $x_M = 32 a_M = 19.4 \ \textrm{mm}$, which is much smaller than the distance between the HPC and the next vacuum chamber (25 cm).} Rearranging \cref{eqn:HPC-coupled-equations}, we see that $P_M$ and $P_L$ are proportional to $P_H$, with prefactors that depend on the local effective pumping speed and aperture geometry:
\begin{equation}
\begin{aligned}
P_M &=  \frac{2 c a_H^2}{S_{\textrm{eff}}-2 c a_M^2} P_H  = \frac{S_{\textrm{turbo}}}{2 c a_M^2} P_L \\
P_L &= \frac{4 c^2 a_M^2 a_H^2 }{S_{\textrm{turbo}} (S_{\textrm{eff}} - 2 c a_M^2)} P_H
\end{aligned}
\label{eqn:HPC-PM-PL}
\end{equation}
The maximum achievable interaction pressure $P_H$ is only limited by the internal bellows burst pressure ($P_M \sim 120 \ \textrm{Torr}$) and the load on the turbopumps ($P_L$). From \cref{eqn:HPC-PM-PL}, we can see that we can reduce $P_M$ and $P_L$ by maximizing the effective rough pump speed $S_{\textrm{eff}}$ and minimizing the aperture sizes ($a_H$, $a_M$). The aperture sizes are set by the laser beam size and divergence, while $S_{\textrm{eff}}$ is conductance-limited by the roughing line connecting the $M$ region to the floor pump. Note that we do not need to know $S_{\textrm{eff}}$ to calculate $P_M$, provided we have accurate measurements of $a_M$ and $P_L$. However, it can be instructive to analyze how the geometry of the HPC assembly affects the effective pumping speed.


\begin{figure}
	\centering
	\includegraphics[width=0.75\textwidth]{figures/chap3/HPC_rough_line_schematic.pdf}
	\caption{Schematic showing the geometry and pressure profile of the HPC's rough vacuum line.}
	\label{fig:HPC_rough_line_schematic}
	% chap3 powerpoint
\end{figure}

\begin{figure}
	\centering
	\includegraphics[width=0.75\textwidth]{figures/chap3/HPC_press_performance.pdf}
	\caption{Measured HPC pressure performance using the temporary spectroscopy station. $P_H$ is measured from the gas source's inline pressure regulator; $P_{\textrm{baratron}}$ is measured using a diaphragm gauge immediately outside the chamber wall; $P_L$ is measured using a cold cathode gauge (PTR90) and corrected using values from the manufacturer's datasheet. Linear fits to $P_{\textrm{baratron}}$ and $P_L$ are performed for  $P_H < 3000 \ \textrm{Torr}$. Regulators were changed at $P_H \sim 3000$ and $\sim 8000 \ \textrm{Torr}$, which account for the discontinuities.}
	\label{fig:HPC_press_performance}
	% chap3 powerpoint
\end{figure}

We need to make a few key assumptions before we can calculate the effective pumping speed, $S_{\textrm{eff}}$. First, we assume that the pressure in the rough line is on the order of a few Torr (this was later confirmed experimentally using the Baratron gauge). Assuming a characteristic internal length scale of $L = 1 \ \textrm{cm}$ and mean free path of ${\lambda = (5 \ \textrm{Torr}/760 \ \textrm{Torr}) \times 80 \ \textrm{nm}}$, we can calculate the \textit{Knudsen number}, $Kn = \lambda / L \sim 0.001$, which meets the criteria for continuum flow ($Kn < 0.01$). Next, assuming an effective pump speed of 5 L/s and a pipe diameter of $2 \ \textrm{cm}$, we can calculate the \textit{Reynolds number}: $Re = \rho u d / \mu \sim 1$, which meets the criteria for laminar flow ($Re < 2300$). By making reasonable assumptions, we have shown that vacuum tubing between the $M$ region and the floor pump is well within the laminar flow regime. Again, this analysis ignores the structure of the supersonic plume; the flow is likely turbulent near the boundary of the plume. The effective pumping speed in the medium pressure region $S_{\textrm{eff}}$ can therefore be calculated using standard conductance formulae for laminar flow (SI units) \cite{joustenHandbookVacuumTechnology2016}:
\begin{equation}
\begin{aligned}
\frac{1}{S_{\textrm{eff}}} &= \frac{1}{C_{\textrm{annular}}} + \frac{1}{C_{\textrm{bellows}}} + \frac{1}{C_{\textrm{pipe}}} + \frac{1}{S_{\textrm{RV}}} \\
C_{\textrm{annular}} &= \frac{\pi}{128} \frac{1}{\eta} \frac{1}{L_{\textrm{annular}}} \left( d_{\textrm{out}}^4 - d_{\textrm{in}}^4 - \frac{(d_{\textrm{out}}^2 - d_{\textrm{in}}^2)^2}{\ln \left[d_{\textrm{out}}/d_{\textrm{in}}\right]} \right) \frac{P_M + P_{\textrm{bellows}}}{2} \\
C_{\textrm{bellows}} &= \frac{\pi}{128} \frac{1}{\eta} \frac{d_{\textrm{bellows}}^4}{L_{\textrm{bellows}}} \frac{P_{\textrm{bellows}} + P_{\textrm{baratron}}}{2} \\
C_{\textrm{pipe}} &= \frac{\pi}{128} \frac{1}{\eta} \frac{d_{\textrm{pipe}}^4}{L_{\textrm{pipe}}} \frac{P_{\textrm{baratron}} + P_0}{2}
\end{aligned}
\label{eqn:HPC-Seff-equations}
\end{equation}
where the pressures are measured in Pa, $\eta$ is the dynamic viscosity of the gas in Pa$\cdot$s, distances are in meters, and $S_{\textrm{RV}}$ is the rated pump speed of the floor pump in m\textsuperscript{3}/s. The geometry of the rough line system is defined in \cref{fig:HPC_rough_line_schematic}. We include in the calculation of $S_{\textrm{eff}}$ the three main vacuum elements between the outer shroud of the HPC and the floor pump:
\begin{enumerate}
	\item the short annular region formed between the inner pipe's Swagelok fittings and the inner wall of the shroud (see \cref{fig:HPC_cutaway2}), defined by inner diameter $d_{\textrm{in}} = 1.283 \ \textrm{cm}$, outer diameter $d_{\textrm{out}} = 1.575 \ \textrm{cm}$, length $L_{\textrm{annular}} = 2 \ \textrm{cm}$, entrance pressure $P_M$ and exit pressure $P_{\textrm{bellows}}$;
	\item the edge-welded bellows assembly, defined by length $L_{\textrm{bellows}} = 27.6 \ \textrm{cm}$, interior diameter $d_{\textrm{bellows}} = 1.7272 \ \textrm{cm}$, entrance pressure $P_{\textrm{bellows}}$ and exit pressure $P_{\textrm{baratron}}$;
	\item the length of flexible PVC pipe connecting the exterior of the chamber to the floor pump, defined by length $L_{\textrm{pipe}} = 150 \ \textrm{cm}$ and interior diameter $d_{\textrm{pipe}} = 2 \ \textrm{cm}$, entrance pressure $P_{\textrm{baratron}}$ and exit pressure $P_0$.
\end{enumerate}

Given the above dimensions, the annular region has an outsized impact on the total conductance of the differntial pumping system:
\begin{equation*}
\begin{aligned}
C_{\textrm{annular}} &= (5.83 \times 10^{-10} \ \textrm{m}^{3}) \times (\bar{p}_{\textrm{annular}}/\eta) \\
C_{\textrm{bellows}} &= (7.91 \times 10^{-9} \ \textrm{m}^{3}) \times (\bar{p}_{\textrm{bellows}}/\eta) \\
C_{\textrm{pipe}} &= (2.62 \times 10^{-9} \ \textrm{m}^{3}) \times (\bar{p}_{\textrm{pipe}}/\eta)
\end{aligned}
\end{equation*}
where $\bar{p}_i$ is the average pressure in each element. For a floor pump speed with ${S_{\textrm{RV}} = 5 - 11 \ \textrm{L/s}}$, the effective pump speed $S_{\textrm{eff}}$ in the $M$ region will be between 20\% and 88\% of $S_{\textrm{RV}}$, assuming a constant average pressure $\bar{p}$ in the range ${1 - 30 \ \textrm{Torr}}$.

The HPC was initially tested in a simplified vacuum system, called the \textit{temporary spectroscopy station}, as the TABLe's interferometer was being used by another graduate student at the time. The temporary spectroscopy station consisted of the TABLe's target chamber (repurposed as a generation chamber) connected to the photon spectrometer via a differential pumping chamber. There was no XUV refocusing optic. Instead of using the TABLe's high throughput RV system (see \cref{fig:rough_vacuum_schematic}), individual floor pumps to back the turbos. The generation turbo was backed by a Leybold D40B ($S_{\textrm{RV}} = 13.3 \ \textrm{L/s}$); the HPC rough line was pumped by a scroll pump ($S_{\textrm{RV}} \sim 10 \ \textrm{L/s}$), and two smaller Leybold D16B pumps ($S_{\textrm{RV}} = 5.5 \ \textrm{L/s}$) were used to back the differential and photon spectrometer turbos.

\begin{figure}
	\centering
	\includegraphics[width=0.75\textwidth]{figures/chap3/HPC_on-axis-pressure.pdf}
	\caption{Calculated on-axis pressure and density profiles for helium with $P_H = 760$ Torr, $P_M = 0.7$ Torr and $P_L = 3 \times 10^{-4}$ Torr.}
	\label{fig:HPC_on-axis-pressure}
	% plot made in \Python Scripts\HPC\HPCvsLPC.py
\end{figure}

\cref{fig:HPC_press_performance} shows the HPC's pressure performance for helium and argon gas as measured in the temporary spectroscopy station. Pressures were measured with a Baratron diaphragm gauge ($P_{\textrm{baratron}}$, located immediately outside the chamber wall) and a Leybold PTR90 cold cathode gauge ($P_L$, located in the generation chamber). The pressure readings from the cold cathode gauge are corrected using the manufacturer's datasheet; the Baratron's readings are gas species-independent. Here, we see that $P_{\textrm{baratron}}$, which is approximately equal to $P_M$, scales linearly with respect to the backing pressure $P_H$ over a wide range of pressures. Note that the slope for helium is approximately 3 times that for argon, which is consistent with their gas constants ($c=14 \ \textrm{L/cm}^2\textrm{/s}$ for Ar, $c=45 \ \textrm{L/cm}^2\textrm{/s}$ for He).

Turning our attention the chamber pressure ($P_L$), we see a rollover in argon around $P_H \sim 8000 \ \textrm{Torr}$, and in helium we see a discontinuity in the slope at $P_H \sim 3000 \ \textrm{Torr}$. These features are unphysical and are likely caused by the cold cathode gauge malfunctions. Regardless, we can see that the chamber pressure stays in the milliTorr regime even when $P_H$ up to 5,000 Torr for helium and 14,000 Torr for argon.

\begin{figure}
	\centering
	\includegraphics[width=0.75\textwidth]{figures/chap3/HPC_absorption.pdf}
	\caption{Expected XUV reabsorption in the $P_M$ region for different generating media. $P_M$ calculated using $P_H = 760 \ \textrm{Torr}$ via \cref{eqn:HPC-PM-PL}. Absorption data from \cite{gulliksonCXROXRayInteractions}.}
	\label{fig:HPC_absorption}
	% plot made in \Python Scripts\CXRO\test\CXRO.py or \HPCvsLPC.py
\end{figure}

\cref{fig:HPC_on-axis-pressure} shows the calculated on-axis density and pressure profile of the HPC for helium and an interaction pressure of 760 Torr. The intermediate pressure $P_M$ is assumed to be equal to $P_{\textrm{baratron}}$ and is calculated using the linear coefficient obtained in \cref{fig:HPC_press_performance}. In calculating $P_M$, we omit the offset of $4.53 \ \textrm{Torr}$ which is a measurement artifact of the Baratron gauge. The on-axis pressure shows that there is a non-neglible amount of gas in the $M$ region, which is several millimeters long. If we assume that HHG occurs solely in the $H$ region, then XUV flux may inadvertently be reabsorbed when propagating through the $M$ region.

\begin{figure}
	\centering
	\includegraphics[width=0.75\textwidth]{figures/chap3/HPC-gas-flow-int-length.pdf}
	\caption{HPC calculated performance as a function of interaction pressure, assuming $a_H = 100 \ \mu \textrm{m}$ and the fit from in  \cref{fig:HPC_press_performance}. For both panels, blue (red) lines correspond to argon (helium), dashed (dotted) lines correspond to 30 (100) eV photons. Vertical dashed lines indicate the highest experimentally achieved pressures.}
	\label{fig:HPC-gas-flow-int-length}
	% plot made in \Python Scripts\CXRO\test\CXRO.py or \HPCvsLPC.py
\end{figure}

\cref{fig:HPC_absorption} estimates the XUV transmission (via $T = \int_M \dd{z} \exp (- \rho(z) \mu_a z )$) in the $M$ region assuming the aforementioned pressure profile, and \cref{fig:HPC-gas-flow-int-length} shows the XUV transmission in the $M$ region for 30 and 100 eV photons over a range of pressures. In both figures, XUV absorption in the $L$ region is neglected due to the low pressures. \cref{fig:HPC-gas-flow-int-length} also compares the absorption length ($L_{\textrm{abs}} = 1 / \rho \mu_a$) to the medium length $L_{\textrm{med}}$. We can see that the HPC can reach sufficiently high interaction pressures and length to meet the phase matching guidelines laid out in \cref{sec:XUV_reabsorption}.

%Of the above pressures, only $P_M$ is relevant to our discussion as it can be used to estimate XUV reabsorption. A diaphragm (Baratron) pressure gauge is installed between the pipe and the bellows outside the chamber to measure $P_{\textrm{baratron}}$ during experiments, and $P_L$ is measured using a PTR90 gauge. Using conservation of mass flow, we know that $q = S_{eff} P_M = \ \textrm{constant}$), and we can write the following:
%\begin{equation}
%C_{\textrm{eff}} = \frac{P_M S_{\textrm{eff}}}{P_M - P_{\textrm{baratron}}}
%\label{eqn:HPC-Ceff}
%\end{equation}
%where $C_{\textrm{eff}}$ is the combined conductance of the annular and bellows sections. Using this quantity, we can solve for the $P_M$ pressure in terms of the $P_{\textrm{bellows}}$ and $P_{\textrm{annular}}$:

%Combined with the backing pressure $P_H$, we can rearrange \cref{eqn:HPC-Seff-equations,eqn:HPC-Ceff} to solve for the intermediate pressure $P_M$ in terms of the:
%\begin{equation}
%\begin{aligned}
%P_M &= \frac{P_L S_{\textrm{turbo}}}{2 c a_M^2} \\
%T_M &= \frac{P_L S_{\textrm{turbo}}}{2} \\
%S_{\textrm{eff}} &= \frac{2 c a_M^2 (2 c a_H^2 P_H - P_L S_{\textrm{turbo}})}{P_L S_{\textrm{turbo}}}
%\end{aligned}
%\end{equation}


%We assume that the high pressure region is an infinite gas reservoir held at pressure $P_H$ set by the regulator on the gas cylinder.\footnote{In early experiments, we would use the regulator gauge to determine $P_H$. In later experiments, we used an inline pressure gauge (? brand and model ?) to indepedently monitor gas line pressure.} The medium pressure region has a net gas throughput of $Q_M = 2(T_H - T_M)$ and an effective pumping speed $S_{\textrm{eff}}$. The low pressure region has a gas throughput of $2T_M$ and a pumping speed of $S_{\textrm{turbo}}$. The pressure in each region is simply $P = Q / S$, where $Q$ is the net throughput and $S$ is the effective pumping speed of the region. Owing to the large pressure differentials between adjacent regions, the gas throughput of the apertures is supersonic and is proportional to the area of the aperture and the backing pressure, following \cref{eqn:nozzle_thruput}:
%\begin{equation}
%T_H = c P_H a_H^2,
%\end{equation}
%where $a_H$ is the diameter of the laser drilled aperture. The supersonic expansion gives rise to a plume structure with an on-axis spatial extent given by $x_M$, defined in \cref{eqn:Mach-disk}. At on-axis distances larger than $x_M$, we can ignore the supersonic plume structure and consider only the background pressure $P_M$. For $P_H = 760$ Torr, $P_M = 3$ Torr and $a_H = 100 \ \mu$m, $x_M = 10.6 a_H = 1.06$ mm. Since the distance between the laser drilled aperture and the outer shroud's machined hole (6.2825 mm) is much greater than $x_M$, we can ignore the supersonic plume structure when considering the gas flow from the medium pressure region to the low pressure region. As a result, $T_M$ has the same form as $T_H$:
%\begin{equation}
%T_M = c P_M a_M^2
%\end{equation}
%We can calculate the Mach disk location for the machined apertures using measured pressures in each region: for $P_M = 5$ Torr, $P_L = 3 \times 10^{-4}$ Torr, and $a_M = 600 \ \mu$m, we have $x_M = 67 a_M = 40.2$ mm. This distance is much smaller than the distance between the HPC and the next vacuum chamber (25 cm), so we can ignore the effect of the HPC's supersonic plumes on the rest of the beamline.

%The medium pressure region is pumped by a small RV pump with pumping speed $S_{\textrm{RV}}$. This pumping speed is reduced by the geometry of the pipes between the shroud's apertures and the mouth of the pump. First, the inner pipe and outer shroud form an annual pipe; secondly, the bellows and the several feet of soft PVC tubing further reduce the pumping speed. Given the relatively high pressures in the medium pressure region (a few Torr), the mean free path of the gas is much smaller than the characteristic length scale of the system and we are in the viscous regime. Therefore, we can compute the effective pumping speed $S_{\textrm{eff}}$ of the medium pressure region by successive application of the standard conductance equations \cite{hablanianHighvacuumTechnologyPractical1997,hoffmanHandbookVacuumScience1998}. First, the conductance of the annual region (liter/s) is:
%\begin{equation}
%C_{\textrm{annular}} = \frac{1}{1000} \frac{\pi}{8 \eta} \frac{P_1 + P_2}{2 L} \left( r_{out}^4 - r_{in}^4 - \frac{(r_{out}^2 - r_{in}^2)^2}{\log \left[r_{out}/r_{in}\right]} \right)
%\end{equation}
%where $r_{out}$ and $r_{in}$ are the outer and inner radii of the annular region (in cm), $\eta$ is the viscocity (Torr*seconds), $L$ is the length of the annular region (cm), and the pressures on either side of the annular region are $P_1$ and $P_2$ (in Torr). The conductance of the bellows assembly, the chamber feedthrough and the several feet of PVC tubing is treated using standard formalism
%\begin{equation}
%C_{KF} = 179 \frac{d^4}{L} \frac{P_1 + P2}{2}
%\end{equation}
%where $d$ is the diameter of the pipe, $L$ is the length, and $(P_1 + P_2)/2$ is the average pressure along the pipe. With a known conductance $C$ and pumping speed $S$, we can calculate the effective pump speed $S_{\textrm{eff}}$ in the usual way:
%\begin{equation}
%\frac{1}{S_{\textrm{eff}}} = \frac{1}{C} + \frac{1}{S}
%\end{equation}
%We now have a framework in which to calculate the pressure profile of the HPC. To pump the HPC, we use a small 5 L/s rough pump connected to the chamber via a 150 cm long KF25 soft PVC tube, which delivers a pumping speed of 4.97 L/s to the chamber wall. The bellows assembly (KF16 diameter, 30 cm long) reduces the pumping speed to 4.69 L/s at the beginning of the annular section. The annular section of the HPC assembly is very restrictive ($r_{\textrm{out}} = 0.7875 \textrm{ cm, } r_{\textrm{in}} = 0.6415 \textrm{ cm, } L = 8.661 \textrm{ cm}$), with a conductance of $C_{\textrm{annular}} = 1.64 \textrm{ L/s}$ and an effective pumping speed for the medium pressure region of $S_{\textrm{eff}} = 1.22 \textrm{ L/s}$. If we assume a laser drilled aperture diameter of $a_H = 193 \ \mu$m and a backing pressure of $P_H = 760$ Torr, and a turbo pump speed of $S_{\textrm{turbo}} = 1000$ Liter/s, then we will have $P_M = 3.12$ Torr and $P_L = 3.16 \times 10^{-4}$ Torr.

%interaction region of HPC: inner diameter of metal pipe = 0.069 inch = 1.75 mm. outer diameter of metal pipe = 0.125 inch = 3.175 mm.

%\cref{fig:HPC_on-axis-pressure} shows the on-axis pressure for the HPC. We can see that the HPC concentrates the gas within the inner pipe (760 Torr) where most of the XUV light will be produced. Due to the lower pressures and the finite Rayleigh range, harmonics will not be produced in the medium pressure region ($P_M = 3 \textrm{ Torr}$). However, the extended interaction length (6.9 mm) will lead to significant transmission losses at lower photon energies (20 - 50 eV), depending on the generating media. This effect is shown in \cref{fig:HPC_absorption}. Even with these absorption losses, the improved pressure-length product of the HPC makes it a bright XUV source in this energy range, as will be shown below.

%The pressure of the bellows inside the bellows is monitored with a Baratron diaphragm pressure sensor, which records the pressure with an accuracy that is independent of gas species. It is important to monitor the internal pressure, as the positive pressure differential cannot exceed 120 Torr without damaging the bellows. The measured bellows pressure is shown as a function of backing pressure in the central panel of \cref{fig:HPC-gas-flow-int-length} for argon and helium. We observe a linear relationship:
%\begin{equation}
%P_{\textrm{bellows}} = \alpha P_H + \beta
%\end{equation}
%Note that the discontinuities at $P_H \sim 3000 \ \textrm{Torr}$ and $P_H \sim 8000 \ \textrm{Torr}$ are due to a change in gas regulators during the experiment. For argon, we fit $\alpha = 3.74 \times 10^{-4} \ \textrm{Torr}^{-1}$ and $\beta = 4.71 \ \textrm{Torr}$; for helium, we fit $\alpha = 9.21 \times 10^{-4} \ \textrm{Torr}^{-1}$ and $\beta = 4.52 \ \textrm{Torr}$.

%The top panel of \cref{fig:HPC-gas-flow-int-length} shows the calculated transmission through the $P_M$ region. The transmission is obtained by integrating the on-axis density $\rho$ in the $P_{\textrm{bellows}}$ region (from the laser-drilled aperture to the machined aperture) downstream of the $P_H$ region and the atomic scattering factors \cite{gulliksonCXROXRayInteractions}.


\subsubsection{HHG in the HPC}

An effort was made to increase the XUV flux in the energy range 90 - 130 eV (near the Si $L$-edge). To this end, HHG experiments were performed in the temporary spectroscopy station using the HPC and an $f = 40 \ \textrm{cm}$ CaF\textsubscript{2} lens. Input power and phase matching were simultaneously controlled using an adjustable aperture located before the focusing optic. The interaction pressure was controlled with the gas cylinder's regulator and measured using an inline digital pressure gauge (Ashcroft model 2274). The fundamental and $<50$ eV harmonics were blocked using a $200 \ \mu\textrm{m}$ Zr filter located approximately $75 \ \textrm{cm}$ downstream of the HPC. The spectrometer's energy axis was crudely calibrated by counting the harmonics above the Al $L$-edge and assuming $2\omega_1$ spacing, and fitting to a \nth{5} degree polynomial (see \cref{sec:XUV-spectral-calibration}), and all HHG yields shown are scaled by the Jacobian. To isolate the effects of pressure scaling, phase matching conditions were optimized at low pressure and held constant as the interaction pressure was controlled. To maximize harmonic flux, the HPC's position relative to the focus was optimized (at the focus for helium and downstream of the focus for argon). Phase matching conditions were optimized for photon energies above 100 eV. Pulse energy was measured using a power meter immediately before the generation chamber; the reported values do not take into account any transmission losses of the fundamental due to the two HPC vacuum apertures between the generation chamber's window and the interaction region ($H$). Note that calculating the interaction intensity is complicated by diffraction resulting from these apertures.

\begin{figure}
	\centering
	\includegraphics[width=0.75\textwidth]{figures/chap3/HPC_P_scaling_He800.pdf}
	\caption{Total harmonic yield as function of interaction pressure in the HPC.}
	\label{fig:HPC_P_scaling_He800}
	% plot made in \Python Scripts\HPC\HPC_800nm.py
\end{figure}

The total harmonic yield for helium using 800 nm light as a function of interaction pressure is shown in \cref{fig:HPC_P_scaling_He800}. Unlike the LPC, we can see a maximum in the harmonic yield with respect to interaction pressure, indicating that we are no longer limited by the vacuum performance of the gas source. Helium's high ionization potential ($I_p = 24.5874 \ \textrm{eV}$) necessitates high laser intensities to efficiently drive the HHG process, and we see that a $<300 \ \mu \textrm{J}$ increase in pulse energy from 1.84 to 2.13 mJ increases total harmonic yield by nearly an order of magnitude. Each spectrum shown here corresponds to two 25-second exposure images averaged together and background subtracted. We report the normalized yield, which is the background subtracted average of the aforementioned dataset, which is then normalized by both exposure time and spatial height (number of pixels) on the camera sensor.

\begin{figure}
	\centering
	\includegraphics[width=0.9\textwidth]{figures/chap3/HPC_800nm_He_spectrogram.pdf}
	\caption{Harmonics generated from helium as a function of interaction pressure for $\lambda_1 = 800 \ \textrm{nm}$, a pulse energy of 2.13 mJ and a $200 \ \mu \textrm{m}$ Zr filter. Note that different energy harmonics are phase matched at different pressures.}
	\label{fig:HPC_800nm_He_spectrogram}
	% plot made in \Python Scripts\HPC\HPC_800nm.py
\end{figure}

\cref{fig:HPC_800nm_He_spectrogram} shows a spectrogram of the 800 nm 2.13 mJ helium dataset. We can see a broad maximum in yield for energies below $\sim 110 \ \textrm{eV}$ below $650 \ \textrm{Torr}$. At the optimum pressure, harmonic yield above 110 eV is maximized above $650 \ \textrm{Torr}$ at the expense of lower energy light; light below 80 eV is suppressed when $P_H > 1600 \ \textrm{Torr}$. This observed dispersion matches the general $1/\Delta n$ pressure scaling in \cref{fig:recip_deltan_plot}.

\begin{figure}
	\centering
	\includegraphics[width=0.75\textwidth]{figures/chap3/HPC_vs_LPC_800He.pdf}
	\caption{Comparison of the LPC and the HPC in helium at 800 nm. Generation conditions for each were optimized (LPC: 17 Torr interaction pressure, 1.85 mJ; HPC: 650 Torr, 2.13 mJ) in helium at 800 nm. The higher interaction pressure extends the phase matched region from $ \sim 110 \ \textrm{eV}$ to $\sim 130 \ \textrm{eV}$, and increased pressure-length product increases the XUV brightness by at least two orders of magnitude across the spectrum. Note that harmonics with energies less than half the cutoff are contaminated by \nth{2} order diffraction from the XUV spectrometer's grating. Faint blue line includes the calculated transmission factor in the $M$ region.}
	\label{fig:HHG-HPCvsLPCHPC}
	% plot made in \Python Scripts\HPC\HPCvLPC_comparison.py
\end{figure}

\begin{figure}
	\centering
	\includegraphics[width=0.75\textwidth]{figures/chap3/HPC_He800nm_enhancement.pdf}
	\caption{Relative performance of the HPC compared to the LPC in the energy range 75 - 150 eV. Faint blue line includes the calculated transmission factor in the $M$ region. Datasets are the same as in \cref{fig:HHG-HPCvsLPCHPC}.}
	\label{fig:HPC_He800nm_enhancement}
	% plot made in \Python Scripts\HPC\HPCvLPC_comparison.py
\end{figure}

\cref{fig:HHG-HPCvsLPCHPC} compare the performance of the HPC to the HPC. We can see that the total harmonic yield of the HPC exceeds that of the LPC by more than two orders of magnitude. Additionally, the highest discernable harmonic energy is increased from $\sim 110$ eV to $\sim 130$ eV. Note that the lower octave of each spectrum is contaminated with to $m=2$ XUV light. \cref{fig:HPC_He800nm_enhancement} shows the relative enhancement of HHG using the HPC compared to the LPC. Over the energy range 75 - 150 eV, we can see 100x improvement in yield, with a 250x peak centered at 110 eV. The position of this enhancement peak is very sensitive to generation conditions and can be shifted by tens of eV by adjusting the iris.

We can compare the measured performance to what was expected from our previous discussions on HHG. From \cref{eqn:HHG_Nout_2}, we expect the harmonic yield to scale as $(PL_{\textrm{med}})^2$. Using the pressure models developed above and a factor of $\sim 5$ to account for the increased pulse energy, we would expect the HPC to outperform the LPC by a factor of roughly 700. After taking into account the XUV absorption in the $M$ region, we measured a factor of 280, which is reasonably close to the expected value. 

ionization-induced defocusing of fundamental, and its effect on HHG yield vs pressure: \cite{altucciInfluenceAtomicDensity1996}

\textbf{open question (helium): does increasing the pulse energy by 300 uJ extend the cutoff in the LPC or the HPC?}

\textbf{make comparisons to literature values of optimum pressures (helium, 800 nm; argon, 1450 nm)}

\textbf{another question: why normalize to spatial height on sensor?}


\begin{figure}
	\centering
	\includegraphics[width=0.9\textwidth]{figures/chap3/HPC_1450nm_Ar_spectrogram.pdf}
	\caption{Performance of the HPC using argon at 1450 nm.}
	\label{fig:HPC_1450nm_Ar_spectrogram}
	% plot made in \Python Scripts\HPC\HPC_Ar_signal.py
\end{figure}

Similar experiments were performed using argon using signal wavelengths. \cref{fig:HPC_1450nm_Ar_spectrogram} shows a spectrogram of argon's harmonic yield at 1450 nm under typical operating conditions. Note that the lower ionization energy ($I_P = 15.75962 \ \textrm{eV}$) translates into a lower peak intensity for optimal harmonic generating conditions. We observe a rolling maximum in the harmonic yield, from 70 to 90 eV, as the interaction pressure is increased from 200 to 600 Torr. At higher energies ($> 90 \ \textrm{eV}$), we see a broad maximum between 250 and 850 Torr. As the pressure increases past 850 Torr, harmonic yield uniformly decreases.

\begin{figure}
	\centering
	\includegraphics[width=0.9\textwidth]{figures/chap3/HPC_1450nm_Ar_lineouts.pdf}
	\caption{Spectral lineouts of \cref{fig:HPC_1450nm_Ar_spectrogram} for pressures below the maximum in harmonic yield.}
	\label{fig:HPC_1450nm_Ar_lineouts}
	% plot made in \Python Scripts\HPC\HPC_Ar_signal.py
\end{figure}

\cref{fig:HPC_1450nm_Ar_lineouts} shows spectral lineouts of \cref{fig:HPC_1450nm_Ar_spectrogram} for lower interaction pressures. In this figure, we see that the harmonic yield above 90 eV increases about an order of magnitude as the interaction pressure is increased from 165 to 527 Torr, which is perfectly in line with the expected $P^2$ scaling ($527^2/165^2 = 10.2$). At 450 Torr and above, we observe a blueshifting of the harmonic comb, which is suggestive that the fundamental is experiencing nonlinear propagation effects near the focus. This effect was not observed in helium, which has a significantly higher ionization potential.

Although we do not have a directly comparable LPC HHG dataset, we can make some general comparisons between the two gas sources. Reaclling \cref{fig:LPC_IntPress_AbsLen}, the LPC is limited to a maximum interaction pressure of between 50 and 175 Torr (depending on laser aperture size). If HHG follows the observed $P^2$ scaling, then the HPC would be between 9 and 110 times brighter than the LPC if both systems are optimized for maximum HHG yield.

note that the increased XUV absorption of argon results in a lower overall enhancement factor than for helium \cite{popmintchevPhaseMatchingHigh2009}.

lower IP means lower pulse energy for optimal HHG.

question: is the harmonic yield between helium and argon comparable? i.e., do they have the same units? because it looks like argon is 1000x brighter.


harmonics exist out to about 150 eV.

lower IP: pressure-induced blueshift starts at lower pressures

do we have direct comparisons of HPC vs LPC for argon?




\textbf{general HPC comments:}

- limited pump speed $\rightarrow$ differential pumping is required


harmonic yield results

advantages: much brighter due to pressure-length product. future application: can operate in low-pressure mode and reduce downstream generation gas contamination of target chamber.

disadvantages: difficult to align and initially install (once it's installed, alignment is easy). messed up mode. HHG instability at higher pressures.

pictures of the HPC.



\subsection{Pulsed Amsterdam Piezovalve}
% piezovalve yield vs pressure: C:\testdata\2019_10_10

\begin{figure}
	\centering
	\includegraphics[width=0.75\textwidth]{figures/chap3/piezovalve_picture.png}
	\caption{Amsterdam Piezo Valve. The nozzle is located at the center of the front face (flat side).}
	\label{fig:piezovalve_picture}
	% picture supplied by andrew piper
\end{figure}

\begin{figure}
	\centering
	\includegraphics[width=0.75\textwidth]{figures/chap3/piezovalve_pscan.pdf}
	\caption{Integrated harmonic yield from the piezo valve as a function of interaction pressure. The inset shows excellent quadratic behavior with respect to interaction pressure. Assumed on-axis distance is $x = 250 \ \mu \textrm{m}$.}
	\label{fig:piezovalve_pscan}
	% plot made in \Python Scripts\HPC\piezovalve_1450nm.py
\end{figure}

\begin{figure}
	\centering
	\includegraphics[width=0.9\textwidth]{figures/chap3/piezovalve_spectra.pdf}
	\caption{piezo valve spectra vs pressure. Spectra are vertically offset for visual clarity. Assumed on-axis distance is $x = 250 \ \mu \textrm{m}$.}
	\label{fig:piezovalve_spectra}
	% plot made in \Python Scripts\HPC\piezovalve_1450nm.py
\end{figure}

We briefly had access to a commercial pulsed valve (Amsterdam Piezo Valve by MassSpecpecD BV)\footnote{Special thanks to Andrew Piper for letting us borrow his equipment.}. Operating a valve in pulsed mode greatly reduces the gas throughput into the vacuum chamber, as the throughput scales approximately with the duty cycle of the valve. Typical valve opening times are on the order of tens of microseconds \cite{irimiaSituCharacterizationCold2009,mengMeasurementDensityProfile2015,irimiaShortPulseMicrosecond2009}. When operated at 1 kHz to match our laser repetition rate, we achieve about two orders of magnitude reduction in chamber operating pressure, compared to a DC gas source of similar dimensions. The Amseterdam piezo valve utilizes an o-ring mounted to a cantilever piezoelectric flapper to briefly open an internal gas inlet port. Our model has a $d = 500 \ \mu \textrm{m}$ diameter straight-channel nozzle and a backing pressure range of 0 - 15 bar.

this data was collected using the full TABLe setup and therefore the XUV spectra are not directly comparable to the HPC datasets shown in \cref{sec:HPC}.

For these experiments, we used a $75 \ \mu \textrm{s}$ opening time and an operating voltage of 150 V. The on-axis distance from the aperture to the optical axis is estimated to be $x = 250 \ \mu \textrm{m}$.

We use \cref{eqn:Scoles_centerline2,eqn:mach_rho,eqn:mach_pressure} to calculate the on-axis jet parameters: the Mach number is $M=1.45$, the interaction density is $\rho/\rho_0 = 0.45$ and the interaction pressure is $P/P_0 = 0.27$.

delay introduced by Quantum Composer.

thanks to andrew piper for letting us use his piezovalve.

the piezo valve is shown in \cref{fig:piezovalve_picture}.

pressure scaling of total harmonic yield is shown in \cref{fig:piezovalve_pscan}.

evolution of spectra / shape of spectra changes with pressure, as shown in \cref{fig:piezovalve_spectra}. this is due to the photon energy dependence of phase matching conditions. this is typical for all gas sources.

for more detail and characterization of the piezo valve, see andrew piper's dissertation \cite{piperAndrewPiperDissertation2022}.

consider putting in the XUV spectra from Slava's paper.

\section{Characterization of XUV Source}

\subsection{Knife Edge Measurements}

\begin{figure}
	\centering
	\includegraphics[width=0.75\textwidth]{figures/chap3/knife_edge_cartoon.pdf}
	\caption{Schematic of XUV knife edge measurement. EM: ellipsoidal mirror, $z_0$: XUV focal plane.}
	\label{fig:knife_edge_cartoon}
\end{figure}

\begin{figure}
	\centering
	\includegraphics[width=0.75\textwidth]{figures/chap3/XUV_focus_knife_edge.pdf}
	\caption{A typical XUV knife edge measurement near the focal plane. The sample motor position is $k=11.0000$ mm. A fit to equation \cref{eqn:knife_edge} yields a beam waist of 10.82 $\mu$m at this position.}
	\label{fig:XUV_focus_knife_edge}
	% dataset: C:\testdata\2019_08_23\knife\11.0000
	% python file: \Python Scripts\Spectrometer\test\knife_edge.py
\end{figure}

\begin{figure}
	\centering
	\includegraphics[width=0.75\textwidth]{figures/chap3/XUV_waist_vs_k.pdf}
	\caption{Evolution of XUV beam waist as a function of propagation direction, $z$. The Rayleigh range $z_R$ and beam waist $w_0$ are extracted from the fit to \cref{eqn:beam_waist_evolution}.}
	\label{fig:XUV_waist_vs_k}
	% question: what is $M^2$ value of the XUV?. or, does w0 and zR change with XUV wavelength?
	% dataset: C:\testdata\2019_08_23\knife\11.0000
	% python file: \Python Scripts\Spectrometer\test\knife_edge.py
\end{figure}

We characterize the XUV focus in the target chamber by performing knife edge measurements at different $k$-positions, as depicted in \cref{fig:knife_edge_cartoon}. We use the interior angled edge of the Si frame on a broken sample heterostructure as a knife edge (see \cref{fig:Sample_Geometry}). This frame makes an excellent knife edge as it has a very well-defined geometry and fits in the sample holder. Recalling Gaussian optics, the assumed profile of the XUV beam is:
\begin{equation}
I(x,y,z) = I_0 \left( \frac{w_0}{w(z)} \right)^2 \exp \left( -2 \frac{ (x-x_0)^2 + (y-y_0)^2 }{w(z)^2} \right),
\end{equation}
using the coordinate system defined in \cref{fig:knife_edge_cartoon}. The XUV focus is at position $(x_0,y_0,z_0)$. The beam waist $w(z)$ will evolve as:
\begin{equation}
w(z) = w_0 \sqrt{ 1 + \left( \frac{z-z_0}{z_R} \right)^2 },
\label{eqn:beam_waist_evolution}
\end{equation}
where $z_R$ is the Rayleigh range. If we use the knife edge to block the transmission as depicted in \cref{fig:knife_edge_cartoon}, then the transmitted power will be:
\begin{equation}
P(x, z) = P_0 + \frac{P_{max}}{2} \left( 1 - \erf \left( \frac{\sqrt{2}(x-x_0)}{w(z)} \right) \right),
\label{eqn:knife_edge}
\end{equation}
where $x$ is the insertion of the knife in the beam, $z$ represents the location of the knife plane in the propagation direction, and $\erf$ is the error function.

A typical knife edge measurement is shown \cref{fig:XUV_focus_knife_edge}. In this measurement, the knife edge is translated across the XUV spot in 1 $\mu$m steps until the XUV light is completely blocked. A 2D spectrum is saved at each knife edge position. Each image is background subtracted, normalized and summed (integrating over all divergences and wavelengths), which yields the XUV flux as a function of knife position. The resulting curve is fit to \cref{eqn:knife_edge} and the beam waist $w(z)$ is extracted for this $z$-position.

The knife edge measurement is repeated at different $z$-positions until enough data has been acquired to determine the focal plane. The evolution of the XUV beam waist is shown in \cref{fig:XUV_waist_vs_k}. In this figure, the beam waist has been fit to \cref{eqn:beam_waist_evolution} to determine the focal plane $z_0$, the Rayleigh range $z_R$ and the beam waist $w_0$. In both figures, a reasonably good fit is obtained, indicating that the XUV light has a Gaussian spatial profile near the focus.


\subsection{harmonic yield stability}

\subsection{XUV spectra optimized for various HHG conditions}

\subsection{Measured Transmission of Metallic Filters}

\subsection{Ground State Measurements of Condensed Matter Samples}

\section{characterization of interferometric stability}

\section{MCP response}
% MCP voltage data taken on 2019_10_07.
scaling of yield and noise with respect to MCP voltage

\chapter{ATAS Experiments in Germanium}

\section{Introduction}

\section{Experimental Considerations}

\subsection{sample requirements}

\begin{figure}
	\centering
	\includegraphics[width=0.75\textwidth]{figures/chap4/Sample_transmission_CXRO.pdf}
	\caption{Calculated XUV transmission of various materials. Data from \cite{gulliksonCXROXRayInteractions}.}
	\label{fig:Sample_trans_CXRO}
	% figure generated using \PythonScripts\CXRO\test\CXRO.py
\end{figure}

There are several sample requirements for a successful condensed matter transient absorption experiment. First and foremost, the sample needs to have an absorption edge within the bandwidth of the XUV source. Second, the material must be the correct thickness for a transmission measurement, given the capabilities of the XUV light source and detector. If the material is too thick, the ground state will absorb most of the XUV flux and the recorded spectrum will be too close to the noise floor of the apparatus. If it is too thin, the laser-induced change of the ground state (on the order of $1-10\%$) will be lost in the noise. As a general guideline, a sample that absorbs 50\% at the spectral feature of interest provides a good compromise between these conflicting requirements. \cref{fig:Sample_trans_CXRO} shows the expected transmission of several materials, calculated from the atomic scattering factors \cite{gulliksonCXROXRayInteractions}. We can see that a typical sample will be on the order of 10 - 200 nm thick, depending on the material.

Another upper bound for sample thickness comes from material dispersion. In any material, the XUV light ($n_{\text{IR}} \sim 1$) will outpace the IR light ($n_{\text{IR}} > 1$). This effect can be significant even for ultrathin films. In order to keep the phase slippage between the XUV and IR light below half an IR period, the sample thickness $L$ must obey the following relationship:
\begin{equation}
L \le \frac{1}{2} \frac{\lambda_{\text{IR}}}{n_{\text{IR}} - n_{\text{XUV}}}
\end{equation}
For germanium excited with $\lambda_{\text{IR}}$ = 1430 nm and probed with 30 eV XUV at the $M_{4,5}$ edge, $n_{\text{IR}}$ = 4.2481 \cite{nunleyOpticalConstantsGermanium2016} and $n_{\text{XUV}}$ = 0.992536 \cite{gulliksonCXROXRayInteractions}, which gives a maximum thickness of 220 nm.

Next, the sample needs to be excitable using laser sources present in our lab (i.e., ultrafast pulses with wavelengths between 800 nm and a couple of microns). To minimize the slow build up of heat (on the order of seconds) and laser-induced damage, the sample needs to be rastered through the laser focus as the experiment is performed. This rastering method necessitates both a large clear aperture ($\sim$ 1 mm$^2$ - 1 cm$^2$) and good sample uniformity. Samples that meet the above thickness and clear aperture requirements are extremely delicate, with thicknesses between 5,000 and 100,000 times smaller than their freestanding lateral dimensions. As such, one should expect most samples to break before, during and after measurements, so a successful experiment will have a materials pipeline that is capable of producing multiple, consistent samples in a short time frame.

\subsection{The Supporting Nitride Membrane}

\begin{figure}
	\centering
	\includegraphics[width=0.75\textwidth]{figures/chap4/SiN_Al_transmission.pdf}
	\caption{XUV transmission measurements of Al metallic filter and silicon nitride membrane. Left panel: normalized XUV counts for i) unfiltered HHG signal, ii) HHG going through a 200 nm Al filter and iii) HHG going through a 200 nm Al filter and 30 nm of silicon nitride. Counts are scaled by the Jacobian. Right panel: transmission curves obtained from the left panel's data. Also shown are literature values for 20 nm of silicon nitride and 200 nm of Al with two 4 nm oxide layers \cite{gulliksonCXROXRayInteractions}. Multilayer interference is not taken into account. Oscillations in measured transmission are numerical artifacts which will be discussed in the text.}
	\label{fig:SiN_Al_transmission}
	% dataset: \2019_05_02\
	% plotted using: \Python Scripts\Spectrometer\test\nitride_trans.py
\end{figure}

\begin{figure}
	\centering
	\includegraphics[width=0.75\textwidth]{figures/chap4/nitride_map.pdf}
	\caption{XUV transmission map of 30 nm silicone nitride freestanding membrane. Left panel: integrated XUV counts in the range 30 -- 34 eV. Sample holder motor positions are indicated by x- and y-positions. Right panel: histrogram of logarithmic deviation of counts from the average. Dashed line shows a normal distribution.}
	\label{fig:nitride_map}
	% figure created using \Python Scripts\Spectrometer\test\rastermap.py
	% dataset: C:\testdata\2019_09_10\4_55_32 PM_nitride_map1
\end{figure}

\begin{figure}
	\centering
	\includegraphics[width=0.75\textwidth]{figures/chap4/Sample_Geometry.pdf}
	\caption{Cartoon showing the cross section of the free standing sample heterostructure. A 500 $\mu$m thick Si frame supports a freestanding 30 nm low stress silicon nitride membrane (Norcada QX7300X), upon which 100 nm of germanium has been deposited. The Si frame has a 3x3 mm$^2$ square clear aperture and a 7.5x7.5 mm$^2$ square external dimension. The taper of the Si frame thickness along the perimeter of the clear aperture forms a knife edge. In an ATAS experiment, the XUV and IR pulses propagate from the top to bottom of the figure.}
	\label{fig:Sample_Geometry}
\end{figure}

While most materials have an absorption edge within the range 25 - 150 eV, there are very few commercially available pre-fabricated materials with both the requistite large clear aperture and thickness. Note that either characteristic is relatively easy to achieve individually, but their combination presents unique materials challenges. We considered three synthesis methods to produce this quasi-2D sample:
\begin{enumerate}
	\item sample growth on a traditional substrate, followed by chemical back-etching or milling of the substrate until sub-micron thickness of the heterostructure is achieved;
	\item sample growth on a traditional substrate, followed by mechanical transfer onto a membrane;
	\item sample growth on a membrane.
\end{enumerate}
Sample quality and composition is heavily impacted by local growth conditions such as substrate temperature, deposition rate, substrate crystal cut, substrate-sample lattice mismatch, etc. Many of these characteristics are changed when growing on a substrate of a different cut, or by replacing a substrate with a membrane. In general, one should not expect success when applying a substrate-optimized growth recipe to a freestanding membrane. Therefore, methods 1 and 2 will yield the highest quality samples, as they leverage already-developed sample recipes. However, both methods require a technically difficult second step that is prone to failure.

Selective chemical etching recipes exist for certain compounds, but they usually require an additional chemically intert layer in the heterostructure to protect the sample. Adding this layer will come at the expense of the total XUV flux transmitted by the heterostructure. Additionally, the chemical etching rates are highly dependent on local chemistry, fluid convection and temperature \cite{chiuPhotoluminescenceEvolutionGaAs2015}, which ultimately means that the amount of material removed is uncontrollable and unrepeatable within our requirements (499.9 $\mu$m $\pm$ 10 nm removed from a 500 $\mu$m substrate). For these reasons, we decided to not pursue a chemical etch recipe. Ion or electron milling is more controllable, but too expensive to implement on a large scale. The above reasons preclude the use of Method 1.

Mechanical transfer of thin samples is a tried and true method, but it usually results in flakes with lateral dimensions on the order of 100 $\mu$m. Repeated transfer of many flakes is possible, but there little control over their exact positioning on the membrane. This results in a random distribution of flakes; the flakes are sometimes folded or overlapping one another. These mishaps increase the effective optical density of the sample, changing the IR and XUV absorption properties significantly.

An XUV spatial measurement needs to be taken prior to any ATAS experiment, but a non-uniform distribution of flakes on a membrane would require a much higher resolution map. This is because the flakes are on the order of the XUV and IR focii, so it is critical that the raster points in \cref{fig:Rastering_Methods} correspond to the center of each flake to avoid edge diffraction and to minimize the effects of slow laser pointing drift. For a uniform film, a map can be taken using 200-250 $\mu$m step sizes, as the most important feature is the border of the clear aperture. On the other hand, each flake would have to be sampled $\sim$5 times in each direction to find its center. As a conservative estimate, a membrane covered with $100 \times 100\text{ }\mu$m$^2$ flakes would require a step size of 20 $\mu$m, which increases the number of raster points by a factor of $10^2 = 100$. Considering that a $3 \times 3 \text{ mm}^2$ clear aperture sampled with 200 $\mu$m steps takes $\sim$45 minutes to map, a random distribution of flakes would take a prohibitively long time to map out.

With the first two methods ruled out, we turn to the third method of growing directly on a freestanding membrane. Although it will result in a lower quality sample, it does not have the same technical hurdles of the previous two methods. However, the large clear area makes the heterostructure extremely fragile. We initially attempted to circumvent this problem by using an array of smaller clear apertures.

As shown in \cref{fig:Rastering_Methods}, most of the sample's area isn't directly used by the laser - it exists as a buffer between the grid of sample points. An alternative to a single clear aperture is an array of micro-apertures, each with a diameter on the order of the IR spot size. The micro-apertures exist within a mechanically robust substrate and a thin membrane lies on top of the structure. This configuration significantly eases the material strength requirements by reducing the size of the unsupported area from cm-scale to sub-mm-scale. The regular grid of apertures avoids the difficulties of a randomly distributed sample, easing the XUV mapping step size requirements. Fortunately, these arrays are commercially available from Silson, Norcada (silicon nitride membranes) and US Applied Diamond (diamond membranes) but we encountered technical difficulties in their implementation. Because the aperture size is on the order of the size of the IR focal spot, there is very little room for positioning error, and our motors were insufficiently precise for this application. Further, these arrays are typically only available in at most a $3\times3$ array, which provides an insufficient number of raster points for an ATAS experiment.

With these limitations in mind, we decided to use large aperture x-ray windows from Norcada. These windows consist of a mechanically robust Si frame substrate with a square clear aperture cut through the center. The structure is fabricated so that a thin membrane covers the clear aperture. A schematic of the cross section is shown in \cref{fig:Sample_Geometry}.

Norcada offers these structures with either a silicon (polycrystalline or single-crystal) or a silicon nitride membrane. An ideal membrane is transparent to both XUV and IR wavelengths with a high damage threshold. Referring to \cref{fig:Sample_trans_CXRO}, 100 nm of Si provides a relatively flat transmission curve from 25 to 100 eV. In constrast, 30 nm of silicon nitride has poor, but featureless, transmission at lower energies. Both materials transmit light below their bandgaps (5 eV for SiN and 1.14 eV for Si). Finally, silicon nitride's higher bandgap results in a significantly higher laser damage threshold \cite{gamalyAblationSolidsFemtosecond2002,austinFemtosecondLaserDamage2018,keldyshIonizationFieldStrong1965}. Taking all these factors into account, we decided to use 30 nm silicon nitride membranes for germanium transient absorption experiments. The measured transmission of a typical membrane is shown in \cref{fig:SiN_Al_transmission}.


\subsection{rastering of sample through focus to avoid heating, charge build-up}

\begin{figure}
	\centering
	\includegraphics[width=0.75\textwidth]{figures/chap4/rastering_methods.pdf}
	\caption{Schematic of competing raster methods, shown in the sample's reference frame. The clear aperture of the sample is represented by the interior of the black square. The laser propagation direction is out of the page. The laser focal spots are shown as red circles, and the movement of the sample holder relative to the laser focus is indicated by arrows. A 200 $\mu$m border exists between the raster array and the perimeter of the sample's clear aperture. This diagram is to scale for a $1\times1$ mm$^2$ clear aperture sample, a 60 $\mu$m diameter IR focal spot and a 200 $\mu$m step size.}
	\label{fig:Rastering_Methods}
	%figure created using \Python Scripts\rastering\raster_diagram.py
\end{figure}

\subsection{XUV maps of samples}

\subsection{IR propagation in thin films (TMM starting with LightPipes output)}

\begin{figure}
	\centering
	\subfloat[]{
		\includegraphics[width=0.75\textwidth]{figures/chap4/GeSiN_RAT_1430nm.pdf}
		\label{fig:GeSiN_RAT_1430nm}}

	\subfloat[]{
		\includegraphics[width=0.75\textwidth]{figures/chap4/GeSiN_RAT_1450nm.pdf}
		\label{fig:GeSiN_RAT_1450nm}}
	\caption{Thin film calculations for the sample heterostructure. The red region represents germanium; green is silicon nitride. Top panel shows the absorbed energy density per unit input power. Bottom panel shows the local intensity $S(z) \equiv \vb{S}(z) \vdot \hat{z}$, normalized by the input intensity $S(0) \equiv \vb{S}(0) \vdot \hat{z}$. Calculations performed using the TMM package for Python \cite{byrnesTmmSimulateLight2017,byrnesMultilayerOpticalCalculations2019}.}
	\label{fig:GeSiN_RAT:delay}
	% python script: \Python Scripts\thinfilm\thinfilm.py
\end{figure}

\begin{figure}
	\centering
	\includegraphics[width=0.75\textwidth]{figures/chap4/tmm_vs_WL_1200-1600nm.pdf}
	\caption{TMM calculation showing the spectral response of the sample heterostructure.}
	\label{fig:tmm_vs_WL_1200-1600nm}
\end{figure}


\subsection{orbital-resolved excitation probability vs wavelength (band structure calculations)}

\begin{figure}
	\centering
	\includegraphics[width=0.75\textwidth]{figures/chap4/Ge_band_diagram_Zurch2017.pdf}
	\caption{Band structure and orbital character of germanium. Purple arrows indicate XUV-induced transitions from the $3d$ core levels to the valence bands. Red arrow indicates IR-induced transition across the direct band gap. Figure adapted from \cite{zurchDirectSimultaneousObservation2017}.}
	\label{fig:Ge_band_diagram}
\end{figure}

\begin{figure}
	\centering
	\includegraphics[width=0.75\textwidth]{figures/chap4/TOPAS_1400nm_spectral_inten.pdf}
	\caption{TOPAS spectral content at $\lambda = 1400 \ \textrm{nm}$.}
	\label{fig:TOPAS_1400nm_spectral_inten}
	% figure made in \Python Scripts\FROG\FROG2.py
\end{figure}

\cref{fig:TOPAS_1400nm_spectral_inten} shows the spectral content of the TOPAS signal at $\lambda = \ \textrm{nm}$. If we make the assumption that the spectral lineshape of the TOPAS output is invariant under wavelength shifts, then we can use this spectrum to estimate the spectral intensity at the bandgap when the TOPAS is set to 1430 or 1450 nm. The red line shows the spectral intensity, and the blue line is the integrated sum of the red line. The two dots are located at wavelengths 100 nm (blue) \& 120 nm (red) above the central wavelength of the pulse, which correspond to the location of the bandgap when the TOPAS is set to 1450 and 1430 nm, respectively. We can see that over 99\% of the pulse energy is contained in wavelengths below the bandgap.


\subsection{laser damage}

\subsection{estimation of excited carrier density}

\textbf{need to include orbital-resolved excitation probability results in this section}

We are concerned with two quantities: the peak excitation fraction in the sample and the average excitation fraction at the location of the XUV focus. The former quantity is relevant when considering sample damage, whereas the latter will be proportional to the measured signal. If the XUV and IR spots are perfectly overlapped at the sample plane, then these two quantities are approximately equal. We first calculate the peak excitation fraction, then we consider how a misaligned beam will affect the measured signal.

The laser propagation calculations in \cref{fig:pump_on_focus_calculation} were done for vacuum, but we are concerned with the field in our sample. The electric field inside a dielectric $E_{\text{int}}$ is related to the external electric field by the following equation \cite{schultzeAttosecondBandgapDynamics2014}:
\begin{equation}
E_{\text{int}} = \frac{2}{1+\sqrt{\epsilon}} E_{\text{ext}}
\label{eqn:internal_external_Efield}
\end{equation}
where $E_{\text{int}}$ is the electric field inside the sample, $E_{\text{ext}}$ is the electric field outside the sample, $\epsilon$ is the dielectric constant and its square root is the refractive index $n_{\text{IR}}$. The internal intensity $I_{\text{int}}$ is the square of the internal electric field. For germanium at $\lambda$ = 1430 nm, $n_{\text{IR}}$ = 4.2481, and we have the following relations:
\begin{equation}
\begin{aligned}
E_{\text{int}} &= 0.381 \times E_{\text{ext}} \\
I_{\text{int}} &= 0.145 \times I_{\text{ext}}
\end{aligned}
\end{equation}

Given our laser paremeters, we can estimate the highest carrier density within the sample. First, we estimate the absorbed laser fluence, $F_{\text{abs}}$ \cite{harbCarrierRelaxationLattice2006}:
\begin{equation}
F_{\text{abs}} = F_{\text{inc}} \left(1-R\right) \left( 1-\exp(-\alpha L) \right) \left(1+R \exp(-\alpha L)\right),
\label{eqn:absorbed_fluence}
\end{equation}
where $F_{\text{inc}}$ is the incident fluence, $R$ is the reflectivity equal to the square of the Fresnel coefficient, $\alpha$ is the absorption coefficient and $L$ is the sample thickness. The bracketed terms in \cref{eqn:absorbed_fluence} are the fraction of fluence transmitted by the first surface, the fraction absorbed by a single pass through the sample, and the additional absorption due to a back reflection off the rear face of the sample. Note that the back-propagating beam will arrive (on average) at a delay of $n_{\text{IR}} L/(2c) \approx 0.7 \text{ fs}$ later than the forward-propagating beam. This time scale is nearly two order of magnitude less than the IR pulse duration, so we should expect any electron dynamics initiated by the back reflection to contribute to the measured signal.

If each absorbed photon corresponds to an excited electron, then the excited carrier density $\Delta N$ is given by the following expression \cite{cushingDifferentiatingPhotoexcitedCarrier2019}:
\begin{equation}
\Delta N = \frac{F_{\text{abs}}}{\hbar \omega} \frac{1}{L},
\label{eqn:excitation_fraction}
\end{equation}
where $\hbar \omega$ is the IR photon energy. In \cref{eqn:excitation_fraction}, the quantity $F_{\text{abs}} / (\hbar \omega)$ represents the number of absorbed photons per unit area; dividing this quantity by the sample thickness gives the number of absorbed photons per unit volume. This assumes that the skin depth of the material is greater than membrane thickness, which is true for germanium at these wavelengths.

Finally, we convert the excited carrier density to a fractional excitation. Germanium has $N_{\text{u.c.}}=2$ valence electrons per unit cell, and each unit cell has a volume $V_{\text{u.c.}}=4.527 \times 10^{-23} \text{ cm}^{3}$. Therefore the fractional carrier excitation is
\begin{equation}
f = \Delta N \frac{V_{\text{u.c.}}}{N_{\text{u.c.}}}
\end{equation}

We can use literature values for 100 nm of germanium pumped at $\lambda$ = 1430 nm light. From the literature \cite{nunleyOpticalConstantsGermanium2016}, $R = 0.38315$, $\alpha = 5803.4 \text{ cm}^{-1}$, and so $F_{\text{abs}} = 0.0413 \times F_{\text{inc}}$. Therefore, only about 4.13\% of the incident fluence is absorbed by the sample.

According to the calculations in \cref{fig:pump_on_focus_calculation}, for each 1 $\mu$J energy input pulse (measured at L4), 0.413 $\mu$J makes it to the focal plane. 49\% of that energy is within the main lobe, which contains 0.202 $\mu$J of energy. Approximating the central lobe as a Gaussian beam with a FWHM of 35 $\mu$m and a pulse energy of 0.202 $\mu$J, the peak fluence is calculated by dividing the total energy of the Gaussian by $\pi w^2/2$. The Gaussian beam waist $w$ is related to the FWHM via $w^2 = \text{FWHM}^2 / (2 \ln 2)$. Thus, for each 1 $\mu$J input energy, the peak fluence in the central lobe is 14.6 mJ/cm$^2$ and the absorbed peak fluence is 0.60 mJ/cm$^2$. This corresponds to an peak excited carrier density of $4.3 \times 10^{20} \text{ cm}^{-3}$ and an excitation fraction of 0.98\% (per 1 $\mu$J of input energy).


\subsection{XUV-IR spatial overlap}

\begin{figure}
	\centering
	\includegraphics[width=0.5\textwidth]{figures/chap4/XUV-IR_overlap_integral.pdf}
	\caption{XUV-IR overlap function, as defined in \cref{eqn:XUV-IR_integral}, calculated using the numerical simulation results of \cref{fig:pump_on_focus_calculation} and a Gaussian XUV beam with a 6 $\mu$m waist. The result has been normalized to perfect overlap, $a_{max}$.}
	\label{fig:XUV-IR_integral}
	% plot made with \Python Scripts\LightPipes\pump_intensity.py using N=2**13 gridside
\end{figure}

The excitation fraction can be computed for each spatial coordinate on the sample using the above method and the predicted intensity distribution from the numerical beam propagation calculations. Because the electrons are being excited via a single-photon process, the excited carrier density will be proportional to the fluence, and thus proportional to the intensity shown in \cref{fig:pump_on_focus_calculation}. Because the intensity of the XUV is very weak, the absorption of the XUV by the sample is also linear. Thus, we should expect the ATAS signal to be proportional to the XUV-IR overlap integral:
\begin{equation}
a = \frac{ \int dV \text{ } I_{\text{IR}} I_{\text{XUV}} }{ \int dV \text{ } I_{\text{IR}} \int dV \text{ } I_{\text{XUV}} }
\label{eqn:XUV-IR_integral_volume}
\end{equation}
Here, the integration volume is over the entire sample. If we assume that the intensity distribution does not appreciably change over the thickness of the sample, we can simplify the above equation. This is a reasonable assumption because the sample ($L = 100 \text{ nm}$) is much thinner the Rayleigh range ($z_R \sim 1 \text{ mm}$), and the absorption is low ($\sim 4\%$). So we assume the sample is a $\delta$-function in thickness and only evaluate the intensities at the focal plane. With this assumption, the overlap integral becomes:
\begin{equation}
a = L \frac{ \int dA \text{ } I_{\text{IR}} I_{\text{XUV}} }{ \int dA \text{ } I_{\text{IR}} \int dA \text{ } I_{\text{XUV}} }
\label{eqn:XUV-IR_integral}
\end{equation}
Knife edge measurements have been performed on the XUV light, showing that it has a Gaussian spatial profile with a beam waist of 6 $\mu$m. We write down the spatial profile of the XUV light at the focus:
\begin{equation}
I^{XUV} = I_0^{XUV} \exp \left( - 2 ((x-x_0)^2 + (y-y_0)^2) /  w_{XUV}^2 \right)
\end{equation}
Here, $I_0^{XUV}$ is the peak intensity and $w_{XUV}$ is the beam waist (radius), defined as the point where the intensity falls to $e^{-2} = 13.5\%$ of its maximum. The lateral shift from the center of the IR focal spot in the horizontal and vertical directions is $x_0$ and $y_0$, respectively. With this formulation, and using the simulation results for the IR spot, the XUV-IR overlap integral is calculated as a function of XUV-IR misalignment $(x_0, y_0)$. This result is shown in \cref{fig:XUV-IR_integral}. Here, XUV beam is translated relative to the IR beam in the horizontal direction ($x_0$ with $y_0=0$) and the overlap is computed from \cref{eqn:XUV-IR_integral}.

\cref{fig:XUV-IR_integral} shows the sensitivity of a condensed matter ATAS experiment to relative alignment.\footnote{Note that the relevant parameter in \cref{eqn:XUV-IR_integral} is the relative positions of the two focal spots. We have yet to calculate the sensitivity of spatial overlap to deviations in the input laser pointing.} A spatial overlap deviation of 10 $\mu$m will cause the XUV-IR overlap - and thus the measured signal - to drop by 20\%. Note that a 10 $\mu$m displacement of the IR at the sample corresponds to a 15 $\mu$rad tilting of the hole mirror (HM). There are two ways misalignment can affect experimental results. If the relative positions of the XUV and IR focal spots changes as an experiment is performed, then the recorded ATAS signal would be a function of both the laser-induced dynamics and the XUV-IR spatial misalignment. On the other hand, if the entire experiment is performed using a constant misalignment, we would be exciting the sample to some peak excitation fraction $f$, but our probe would be measuring a lower excitation fraction ($\approx f a / a_{max}$). Consequently, the measured ATAS signal would be lower than otherwise expected, and any attempts to boost the signal by increasing the interaction intensity could result in permanent laser-induced sample damage.

A condensed matter ATAS experiment has much tighter alignment tolerances than a gas phase experiment. This discrepancy is a simple consequence of sample geometry and density. In either experiment, the measured signal comes from the region of space where the sample density, XUV intensity and IR intensity overlap. The transmission of XUV through the sample is, to first order, $T= \exp(- n \mu_a d)$, where $n$ is the number density, $d$ is the sample thickness and $\mu_a$ is the photoabsorption cross section. As discussed above, for technical reasons the experiment should be designed with $T \approx 1/2$. Therefore, the product $n \mu_a d$ will be approximately constant for any transient absorption experiment.

The number density of a condensed phase sample is determined by the chemistry of the compound and is on the order of $4 \times 10^{22} \text{ atoms}/\text{cm}^3$. The experimentalist is free to engineer clever sample geometries, heterostructures and/or nanopatterns, but the high atomic density (and thus absorption coefficient) dictates a total sample thickness on the order of 100 nm. On the other hand, the spatial profile and density of a gas phase sample is determined by the gas nozzle design and its backing pressure, respectively. A typical nozzle used in our lab produces a gas plume with lateral dimensions on the order of 200 - 500 $\mu$m. This effectively creates a sample that is three orders of magnitude thicker than a condensed phase sample, which relaxes the alignment constraints significantly. This has important consequences for the alignment of the sample.

If the XUV and IR are perfectly collinear, then the beam overlap region is effectively infinite in the propagation direction. In this case, the XUV-IR overlap integral will be positive regardless of any displacement of the sample plane from the focal plane, and maximal when the sample lies in the focal plane. However, if there is a small angle $\delta \theta$ between their $k$-vectors, then the beams will only spatially overlap within a finite region. In this case, the position of the sample plane relative to the beam crossing plane becomes a critical experimental parameter. For an infinitely thick sample (i.e., a chamber effusively filled with gas), it wouldn't matter where the beams crossed as long as they overlapped somewhere within the chamber. Then, the overlap integral would decrease as a function of $\delta \theta$, but it would never go to zero. For a thin sample, the bounds of \cref{eqn:XUV-IR_integral_volume} must enclose the beam overlap region, or else the integral will be zero. Thus, the signal strength of a condensed phase ATAS experiment is roughly 3 orders of magnitude more sensitive to the $z$-position of the sample relative to the focal plane than a gas phase ATAS experiment.


\section{Optimizing experimental ATAS parameters for Germanium thin films}

\subsection{rep rate (avoiding ms-scale excitation)}

\begin{figure}
	\centering
	\subfloat[$\tau \approx 0$ fs, PE = $1.03 \text{ } \mu \text{J}$.]{
		\includegraphics[width=0.4\textwidth]{figures/chap4/StaticOD_1kHz_overlap_1p03uJ.pdf}
		\label{fig:1kHz_Ge_ATAS:overlap_1.03uJ}}
	\qquad
	\subfloat[$\tau \approx 0$ fs, PE = $1.43 \text{ } \mu \text{J}$.]{
		\includegraphics[width=0.4\textwidth]{figures/chap4/StaticOD_1kHz_overlap_1p43uJ.pdf}
		\label{fig:1kHz_Ge_ATAS:overlap_1.43uJ}}
	
	\subfloat[$\tau = - \infty$, PE = $1.03 \text{ } \mu \text{J}$.]{
		\includegraphics[width=0.4\textwidth]{figures/chap4/StaticOD_1kHz_NegInf_1p03uJ.pdf}
		\label{fig:1kHz_Ge_ATAS:NegInf_1.03uJ}}
	\qquad
	\subfloat[$\tau = - \infty$, PE = $1.43 \text{ } \mu \text{J}$.]{
		\includegraphics[width=0.4\textwidth]{figures/chap4/StaticOD_1kHz_NegInf_1p43uJ.pdf}
		\label{fig:1kHz_Ge_ATAS:NegInf_1.43uJ}}
	
	\subfloat[PE = $1.03 \text{ } \mu \text{J}$.]{
		\includegraphics[width=0.4\textwidth]{figures/chap4/StaticOD_avg_1kHz_1p03uJ.pdf}
		\label{fig:1kHz_Ge_ATAS:avg_1.03uJ}}
	\qquad
	\subfloat[PE = $1.43 \text{ } \mu \text{J}$.]{
		\includegraphics[width=0.4\textwidth]{figures/chap4/StaticOD_avg_1kHz_1p43uJ.pdf}
		\label{fig:1kHz_Ge_ATAS:avg_1.43uJ}}
	\caption{1 kHz fixed-delay ATAS measurements on 100 nm Ge using a $\lambda = 1450 \text{ nm}$ excitation pulse. See text for details.}
	\label{fig:1kHz_Ge_ATAS}
	% datasets: \testdata\2019_08_06\{Avg1,Avg3,Avg4,Avg5}
	% python script: \Python Scripts\Spectrometer\test\2019_08_06.py
\end{figure}

\begin{figure}
	\centering
	\subfloat[$\tau \approx 0$ fs, PE = $1.75 \text{ } \mu \text{J}$.]{
		\includegraphics[width=0.4\textwidth]{figures/chap4/StaticOD_1kHz_overlap_1p75uJ.pdf}
		\label{fig:1kHz_Ge_ATAS:overlap_1.75uJ}}
	\qquad
	\subfloat[PE scaling at $\tau \approx 0$ fs.]{
		\includegraphics[width=0.4\textwidth]{figures/chap4/StaticOD_avg_1kHz_PE_scaling.pdf}
		\label{fig:1kHz_Ge_ATAS:PE_scaling}}
	
	\subfloat[PE = $1.43 \text{ } \mu \text{J}$.]{
		\includegraphics[width=0.4\textwidth]{figures/chap4/ODvsDelay_1kHz_1p43uJ.pdf}
		\label{fig:1kHz_Ge_ATAS:delay_1.43uJ}}
	\qquad
	\subfloat[PE = $1.75 \text{ } \mu \text{J}$.]{
		\includegraphics[width=0.4\textwidth]{figures/chap4/ODvsDelay_1kHz_1p75uJ.pdf}
		\label{fig:1kHz_Ge_ATAS:delay_1.75uJ}}
	
	\subfloat[PE = $1.43 \text{ } \mu \text{J}$ (rolling average).]{
		\includegraphics[width=0.4\textwidth]{figures/chap4/ODvsDelay_20roll_1kHz_1p43uJ.pdf}
		\label{fig:1kHz_Ge_ATAS:roll_delay_1.43uJ}}
	\qquad
	\subfloat[PE = $1.75 \text{ } \mu \text{J}$ (rolling average).]{
		\includegraphics[width=0.4\textwidth]{figures/chap4/ODvsDelay_20roll_1kHz_1p75uJ.pdf}
		\label{fig:1kHz_Ge_ATAS:roll_delay_1.75uJ}}
	\caption{1 kHz ATAS measurements in Ge using a $\lambda = 1450 \text{ nm}$ excitation pulse. \cref{fig:1kHz_Ge_ATAS:overlap_1.75uJ}: fixed-delay ATAS measurements with a pulse energy of 1.75 $\mu$J. \cref{fig:1kHz_Ge_ATAS:PE_scaling}: Pulse energy scaling at overlap of 1 kHz measurements. \cref{fig:1kHz_Ge_ATAS:delay_1.43uJ,fig:1kHz_Ge_ATAS:delay_1.75uJ,fig:1kHz_Ge_ATAS:roll_delay_1.43uJ,fig:1kHz_Ge_ATAS:roll_delay_1.75uJ}: delay scans at 1 kHz.  \cref{fig:1kHz_Ge_ATAS:delay_1.43uJ,fig:1kHz_Ge_ATAS:delay_1.75uJ}: raw delay scan data. \cref{fig:1kHz_Ge_ATAS:roll_delay_1.43uJ,fig:1kHz_Ge_ATAS:roll_delay_1.75uJ}: rolling average of the raw data with a 65 fs window (20 delay points). The left panel on each spectrogram shows the ground state spectrum $S_{gs}(E)$. See text for details.}
	\label{fig:1kHz_Ge_ATAS:delay}
	% datasets: \testdata\2019_08_06\{Delay1,Delay2}
	% python script: \Python Scripts\Spectrometer\test\2019_08_06.py
\end{figure}

\begin{figure}
	\centering
	\includegraphics[width=0.75\textwidth]{figures/chap4/StaticOD_avg_500Hz_1p67uJ.pdf}
	\caption{500 Hz ATAS measurements in Ge using a $\lambda = 1450$ nm, 1.67 $\mu$J excitation pulse. Each delay curve is an average of 104 identical measurements. The sample shows no delay dependance within the uncertainty of the measurement.}
	\label{fig:500Hz_Ge_ATAS:delays}
	% dataset: C:\testdata\2019_08_06\HWP1{Avg8,Avg9,Avg10}
	% python file: \Python Scripts\Spectrometer\test\2019_08_06.py
\end{figure}

\begin{figure}
	\centering
	\includegraphics[width=0.75\textwidth]{figures/chap4/StaticOD_avg_125Hz_2p64uJ.pdf}
	\caption{125 Hz ATAS measurements in Ge using a $\lambda = 1450$ nm, 2.64 $\mu$J excitation pulse. Each lineout represents the average of 394 measurements. See text for details.}
	\label{fig:125Hz_Ge_ATAS:static_delays}
	% dataset: C:\testdata\2019_08_13\Avg1,2,3
	% python file: Python Scripts\Spectrometer\test\2019_08_13.py
\end{figure}

\begin{figure}
	\centering
	\includegraphics[width=0.75\textwidth]{figures/chap4/Delay234_1450nm_125HzuJ.pdf}
	\caption{125 Hz delay scan in Ge using a $\lambda = 1450$ nm, 2.74 $\pm$ 0.35 $\mu$J excitation pulse. This is an average of 3 repeated measurements. See text for details.}
	\label{fig:125Hz_Ge_ATAS:delay_scan}
	% dataset: C:\testdata\2019_08_13\Delay2,3,4 averaged together.
	% python file: Python Scripts\Spectrometer\test\2019_08_13.py
\end{figure}

We performed exploratory experiments at 1 kHz to determine the optimal excitation pulse energy. For this set of measurements, we generated harmonics in Argon using $\lambda$ = 1450 nm, a 200 nm Al filter and the $2C$ optics removed (see \cref{fig:beamline_schematic}). Two delay points were recorded: with the XUV and IR pulses overlapped ($\tau = 0$) and with the XUV pulse arriving about 300 fs before the IR ($\tau = -\infty$). For each delay point, we used three pulse energies: 1.03, 1.43 and 1.75 $\mu$J. To increase the signal-to-noise, we performed each experiment 104 times and averaged the datasets. The exposure time was 0.5 seconds, and the sample was rastered so that each measurement was recorded at a new position on the sample. The results are shown \cref{fig:1kHz_Ge_ATAS,fig:1kHz_Ge_ATAS:overlap_1.75uJ,fig:1kHz_Ge_ATAS:PE_scaling}. 

\cref{fig:1kHz_Ge_ATAS:overlap_1.03uJ,fig:1kHz_Ge_ATAS:overlap_1.43uJ,fig:1kHz_Ge_ATAS:NegInf_1.03uJ,fig:1kHz_Ge_ATAS:NegInf_1.43uJ,fig:1kHz_Ge_ATAS:overlap_1.75uJ} show the average $\Delta A$ as a function of the number of averaged measurements, $M$. Spectral features are apparent after averaging about 10 datasets together, but the fidelity of the signal does not appreciably improve after the first $M=50$ datasets. \cref{fig:1kHz_Ge_ATAS:avg_1.03uJ,fig:1kHz_Ge_ATAS:avg_1.43uJ} show the average signal (lines) and the standard deviation (shaded area), as calculated from the entire ensemble of measurements. The data is noisy, but we can see a spectral feature near 30 eV which scales with pulse energy (or perhaps average power). This behavior is evident in \cref{fig:1kHz_Ge_ATAS:PE_scaling}. However, this feature is delay-independent: $\Delta A$ has nearly the same value regardless of whether the XUV arrives before or after the IR pulse.

To confirm the delay-independence of this feature, we performed delay scans at two different pulse energies (1.43 and 1.75 $\mu$J). These measurements are shown in \cref{fig:1kHz_Ge_ATAS:delay}. Delay was controlled by inserting a fused silica wedge into the inteferometer (W in \cref{fig:beamline_schematic}). Each delay step corresponds to approximately 3.25 fs (25 $\mu$m of wedge insertion).

The raw data is shown in \cref{fig:1kHz_Ge_ATAS:delay_1.43uJ,fig:1kHz_Ge_ATAS:delay_1.75uJ}. As this experiment was only performed once ($M=1$), the contrast is poor and the 30 eV feature is barely visible. The spectral feature becomes more prominent after performing a rolling average over 20 delay points (65 fs), which is shown in \cref{fig:1kHz_Ge_ATAS:roll_delay_1.43uJ,fig:1kHz_Ge_ATAS:roll_delay_1.75uJ}. These delay measurements confirm our suspiscion that the absorption feature is delay-independent at 1 kHz.

One possible origin of a delay-independent signal can be a very long-lived excited state with lifetime $1/\Gamma$. Each laser shot initiates an assortment of electron and phonon dynamics, each with their own time scales. If any of these excited states have time scales that approach the inverse rep. rate of the laser ($1/RR$), then the dynamics from the previous shot will still be evolving by time the next shot arrives. Since each exposure integrates over hundreds or thousands of laser shots, measurements at a nomimal delay $\tau$ will contain information from several delays $\tau_i$, each separated by the time between laser shots: $\tau_i = \tau, \tau-\frac{1}{RR}, \tau-\frac{2}{RR}, \dots$, with the amplitude of each contribution weighed by an exponential decay factor $\exp(+ \tau_i \Gamma)$. If this is the case, then the magnitude of the delay-independent signal should decrease as time between laser shots is increased. As the time between laser shots is increased past the lifetime of the state, the excited state population from the previous laser shot will be small enough to observe the dynamics of the shorter-lived states. Experimentally, we can accomplish this by adjusting the rep. rate divider on the Spitfire amplifier (which reduces the rep. rate), and/or using an optical chopper.

The rep. rate was halved to 500 Hz using the Spitfire's rep. rate divider ($m=2$) and the exposure time was doubled to 1 second so that the number of laser shots per exposure was held constant. A series of fixed-delay measurements were recorded using a 1.67 $\mu$J pulse energy, shown in \cref{fig:500Hz_Ge_ATAS:delays}. The spectral feature at 30 eV is still present, but it does not exhibit any delay dependence within the uncertainty of the measurement. It is notable that, for a similar pulse energy, the magnitude of the feature at 500 Hz is half that of the 1 kHz measurement. This is consistent with the hypothesis of a long-lived state contributing to the signal.

The rep. rate was lowered to 125 Hz using a combination of the rep. rate divider ($m=4$) and an optical chopper ($T = 50\%$) placed after the TOPAS. The exposure time was increased to 4 seconds to maintain sufficient counts on the detector. An ATAS spectrum was recorded about 130 fs after temporal overlap and several hundred fs before overlap using a 2.64 $\pm$ 0.34 $\mu$J excitation pulse ($\lambda$ = 1450 nm), as shown in \cref{fig:125Hz_Ge_ATAS:static_delays}. A second measurement after overlap was recorded to ensure repeatability. At negative delays (red curve), we see the familiar 30 eV feature, albeit weaker at $6 \times 10^{-3}$. Near overlap, we see a new feature emerge at 28.7 eV, with a magnitude of $\approx 10 \times 10^{-3}$. This observation is consistent with long-lived excited state at 30 eV. A delay scan at 125 Hz, shown in \cref{fig:125Hz_Ge_ATAS:delay_scan}, reveals that the 28.7 eV feature is indeed time-dependent with a ps-scale lifetime.


\subsection{IR pulse energy}

\subsection{harmonic spectrum ($\lambda$, 2-color generation)}

\begin{figure}
	\centering
	\subfloat[125 Hz ($M=6$)]{
		\includegraphics[width=0.4\textwidth]{figures/chap4/Delay123456_1430nm_125Hz_2p88uJ.pdf}
		\label{fig:125Hz_1430nm_Ge_ATAS:delay123456}}
	\qquad
	\subfloat[250 Hz ($M=2$)]{
		\includegraphics[width=0.4\textwidth]{figures/chap4/Delay89_1430nm_250Hz_2p92uJ.pdf}
		\label{fig:250Hz_1430nm_Ge_ATAS:delay89}}
	
	\caption{125 vs. 250 Hz measurements at $\lambda$ = 1430 nm. }
	\label{fig:125vs250Hz_1430nm_Ge_ATAS:delay}
	% datasets:
	% 125 Hz: \testdata\2019_08_15\125Hz\Delay1-6 averaged
	% 250 Hz: \testdata\2019_08_15\250Hz\Delay8,9 averaged
	% python script: \Python Scripts\Spectrometer\test\2019_08_15.py
\end{figure}

The fundamental wavelength was decreased by 20 nm to 1430 nm, where the absorption length in Ge is about 5\% shorter. For a fixed pulse energy, this increases the excitation fraction and thus the strength of the $\Delta A$ signal. Changing the wavelength also changes which initial and final states near the Fermi level are populated, according to the band structure calculations in \cref{fig:Ge_band_diagram}. Using this shorter wavelength we can observe a more robust sample response, as shown in \cref{fig:125vs250Hz_1430nm_Ge_ATAS:delay}.

At 1430 nm, we can see a sample response from 25.7 to 31 eV. From 25.7 to 30 eV, there is a broad increase in absorption, with the largest increase occuring between 28.4 and 29.5 eV. A decrease in absorption occurs between 30 and 31 eV. These features are present in both the 125 and 250 Hz data. The 30 eV negative delay feature persists in the 250 Hz dataset, but at $12 \times 10^{-3}$ it does not overwhelm the rest of the sample response. Further measurements were performed at either 125 Hz to suppress the static feature, or at 250 Hz to minimize data collection time.


\subsection{optimized ATAS Ge experimental results}

\subsection{post-experiment analysis: verify we didn't permanently damage sample}

\section{Data Analysis}

\subsection{description of data pipeline}
\subsubsection{going from 2D image to 1D spectra}

\begin{figure}
	\centering
	\includegraphics[width=0.75\textwidth]{figures/chap4/data_pipeline.png}
	\caption{this cartoon shows the data pipeline. it is an overview of all the processing steps i do on the data.}
	\label{fig:data_pipeline}
	% dataset: ???
	% python file: ???
\end{figure}

background subtraction, selecting a divergence window, normalization by exposure time \& divergence window, integration over divergence window
\subsubsection{energy calibration}

\begin{figure}
	\centering
	\includegraphics[width=0.75\textwidth]{figures/chap4/Ar_fano_cal.png}
	\caption{this figure shows the argon fano features vs pixel for the purpose of calibrating the spectrometer.}
	\label{fig:Ar_fano_calibration}
	% dataset: ???
	% python file: ???
\end{figure}



\subsubsection{$A$, $\Delta A$ calculation}

\subsection{systematic noise sources in our experiment}

\subsection{methods to numerically correct for harmonic noise and drift}

\subsection{frequency filtering to remove $\omega, 2 \omega$ oscillations}

\section{Physical Interpretation of spectra}
\subsection{decomposition of spectral response}
\subsection{description of observed dynamics}
\chapter{Conclusions}
\label{chap:conclusions}

In this thesis, we laid the groundwork for performing attosecond transient absorption spectroscopy (ATAS) measurements in the condensed phase using mid-infrared (MIR) lasers. We designed, built and tested an attosecond transient absorption beamline (TABLe) and a two-dimensional XUV spectrometer. The vacuum system was designed to accept different detector endstations, much like a user facility. The pump arm of the TABLe was designed to be outside the vacuum system so that nonlinear optics or other systems can be inserted into the beam path, which broadens the potential phase space of future studies. A high pressure cell (HPC) was designed and demonstrated to be nearly two orders of magnitude brighter than existing XUV sources in the DiMauro lab. This technical achievement partially counteracts the low quantum efficiency and difficult phase matching conditions of HHG at longer wavelengths. Finally, this equipment was demonstrated in a prototypical ATAS experiment in a technologically important indirect bandgap semiconductor (germanium) using a ${\lambda = 1430 \ \textrm{nm}}$ wavelength ultrafast excitation pulse. With a Keldysh parameter on the order of 1, the ionization channel is closer to the tunneling regime compared to previous reports in the literature. We observed electron and phonon dynamics in germanium that are consistent with what is reported in the literature. Our limited energy resolution, which is an artifact of our harmonic comb and detector nonlinearities, prevented us from resolving the energy dependence of the dynamics on the femtosecond timescale.

Future efforts should concentrate on making an XUV continuum and exciting the sample with longer wavelengths. An isolated attosecond pulse (IAP) would increase both spectral and temporal resolution. Increasing the pump wavelength would further reduce the Keldysh parameter $\gamma$, changing the nature of the initial ionization. For example, a below-bandgap $2 \ \mu \textrm{m}$ excitation would result in $\gamma\simeq 0.25$ while simultaneously suppressing single-photon ionization.
\begin{appendices}
\chapter{Guidelines for using the TABLe}
\label{appendix:TABLe_manual}

\section{TABLe / OMRON Pump-down Procedure}
\label{sec:OMRON_instructions}

Special thanks to Andrew Piper for coding and wiring the OMRON safety system. Below is an operational guideline to pump the system down to UHV using the OMRON system. Please see Andrew Piper's OMRON manual for additional details on the microcontroller system.

This procedure assumes that the chambers are initially at atmospheric pressure, the rough pumps are turned on, and the solenoid shutoff valves on the roughing line are closed.

\begin{enumerate}
	\item Seal and isolate all chambers. Close the manual valves between each turbo pump and the solenoid shutoff valves. Reattach any blow-off flanges (KF-25 blanks) that may have come off from the previous venting cycle.
	\item (Optional) As a test, ensure the OMRON's safety system is engaged by attempting to open one of the pneumatic gate valves via the control panel. \textbf{Caution: operating a gate valve between two chambers with a pressure differential can cause catastrophic system failure. Only perform this step after you have verified that all chambers are at atmospheric pressure.} If the gate valve opens while both chambers are above the upper setpoint, then the OMRON safety system is disabled. If the OMRON fails this test, check the status of the override switch.
	\item Retract the metal filters from the beam path to protect them from potential pressure surges. 
	\item Initiate the pump-down sequence by pressing the OK button on the OMRON. This will open all solenoid valves simultaneously with a loud \textit{thunk}.
	\item Slowly open the manual valves while monitoring the gas load on the rough pumps. Use the two remote pressure monitoring systems (Raspberry Pis) to monitor the inlet pressure for each blower system. As a rule of thumb, try to keep the inlet pressure below $\sim$100 Torr during this step. \textbf{Warning: quickly opening the manual values at atmosphere will result in pump oil being expelled into the rough pump's exhaust line. Continuing to run in this condition can lead to the overheating and eventual seizure of the rough pumps} \cite{kiesewetterDynamicsNearThresholdAttosecond2019}.
	\item Once the manual valves are fully open on all chambers, you can turn each blower system ON to accelerate the remaining pump-down procedure.
	\item Power on the turbo power supplies and switch the turbos to ON. After a few seconds, the magnetically levitated turbos will start levitating with a soft \textit{thunk}.
	\item Each turbo will automatically start spinning when its chamber reaches the upper set point ($\sim$200 mTorr). The turbos will take a few minutes to reach their final speed.
	\item Wait for the system to pump down. It typically takes 15-45 minutes for the entire system to reach $10^{-6}$ Torr.
	\item The pneumatic gate valves for adjacent chambers will be enabled when both chambers are below their lower setpoint pressure (about $5 \times 10^{-6}$ Torr). Once all chambers are below their lower setpoints, the OMRON considers the system is to be fully pumped down.
\end{enumerate}

Experiments can be performed in this state. When running the gas valve in the target chamber, it is advised to turn on the secondary blower to reduce the chamber pressure.

When experiments are not being performed in the apparatus, the system should be left in an idle UHV state with the blowers off, the chambers armed, and the gate valves separating the chambers closed. To arm the chambers press ESC + [chamber number] on the OMRON interface. The OMRON's display will update to show which chambers are armed (G: generation \& differential pump chambers, M: mirror chamber, T: target chamber, S: photon spectrometer chamber).

\section{TABLe / OMRON Venting Procedure}
\label{sec:OMRON_venting}

The OMRON system was designed with the ability to vent any single chamber or combination of chambers while keeping the rest of the system at UHV. This is not always possible, now that the mirror, target \& spectrometer chambers share a single blower system and exhaust line. Any chambers that share a common exhaust line must be vented simultaneously or else you risk putting unnecessary mechanical stress on the remaining powered turbopumps. For example, the mirror, target \& spectrometer chambers share a common exhaust line, and attempting to vent the one chamber while keeping the other turbos running will cause a turbopump error.

The following venting procedure assumes an initial condition of all chambers pumped down to UHV with the turbos running:

\begin{enumerate}
	\item Turn off all gas sources. If the HPC is installed, follow the HPC shutdown procedure.
	\item Turn off the blowers.
	\item Block the laser into the interferometer.
	\item Close the pneumatic gate valves separating the chambers.
	\item Disarm the chambers by pressing ESC + [chamber number].
	\item Verify the OMRON's safety system is not disabled by checking the bypass switch.
	\item Enable the venting valves by switching ON the vent \& purge controls on the control panel. Note: the chambers will not vent without this step.
	\item Remove the KF clamps on the blow-off valves.
	\item Start the OMRON venting script by pressing ALT + [chamber number] on the OMRON.
	\item The user can now walk away from the system. It will take a few hours to vent.
\end{enumerate}

Once initiated, the venting script will immediately stop the turbopump's motors, open the solenoid venting valves after 30 seconds, close the roughing line's solenoid valves after 30 minutes, and close the solenoid venting valves after about 5 hours. The preceding timeline was chosen following the manufacturer's recommendation, and to avoid closing the roughing line's valves before the turbos had completely stopped spinning. 
%
%\section{Aligning the Interferometer}
%\label{app:aligning-interferometer}
%
%\subsection{the generation arm}
%
%\subsubsection{The Ellipsoidal Mirror}
%
%\subsection{the pump arm}
%
%\subsection{finding spatial overlap}
%
%\subsection{finding temporal overlap}
%
%\section{Pointing into the Interferometer}
%\label{app:pointing-into-TABLE}
%
%the importance of pointing into the interferometer - spatial and temporal alignment

\section{The High Pressure Cell (HPC)}
\label{app:HPC_instructions}

\subsection{Introduction}

\begin{figure}
	\centering
	\includegraphics[width=0.9\textwidth]{figures/app1/HPC_outside_view_lowres.png}
	\caption{Exterior view of the generation chamber with the HPC installed showing the ancillary vacuum hardware. Red arrow indicates input laser path; green arrow points towards the HPC's RV pump.}
	\label{fig:HPC_outside_view}
\end{figure}

This section will discuss the initial setup and day-to-day operational details of the HPC. For the HHG \& pressure performance of the HPC, see \cref{sec:HPC}.

\cref{fig:HPC_outside_view} is a picture of the exterior of the generation chamber, showing the ancillary vacuum hardware associated with the HPC. A small oil-lubricated RV pump (not shown) provides differential pumping to the interior of the HPC. An inline Baratron diaphragm pressure sensor (effective range: 1 -- 760 Torr) tracks the interior pressure of the edge-welded bellows. A manual gate valve is used to isolate the HPC from the RV pump when the additional pumping is not needed. Right angle KF fittings were used to route the HPC's vacuum line above the pump arm of the interferometer. The optics associated with the interferometer can be seen below this hardware on the optical table.

\subsection{Initial Installation and Alignment}
\label{app:initial-alignment-HPC}
First, a note about laser safety. As with all initial alignment procedures, the lowest possible laser power should be used to minimize both accidental laser drilling of the HPC components and the danger posed by stray light and back reflections. The surface most likely to cause laser scatter is the front face of the outer pipe, which is roughened stainless steel located about 3/8" before the focus. The material's roughness and the negative radius of curvature of the incoming light make it unlikely that incident light will coherently focus to a point upon reflection. However, it is possible that the sidewall of the outer pipe's aperture could act as a focusing mirror. Additionally, the hose clamp or mounting bracket could coherently reflect light towards the user. The user is advised to strictly follow all laser safety protocols during this alignment procedure. Whenever possible, direct observation of the laser beam on the surfaces of the HPC should be avoided, instead a remote sensor (webcam) should be used to view the interior of the generation chamber.

The initial installation of the HPC can be time consuming and tedious, but once installed it will retain its alignment indefinitely (weeks or months). As with a free expansion gas jet, fine adjustments to the pointing should be done on a daily basis. However, these adjustments will usually take less than 5 minutes.

Optically, the HPC cell consists of four pinhole apertures (diametrically opposed pinholes on both the inner and outer pipes) with the laser focus near the center of the inner pipe. The optical transmission of the HPC is therefore very sensitive to the relative alignment of these components, as well as the pointing of the laser into the HPC. To simplify the alignment, the two innermost holes are laser drilled \textit{in-situ} after the outer pipe's apetures have been aligned. To maintain the relative alignment of the inner \& outer apertures, the user should refrain from adjusting the inner pipe after the initial alignment is completed. Therefore, daily alignment of the HPC consists of moving all four apertures together via the in-vacuum motorized XYZ stages.

The first step of the HPC installation is installing the rough vacuum feedthrough flange. The TABLe's generation chamber uses a custom flange (a 4.5" CF blank with a KF16 half nipple welded to the air-side and a KF16 bulkhead groove \& tapped holes machined into the vacuum side). To accommodate the length of the in-vacuum bellows, we use a custom 10" CF to 4.5" CF reducing nipple (OAL = 4.25"), which acts as a spacer between the feedthrough flange and the KF16 flange on the HPC. Although it was absent from the original design, a spacer 4.5" CF flange (thickness = 0.68") between the feedthrough and the reducing full nipple is used to relax the compression in the bellows and allow for a larger range of motion. For convenience, the user should install the edge-welded bellows to the bulkhead flange before installing the flange on the chamber.

\begin{figure}
	\centering
	\includegraphics[width=0.9\textwidth]{figures/app1/HPC_on_stage2.png}
	\caption{The HPC with bracket installed on the XYZ translation stage, configured for the generation chamber. The hose clamp and gas supply tube are omitted from this drawing for clarity.}
	\label{fig:HPC_on_stage}
\end{figure}

The supporting bracket for the HPC was designed to be interchangeable with the free gas nozzle's bracket. See \cref{fig:HPC_on_stage} for details. This design allows the user to change the gas source type (HPC, LPC, or free expansion jet) without disturbing the alignment of the XYZ stage to the optical axis. For completeness, we will assume that the XYZ stage has been misaligned or removed from the chamber. First, the user should align the laser to the interferometer so that the laser path in the generation chamber can be used as a reference. Then, the stage should be positioned in the chamber so that the focus is roughly in the center of the stage's motion. Next, the stage's k-direction should made parallel to the optic axis. This can be done by tracking the position of the laser on a card mounted to the stage while moving the stage upstream and downstream of the focus. After clamping the stage to the breadboard, check that the alignment is still true before continuing to the next step.

Now that the XYZ stage is aligned to the laser, we will install the outer pipe of the HPC and align its apertures to the laser. We do so by maximizing the light transmission through the apertures. This is best done in two steps: first, coarse alignment is done visually at low power (insufficiently intense to laser drill the pipe), followed by fine adjustments using a power meter at moderate intensities (above the noise floor of the power meter). Note that drilling out the outer pipe's apertures will lessen the alignment requirements at the expense of the HPC's differential pumping performance. Given the chamber's small internal dimensions, it is not practical to place a power meter in the generation chamber after the focus during the alignment procedure. Rather, it is preferable to divert the beam out of the vacuum system using the linear actuator \& silver mirror assembly located approximately 85 cm downstream of the focus.\footnote{Special thanks to Eric Moore for designing and installing this optomechanical component.} When inserted, the linear actuator intersects the beam path and redirects the beam out of the vacuum system through a window onto the upper deck of the optical table. Note that for most generation focusing conditions, the large beam size at the diverting mirror makes this beam path lossy. To accurately calculate the transmission through the HPC, it is necessary to measure the power immediately after the HPC in the generation chamber.

For the coarse alignment, the input beam intensity should be reduced using an upstream iris, to the point that it is barely visible near the focus. Since a tightened KF connection prevents rotation of the components, the alignment of the outer pipe is done prior to making any KF connections. However, the KF clamps should be fitted on either end of the pipe to ensure that there is enough room to make the connections without disturbing the alignment once finished. The outer pipe should be placed in its cradle, with the aluminum \& hose clamps made snug around the pipe but not taut.\footnote{The HPC's XYZ assembly and bracket were designed for the TABLe generation chamber. If it is being installed elsewhere, the user should verify that the height is correct. When installed correctly, the bottom of the Z-motor range should correspond to the HPC lowered completely out of the way of the laser; the top of the range should correspond to the laser going through the center of the HPC, with about 1 mm to spare.} Transmission should be optimized by iteratively tuning the following parameters: (1) rotation of the pipe in the cradle, (2) height of the cradle using the vertical motor, and (3) horizontal (transverse) position of the assembly using both the horizontal motor and the position of the pipe in the cradle. For fine adjustment, the iris should be adjusted so that the power meter reads about 20--30 mW when measured after the linear actuator.\footnote{This power is appropriate for a 1kHz repetition rate and a generation focal length of 30 or 40 cm.} The clamps should be tightened so that movement of the pipe is possible, but difficult. The area around the power meter should be covered to prevent air currents from affecting the measurement. The transmitted power should be optimized using the same procedure as before.

\begin{figure}
	\centering
	\includegraphics[width=0.9\textwidth]{figures/app1/HPC_outer_can_laser.jpg}
	\caption{Laser filament in the aligned outer pipe of the HPC. The inner pipe is not yet installed. Alignment of the outer pipe is done at very low intensities; after alignment the power was increased to create a filament for illustrative purposes only. This picture was taken in the target chamber during the initial testing of the HPC. The geometry of this chamber requires that the orientation of the mounting bracket is reversed compared to what is shown in \cref{fig:HPC_on_stage}.}
	\label{fig:HPC_outer_can_laser}
\end{figure}

Once the outer pipe is aligned, tighten all connections and connect the bellows to the outer pipe. Check that the alignment has not been changed by torquing these connections. Verify that the unattenuated laser beam can pass through the outer pipe without interference, as shown in \cref{fig:HPC_outer_can_laser}. 

Next we install the inner pipe. First, attach the gas delivery feedthrough flange onto the HPC assembly without the inner pipe. Being mindful to not disturb the alignment of the outer pipe, check that the gas delivery tubing does not interfere with the laser path. Remove the gas delivery feedthrough flange, cut the inner pipe to length (OAL = 1.75"), and make the Swagelok connection between the inner pipe and the KF feedthrough. Make sure the inner pipe is normal to the flange's sealing surface, otherwise the laser will skim the sidewall of the inner pipe rather than go through the center. If this happens, the laser-gas interaction length will be reduced and the HHG performance of the system will suffer. Install the gas delivery assembly onto the HPC assembly by tightening the KF clamp.

Laser drilling the inner pipe will sputter a significant amount of metal onto the inner surfaces of the chamber. It is therefore important to protect the optics from the resulting metal plume. Since the active drilling surface is on the upstream face of the pipe, most of the material will go upstream. Therefore, the laser window needs to be swapped out for a "sacrificial" window prior to drilling.\footnote{Using a window with different optical properties (i.e., thickness or material), or no window at all, will change the pointing and effective focal length of the beam. It has been suggested that the laser window could be protected by placing a thin sheet of transparent plastic between the window and the HPC, but this method has not been tested.} Out of an abundance of caution, close the gate valve to the mirror chamber, retract the linear actuator \& silver mirror from the beam path, and block the generation chamber's vacuum aperture with a card.

Laser drilling should be done with the appropriate safety precautions: wearing laser goggles, notifying coworkers of your activity, and posting signs on the entrances to the lab to reduce unnecessary foot traffic. The user can cover up the chamber's flanges and set up a webcam to remotely monitor the laser drilling status to minimize the risk of inadvertent laser exposure.

\begin{figure}
	\centering
	\includegraphics[angle=270, width=0.75\textwidth]{figures/app1/HPC_drilling.jpg}
	\caption{Laser drilling the inner pipe. A card blocks the laser after exiting the HPC. The HPC was pressurized with Ar gas to enhance the filament for illustrative purposes.}
	\label{fig:HPC_drilling}
\end{figure}

At this point, the actual process of laser drilling is quite simple. There is no way to control the exact positioning of the inner pipe relative to the outer pipe, so there are no adjustments to make. Rather, the design relies on the mechanical alignment of the inner pipe relative to the outer pipe, which is ultimately set by the gas feedthrough weld, the Swagelok / KF fittings. On the other hand, a used inner pipe cannot be reinstalled to the HPC once it is removed, since alignment is effectively impossible. To laser drill the pipe, simply let the unattenuated beam into the chamber and wait a few minutes until the laser emerges from the exit of the HPC. See \cref{fig:HPC_drilling}.

If you are planning on scanning the k-direction of the HPC during an experiment, you should do so now while you are set up for laser drilling. Similarly, if you are using a chromatic focusing scheme and are planning on changing wavelengths during your experiment (which will change the effective focal length), you should step through the full range of wavelengths while drilling. Doing so will open up the apertures slightly, resulting in additional metal deposition on the sacrificial laser window and reduced HPC pressure performance.

After laser drilling is complete, reinstall the laser window and verify the HPC has retained its alignment.

\subsection{Alignment with the HPC Installed}

Once lowered out of the beam path, the HPC does not affect the daily pointing procedure. However, there are some extra considerations that need to be made if the HPC is installed. The small apertures of the HPC and the non-linear nature of HHG demand high accuracy in the pointing into the cell, so small corrections to the positioning of the HPC have to be made after the daily pointing procedure is completed.

\subsubsection{Pointing into the Interferometer}
If the interferometer is already aligned, the presence of the HPC does not really complicate the daily procedure of the beamline. In this case, the user should block the laser into the generation chamber and align the pointing into the interferometer using the pump arm, via the standard procedure.\footnote{Failure to block the laser prior to changing the pointing may result in laser-drilling the HPC.} Since harmonic generation is extremely sensitive to pointing, the user may have to make small tweaks to the transverse position of the HPC after a nominal alignment of the interferometer. This can be achieved by setting a fast camera exposure ($< 0.5$ s) and optimizing HHG yield with respect to the HPC vertical \& horizontal position (take 10 - 25 $\mu$m steps). In our experience, the optimal HPC position is typically within {50 $\mu$m} of the previous day's position.

The HPC's apertures may no longer be circular if the HPC has been subjected to accidental laser drilling or significant laser drift. Non-circular apertures may result in a complex spatial profile of the harmonics, which can make the optimization of the harmonic yield difficult.

\subsubsection{Aligning the Interferometer}
If the interferometer needs to be realigned, then the HPC must be lowered out of the way of the laser path. Unless major changes were made to the interferometer, the angle of the HPC's apertures should remain aligned to the k-vector of the generation arm. In this case, the optimal position of the HPC can be found by maximizing the transmitted power of an the attenuated laser through the HPC, as described in the latter part of \cref{app:initial-alignment-HPC}. If major modifications were made to the interferometer, the user should consider aligning the HPC from scratch.


\subsection{Using the HPC}

The extra vacuum pump adds complications to the TABLe operating procedures described in \cref{sec:OMRON_instructions,sec:OMRON_venting}. Before pumping the TABLe down from atmosphere, ensure the HPC's manual value (see \cref{fig:HPC_outside_view}) is closed.

Increase the backing pressure until the internal bellows pressure reaches $\sim$20 Torr, or until the generation chamber reaches $\sim$3 mTorr (whichever happens first). Then, turn on the HPC's RV pump and slowly open the external manual valve. The pressure inside the system should drop. This order of operations ensures the net gas flow direction will be into the RV pump rather than the generation chamber, minimizing the chance of oil contamination in the main TABLe apparatus. Once the HPC's vacuum pump is running, the user can continue increasing the backing pressure until the operating pressure is reached. \textbf{Note that the pressure differential between the inside of the HPC's bellows and the generation chamber cannot exceed 120 Torr without risking damage to the HPC's bellows.} The shutdown procedure follows the same steps, but in reverse. Lower the backing pressure to the HPC until the generation chamber pressure is $\sim 5 \times 10^{-4}$ Torr, then close the manual valve to the HPC's RV pump. Once the valve is closed, shut off the gas completely and turn off the RV pump. The TABLe system is now ready to be vented (\cref{sec:OMRON_venting}), or left idle.

%For daily alignment, insert the diagnostic mirror into the beam path and measure the transmitted power of an attenuated beam. If the recorded number is lower than expected, optimization can be performed by using the HPC's vertical \& horizontal motors. Once satisfied with the initial alignment, we can begin to let the generating gas into the chamber. Starting with zero backing pressure, open the gas valve to the inner pipe. This will cause the pressure of the inner pipe and the generation chamber to quickly rise, then stabilize. 

%\subsection{Startup and Shutdown}


%\section{Laser System Specifics}
%importance of pointing \& laser performance for our experiments
%
%\subsection{The Spitfire}
%\subsubsection{Regular Maintenance}
%- cleaning the stretcher
%
%- increasing the pump laser currents
%
%- changing the chiller fluid, desiccants, etc
%
%\subsubsection{quirks and features}
%- regen cavity tweaks
%
%- photodiode problems
%
%- software issues - bugs and troubleshooting
%
%\subsection{Pointing Stabilization into the External Compressor}
%- dietrich plots for pointing
%\subsection{The Spitfire's External Compressor}
%\subsubsection{external compressor alignment}
%\subsubsection{cleaning the grating}
%
%\subsection{The TOPAS-HE}
%- aligning
%- importance of power stability and pointing stability
%
%\subsection{stability}
%- boxing things up
%- power stability throughout the day, people in the lab
%- unstable harmonic yield from the HPC at high pressures
%\section{The Shutter System}
%
%\section{The Vacuum System}
%- blower upgrade
%
%- remote pressure sensing
%
%- vacuum calculations for steady state pressure of beamline
%
%\section{Required Maintenance}
%
%vacuum system (rough pumps, blowers, turbos) and spitfire (cleaning the gratings)
%
%\section{Best practices: data acquisition}
%
%read-out noise from camera. (how noise scales)

 
\end{appendices}

\backmatter
% We use BIBTeX for the bibliography---you don't have to
% \nocite{*} % To display all refs, even uncited refs (useful when editing)
% \bibliography{templatebib} % manual bibtex file
\bibliography{Zotero} % automatically updated zotero better bibtex file
\bibliographystyle{abbrv} % use your favorite BIBTeX style

% Note: GS 2010 requires bibliography/references _before_ the appendix
% if you believe their guidelines; however, conversations with GS
% staff suggests _they don't care_. Go figure. So do what you like.

%\appendix
%\chapter{Guidelines for using the TABLe}
\label{appendix:TABLe_manual}

\section{TABLe / OMRON Pump-down Procedure}
\label{sec:OMRON_instructions}

Special thanks to Andrew Piper for coding and wiring the OMRON safety system. Below is an operational guideline to pump the system down to UHV using the OMRON system. Please see Andrew Piper's OMRON manual for additional details on the microcontroller system.

This procedure assumes that the chambers are initially at atmospheric pressure, the rough pumps are turned on, and the solenoid shutoff valves on the roughing line are closed.

\begin{enumerate}
	\item Seal and isolate all chambers. Close the manual valves between each turbo pump and the solenoid shutoff valves. Reattach any blow-off flanges (KF-25 blanks) that may have come off from the previous venting cycle.
	\item (Optional) As a test, ensure the OMRON's safety system is engaged by attempting to open one of the pneumatic gate valves via the control panel. \textbf{Caution: operating a gate valve between two chambers with a pressure differential can cause catastrophic system failure. Only perform this step after you have verified that all chambers are at atmospheric pressure.} If the gate valve opens while both chambers are above the upper setpoint, then the OMRON safety system is disabled. If the OMRON fails this test, check the status of the override switch.
	\item Retract the metal filters from the beam path to protect them from potential pressure surges. 
	\item Initiate the pump-down sequence by pressing the OK button on the OMRON. This will open all solenoid valves simultaneously with a loud \textit{thunk}.
	\item Slowly open the manual valves while monitoring the gas load on the rough pumps. Use the two remote pressure monitoring systems (Raspberry Pis) to monitor the inlet pressure for each blower system. As a rule of thumb, try to keep the inlet pressure below $\sim$100 Torr during this step. \textbf{Warning: quickly opening the manual values at atmosphere will result in pump oil being expelled into the rough pump's exhaust line. Continuing to run in this condition can lead to the overheating and eventual seizure of the rough pumps} \cite{kiesewetterDynamicsNearThresholdAttosecond2019}.
	\item Once the manual valves are fully open on all chambers, you can turn each blower system ON to accelerate the remaining pump-down procedure.
	\item Power on the turbo power supplies and switch the turbos to ON. After a few seconds, the magnetically levitated turbos will start levitating with a soft \textit{thunk}.
	\item Each turbo will automatically start spinning when its chamber reaches the upper set point ($\sim$200 mTorr). The turbos will take a few minutes to reach their final speed.
	\item Wait for the system to pump down. It typically takes 15-45 minutes for the entire system to reach $10^{-6}$ Torr.
	\item The pneumatic gate valves for adjacent chambers will be enabled when both chambers are below their lower setpoint pressure (about $5 \times 10^{-6}$ Torr). Once all chambers are below their lower setpoints, the OMRON considers the system is to be fully pumped down.
\end{enumerate}

Experiments can be performed in this state. When running the gas valve in the target chamber, it is advised to turn on the secondary blower to reduce the chamber pressure.

When experiments are not being performed in the apparatus, the system should be left in an idle UHV state with the blowers off, the chambers armed, and the gate valves separating the chambers closed. To arm the chambers press ESC + [chamber number] on the OMRON interface. The OMRON's display will update to show which chambers are armed (G: generation \& differential pump chambers, M: mirror chamber, T: target chamber, S: photon spectrometer chamber).

\section{TABLe / OMRON Venting Procedure}
\label{sec:OMRON_venting}

The OMRON system was designed with the ability to vent any single chamber or combination of chambers while keeping the rest of the system at UHV. This is not always possible, now that the mirror, target \& spectrometer chambers share a single blower system and exhaust line. Any chambers that share a common exhaust line must be vented simultaneously or else you risk putting unnecessary mechanical stress on the remaining powered turbopumps. For example, the mirror, target \& spectrometer chambers share a common exhaust line, and attempting to vent the one chamber while keeping the other turbos running will cause a turbopump error.

The following venting procedure assumes an initial condition of all chambers pumped down to UHV with the turbos running:

\begin{enumerate}
	\item Turn off all gas sources. If the HPC is installed, follow the HPC shutdown procedure.
	\item Turn off the blowers.
	\item Block the laser into the interferometer.
	\item Close the pneumatic gate valves separating the chambers.
	\item Disarm the chambers by pressing ESC + [chamber number].
	\item Verify the OMRON's safety system is not disabled by checking the bypass switch.
	\item Enable the venting valves by switching ON the vent \& purge controls on the control panel. Note: the chambers will not vent without this step.
	\item Remove the KF clamps on the blow-off valves.
	\item Start the OMRON venting script by pressing ALT + [chamber number] on the OMRON.
	\item The user can now walk away from the system. It will take a few hours to vent.
\end{enumerate}

Once initiated, the venting script will immediately stop the turbopump's motors, open the solenoid venting valves after 30 seconds, close the roughing line's solenoid valves after 30 minutes, and close the solenoid venting valves after about 5 hours. The preceding timeline was chosen following the manufacturer's recommendation, and to avoid closing the roughing line's valves before the turbos had completely stopped spinning. 
%
%\section{Aligning the Interferometer}
%\label{app:aligning-interferometer}
%
%\subsection{the generation arm}
%
%\subsubsection{The Ellipsoidal Mirror}
%
%\subsection{the pump arm}
%
%\subsection{finding spatial overlap}
%
%\subsection{finding temporal overlap}
%
%\section{Pointing into the Interferometer}
%\label{app:pointing-into-TABLE}
%
%the importance of pointing into the interferometer - spatial and temporal alignment

\section{The High Pressure Cell (HPC)}
\label{app:HPC_instructions}

\subsection{Introduction}

\begin{figure}
	\centering
	\includegraphics[width=0.9\textwidth]{figures/app1/HPC_outside_view_lowres.png}
	\caption{Exterior view of the generation chamber with the HPC installed showing the ancillary vacuum hardware. Red arrow indicates input laser path; green arrow points towards the HPC's RV pump.}
	\label{fig:HPC_outside_view}
\end{figure}

This section will discuss the initial setup and day-to-day operational details of the HPC. For the HHG \& pressure performance of the HPC, see \cref{sec:HPC}.

\cref{fig:HPC_outside_view} is a picture of the exterior of the generation chamber, showing the ancillary vacuum hardware associated with the HPC. A small oil-lubricated RV pump (not shown) provides differential pumping to the interior of the HPC. An inline Baratron diaphragm pressure sensor (effective range: 1 -- 760 Torr) tracks the interior pressure of the edge-welded bellows. A manual gate valve is used to isolate the HPC from the RV pump when the additional pumping is not needed. Right angle KF fittings were used to route the HPC's vacuum line above the pump arm of the interferometer. The optics associated with the interferometer can be seen below this hardware on the optical table.

\subsection{Initial Installation and Alignment}
\label{app:initial-alignment-HPC}
First, a note about laser safety. As with all initial alignment procedures, the lowest possible laser power should be used to minimize both accidental laser drilling of the HPC components and the danger posed by stray light and back reflections. The surface most likely to cause laser scatter is the front face of the outer pipe, which is roughened stainless steel located about 3/8" before the focus. The material's roughness and the negative radius of curvature of the incoming light make it unlikely that incident light will coherently focus to a point upon reflection. However, it is possible that the sidewall of the outer pipe's aperture could act as a focusing mirror. Additionally, the hose clamp or mounting bracket could coherently reflect light towards the user. The user is advised to strictly follow all laser safety protocols during this alignment procedure. Whenever possible, direct observation of the laser beam on the surfaces of the HPC should be avoided, instead a remote sensor (webcam) should be used to view the interior of the generation chamber.

The initial installation of the HPC can be time consuming and tedious, but once installed it will retain its alignment indefinitely (weeks or months). As with a free expansion gas jet, fine adjustments to the pointing should be done on a daily basis. However, these adjustments will usually take less than 5 minutes.

Optically, the HPC cell consists of four pinhole apertures (diametrically opposed pinholes on both the inner and outer pipes) with the laser focus near the center of the inner pipe. The optical transmission of the HPC is therefore very sensitive to the relative alignment of these components, as well as the pointing of the laser into the HPC. To simplify the alignment, the two innermost holes are laser drilled \textit{in-situ} after the outer pipe's apetures have been aligned. To maintain the relative alignment of the inner \& outer apertures, the user should refrain from adjusting the inner pipe after the initial alignment is completed. Therefore, daily alignment of the HPC consists of moving all four apertures together via the in-vacuum motorized XYZ stages.

The first step of the HPC installation is installing the rough vacuum feedthrough flange. The TABLe's generation chamber uses a custom flange (a 4.5" CF blank with a KF16 half nipple welded to the air-side and a KF16 bulkhead groove \& tapped holes machined into the vacuum side). To accommodate the length of the in-vacuum bellows, we use a custom 10" CF to 4.5" CF reducing nipple (OAL = 4.25"), which acts as a spacer between the feedthrough flange and the KF16 flange on the HPC. Although it was absent from the original design, a spacer 4.5" CF flange (thickness = 0.68") between the feedthrough and the reducing full nipple is used to relax the compression in the bellows and allow for a larger range of motion. For convenience, the user should install the edge-welded bellows to the bulkhead flange before installing the flange on the chamber.

\begin{figure}
	\centering
	\includegraphics[width=0.9\textwidth]{figures/app1/HPC_on_stage2.png}
	\caption{The HPC with bracket installed on the XYZ translation stage, configured for the generation chamber. The hose clamp and gas supply tube are omitted from this drawing for clarity.}
	\label{fig:HPC_on_stage}
\end{figure}

The supporting bracket for the HPC was designed to be interchangeable with the free gas nozzle's bracket. See \cref{fig:HPC_on_stage} for details. This design allows the user to change the gas source type (HPC, LPC, or free expansion jet) without disturbing the alignment of the XYZ stage to the optical axis. For completeness, we will assume that the XYZ stage has been misaligned or removed from the chamber. First, the user should align the laser to the interferometer so that the laser path in the generation chamber can be used as a reference. Then, the stage should be positioned in the chamber so that the focus is roughly in the center of the stage's motion. Next, the stage's k-direction should made parallel to the optic axis. This can be done by tracking the position of the laser on a card mounted to the stage while moving the stage upstream and downstream of the focus. After clamping the stage to the breadboard, check that the alignment is still true before continuing to the next step.

Now that the XYZ stage is aligned to the laser, we will install the outer pipe of the HPC and align its apertures to the laser. We do so by maximizing the light transmission through the apertures. This is best done in two steps: first, coarse alignment is done visually at low power (insufficiently intense to laser drill the pipe), followed by fine adjustments using a power meter at moderate intensities (above the noise floor of the power meter). Note that drilling out the outer pipe's apertures will lessen the alignment requirements at the expense of the HPC's differential pumping performance. Given the chamber's small internal dimensions, it is not practical to place a power meter in the generation chamber after the focus during the alignment procedure. Rather, it is preferable to divert the beam out of the vacuum system using the linear actuator \& silver mirror assembly located approximately 85 cm downstream of the focus.\footnote{Special thanks to Eric Moore for designing and installing this optomechanical component.} When inserted, the linear actuator intersects the beam path and redirects the beam out of the vacuum system through a window onto the upper deck of the optical table. Note that for most generation focusing conditions, the large beam size at the diverting mirror makes this beam path lossy. To accurately calculate the transmission through the HPC, it is necessary to measure the power immediately after the HPC in the generation chamber.

For the coarse alignment, the input beam intensity should be reduced using an upstream iris, to the point that it is barely visible near the focus. Since a tightened KF connection prevents rotation of the components, the alignment of the outer pipe is done prior to making any KF connections. However, the KF clamps should be fitted on either end of the pipe to ensure that there is enough room to make the connections without disturbing the alignment once finished. The outer pipe should be placed in its cradle, with the aluminum \& hose clamps made snug around the pipe but not taut.\footnote{The HPC's XYZ assembly and bracket were designed for the TABLe generation chamber. If it is being installed elsewhere, the user should verify that the height is correct. When installed correctly, the bottom of the Z-motor range should correspond to the HPC lowered completely out of the way of the laser; the top of the range should correspond to the laser going through the center of the HPC, with about 1 mm to spare.} Transmission should be optimized by iteratively tuning the following parameters: (1) rotation of the pipe in the cradle, (2) height of the cradle using the vertical motor, and (3) horizontal (transverse) position of the assembly using both the horizontal motor and the position of the pipe in the cradle. For fine adjustment, the iris should be adjusted so that the power meter reads about 20--30 mW when measured after the linear actuator.\footnote{This power is appropriate for a 1kHz repetition rate and a generation focal length of 30 or 40 cm.} The clamps should be tightened so that movement of the pipe is possible, but difficult. The area around the power meter should be covered to prevent air currents from affecting the measurement. The transmitted power should be optimized using the same procedure as before.

\begin{figure}
	\centering
	\includegraphics[width=0.9\textwidth]{figures/app1/HPC_outer_can_laser.jpg}
	\caption{Laser filament in the aligned outer pipe of the HPC. The inner pipe is not yet installed. Alignment of the outer pipe is done at very low intensities; after alignment the power was increased to create a filament for illustrative purposes only. This picture was taken in the target chamber during the initial testing of the HPC. The geometry of this chamber requires that the orientation of the mounting bracket is reversed compared to what is shown in \cref{fig:HPC_on_stage}.}
	\label{fig:HPC_outer_can_laser}
\end{figure}

Once the outer pipe is aligned, tighten all connections and connect the bellows to the outer pipe. Check that the alignment has not been changed by torquing these connections. Verify that the unattenuated laser beam can pass through the outer pipe without interference, as shown in \cref{fig:HPC_outer_can_laser}. 

Next we install the inner pipe. First, attach the gas delivery feedthrough flange onto the HPC assembly without the inner pipe. Being mindful to not disturb the alignment of the outer pipe, check that the gas delivery tubing does not interfere with the laser path. Remove the gas delivery feedthrough flange, cut the inner pipe to length (OAL = 1.75"), and make the Swagelok connection between the inner pipe and the KF feedthrough. Make sure the inner pipe is normal to the flange's sealing surface, otherwise the laser will skim the sidewall of the inner pipe rather than go through the center. If this happens, the laser-gas interaction length will be reduced and the HHG performance of the system will suffer. Install the gas delivery assembly onto the HPC assembly by tightening the KF clamp.

Laser drilling the inner pipe will sputter a significant amount of metal onto the inner surfaces of the chamber. It is therefore important to protect the optics from the resulting metal plume. Since the active drilling surface is on the upstream face of the pipe, most of the material will go upstream. Therefore, the laser window needs to be swapped out for a "sacrificial" window prior to drilling.\footnote{Using a window with different optical properties (i.e., thickness or material), or no window at all, will change the pointing and effective focal length of the beam. It has been suggested that the laser window could be protected by placing a thin sheet of transparent plastic between the window and the HPC, but this method has not been tested.} Out of an abundance of caution, close the gate valve to the mirror chamber, retract the linear actuator \& silver mirror from the beam path, and block the generation chamber's vacuum aperture with a card.

Laser drilling should be done with the appropriate safety precautions: wearing laser goggles, notifying coworkers of your activity, and posting signs on the entrances to the lab to reduce unnecessary foot traffic. The user can cover up the chamber's flanges and set up a webcam to remotely monitor the laser drilling status to minimize the risk of inadvertent laser exposure.

\begin{figure}
	\centering
	\includegraphics[angle=270, width=0.75\textwidth]{figures/app1/HPC_drilling.jpg}
	\caption{Laser drilling the inner pipe. A card blocks the laser after exiting the HPC. The HPC was pressurized with Ar gas to enhance the filament for illustrative purposes.}
	\label{fig:HPC_drilling}
\end{figure}

At this point, the actual process of laser drilling is quite simple. There is no way to control the exact positioning of the inner pipe relative to the outer pipe, so there are no adjustments to make. Rather, the design relies on the mechanical alignment of the inner pipe relative to the outer pipe, which is ultimately set by the gas feedthrough weld, the Swagelok / KF fittings. On the other hand, a used inner pipe cannot be reinstalled to the HPC once it is removed, since alignment is effectively impossible. To laser drill the pipe, simply let the unattenuated beam into the chamber and wait a few minutes until the laser emerges from the exit of the HPC. See \cref{fig:HPC_drilling}.

If you are planning on scanning the k-direction of the HPC during an experiment, you should do so now while you are set up for laser drilling. Similarly, if you are using a chromatic focusing scheme and are planning on changing wavelengths during your experiment (which will change the effective focal length), you should step through the full range of wavelengths while drilling. Doing so will open up the apertures slightly, resulting in additional metal deposition on the sacrificial laser window and reduced HPC pressure performance.

After laser drilling is complete, reinstall the laser window and verify the HPC has retained its alignment.

\subsection{Alignment with the HPC Installed}

Once lowered out of the beam path, the HPC does not affect the daily pointing procedure. However, there are some extra considerations that need to be made if the HPC is installed. The small apertures of the HPC and the non-linear nature of HHG demand high accuracy in the pointing into the cell, so small corrections to the positioning of the HPC have to be made after the daily pointing procedure is completed.

\subsubsection{Pointing into the Interferometer}
If the interferometer is already aligned, the presence of the HPC does not really complicate the daily procedure of the beamline. In this case, the user should block the laser into the generation chamber and align the pointing into the interferometer using the pump arm, via the standard procedure.\footnote{Failure to block the laser prior to changing the pointing may result in laser-drilling the HPC.} Since harmonic generation is extremely sensitive to pointing, the user may have to make small tweaks to the transverse position of the HPC after a nominal alignment of the interferometer. This can be achieved by setting a fast camera exposure ($< 0.5$ s) and optimizing HHG yield with respect to the HPC vertical \& horizontal position (take 10 - 25 $\mu$m steps). In our experience, the optimal HPC position is typically within {50 $\mu$m} of the previous day's position.

The HPC's apertures may no longer be circular if the HPC has been subjected to accidental laser drilling or significant laser drift. Non-circular apertures may result in a complex spatial profile of the harmonics, which can make the optimization of the harmonic yield difficult.

\subsubsection{Aligning the Interferometer}
If the interferometer needs to be realigned, then the HPC must be lowered out of the way of the laser path. Unless major changes were made to the interferometer, the angle of the HPC's apertures should remain aligned to the k-vector of the generation arm. In this case, the optimal position of the HPC can be found by maximizing the transmitted power of an the attenuated laser through the HPC, as described in the latter part of \cref{app:initial-alignment-HPC}. If major modifications were made to the interferometer, the user should consider aligning the HPC from scratch.


\subsection{Using the HPC}

The extra vacuum pump adds complications to the TABLe operating procedures described in \cref{sec:OMRON_instructions,sec:OMRON_venting}. Before pumping the TABLe down from atmosphere, ensure the HPC's manual value (see \cref{fig:HPC_outside_view}) is closed.

Increase the backing pressure until the internal bellows pressure reaches $\sim$20 Torr, or until the generation chamber reaches $\sim$3 mTorr (whichever happens first). Then, turn on the HPC's RV pump and slowly open the external manual valve. The pressure inside the system should drop. This order of operations ensures the net gas flow direction will be into the RV pump rather than the generation chamber, minimizing the chance of oil contamination in the main TABLe apparatus. Once the HPC's vacuum pump is running, the user can continue increasing the backing pressure until the operating pressure is reached. \textbf{Note that the pressure differential between the inside of the HPC's bellows and the generation chamber cannot exceed 120 Torr without risking damage to the HPC's bellows.} The shutdown procedure follows the same steps, but in reverse. Lower the backing pressure to the HPC until the generation chamber pressure is $\sim 5 \times 10^{-4}$ Torr, then close the manual valve to the HPC's RV pump. Once the valve is closed, shut off the gas completely and turn off the RV pump. The TABLe system is now ready to be vented (\cref{sec:OMRON_venting}), or left idle.

%For daily alignment, insert the diagnostic mirror into the beam path and measure the transmitted power of an attenuated beam. If the recorded number is lower than expected, optimization can be performed by using the HPC's vertical \& horizontal motors. Once satisfied with the initial alignment, we can begin to let the generating gas into the chamber. Starting with zero backing pressure, open the gas valve to the inner pipe. This will cause the pressure of the inner pipe and the generation chamber to quickly rise, then stabilize. 

%\subsection{Startup and Shutdown}


%\section{Laser System Specifics}
%importance of pointing \& laser performance for our experiments
%
%\subsection{The Spitfire}
%\subsubsection{Regular Maintenance}
%- cleaning the stretcher
%
%- increasing the pump laser currents
%
%- changing the chiller fluid, desiccants, etc
%
%\subsubsection{quirks and features}
%- regen cavity tweaks
%
%- photodiode problems
%
%- software issues - bugs and troubleshooting
%
%\subsection{Pointing Stabilization into the External Compressor}
%- dietrich plots for pointing
%\subsection{The Spitfire's External Compressor}
%\subsubsection{external compressor alignment}
%\subsubsection{cleaning the grating}
%
%\subsection{The TOPAS-HE}
%- aligning
%- importance of power stability and pointing stability
%
%\subsection{stability}
%- boxing things up
%- power stability throughout the day, people in the lab
%- unstable harmonic yield from the HPC at high pressures
%\section{The Shutter System}
%
%\section{The Vacuum System}
%- blower upgrade
%
%- remote pressure sensing
%
%- vacuum calculations for steady state pressure of beamline
%
%\section{Required Maintenance}
%
%vacuum system (rough pumps, blowers, turbos) and spitfire (cleaning the gratings)
%
%\section{Best practices: data acquisition}
%
%read-out noise from camera. (how noise scales)


%\chapter{Satellite stripe plot}

The satellite stripe plot can be made with data from any flight, however, this appendix shows how to create the plot using data from the \gls{anita}-2 and -3 flights (Figure~\ref{sat_stripe}). The plot is a two-dimensional histogram made with ROOT. The longitude of the \gls{anita} payload is in the vertical axis and the azimuthal reconstruction angle of events using their waveforms in \gls{lcp} is in the horizontal axis. The quantity in the horizontal axis, phi, is corrected for heading of the payload and calculated as shown in Equation~\ref{phi}. The color axis in the plot represents the number of events. Over-densities of events can be seen as stripes at certain longitudes. Each stripe is thought to be due to an individual group of satellites.

\begin{equation}
phi = fmod((phi_{LCP} - heading + 360),360)
\label{phi}
\end{equation}

\begin{figure}
\centering
\includegraphics[width=1.0\textwidth]{figures/same_stripes.png}
\caption{Satellite stripe plots for the ANITA-2 and ANITA-3 flights.}
\label{sat_stripe}
\end{figure}

\section{Code for satellite stripe plot}

Example code used to make the satellite stripe plot for the \gls{anita}-3 and -2 flights are shown below. The \gls{anita}-3 code is a macro and runs independent of other \gls{anita} software. It needs to be run on Oakley as the \gls{anita}-3 data is located there. The \gls{anita}-2 code is meant to be compiled and run inside the \path{anita2code} directory of the binned analysis software which is located at:
\href{https://github.com/osu-particle-astrophysics/BinnedAnalysis}{https://github.com/osu-particle-astrophysics/BinnedAnalysis}. 

The code to make the \gls{anita}-3 satellite stripe plot is a macro called \path{plotLonPhi}. A macro is a piece of code in ROOT meant to serve only one function. Inside the macro, that function is written and the name of the macro is the same as the name of the function. 
In this particular macro, first I declare a TChain object. A TChain object is a collection of files containing TTrees. This is useful in \gls{anita} as you can add together the TTrees associated with data files of multiple runs into one object. To make the satellite stripe plot for \gls{anita}-3, we use the output files from runInterferometry.cxx of the \gls{anita}-3 binned analysis. These are saved as multiple ROOT files, one file per run. 

I used the 10\% data to make the \gls{anita}-3 plot. To use the 90\% data, change \path{sample_10} in the directory name to \path{sample_90}. Note that Draw, a member function of both TTree and TChain, is used. This allows one to make the plot without including the classes that data objects inherit from. The function Draw accesses what is inside the TTree directly without requiring a definition of the data type inside the tree. The plot can also be made in the traditional way of filling a histogram with entries from a TChain inside a for loop. 

The idea behind making the satellite stripe plot for \gls{anita}-2 is the same, however, the \gls{anita}-2 analysis software is unique. 
\gls{anita}-2 data is on Kingbee and that is where \path{anita2code} have been run and tested. The code to make the satellite stripe plot for \gls{anita}-2 is part of \path{oindreeskymap.cc} in \path{anita2code}. Note the \path{.h} files that must be included to run this code, especially \path{analysis_info_4pol.h}, which is a struct holding the necessary data variables. 


\par
\begin{verbbox}
/////////////////////FOR ANITA-3///////////////////////
void plotLonPhi();

void plotLonPhi()

{
  //Declare a TChain object (collection of files containing TTrees)
  TChain tchain("resultTree");
  
  //Add files containing data processed by interferometry
  tchain.Add("/fs/scratch/PAS0174/anita
  /2015_05_19/sample_10/geomFilter/analyzerResults_*.root");

  //Declare a TH2D object 
  //First argument is the name of the histogram
  //which is same as the variable name here 
  TH2D hlonPhi00("hlonPhi00","ANITA-3 10% Data LPol;
  phi (degrees);longitude (degrees)",360,0,360,360,-180,180);
  
  //Declare a TCanvas object 
  //which is needed to make a plot in ROOT
  TCanvas cL("cL","cL",900,800);
  
  //Use the Draw function to plot the histogram
  //TTree and TChain have this useful function Draw 
  //Draws and puts the histogram in the TH2D object you specified
  tchain.Draw("longitude:(fmod((peak[0][0].phi - heading + 360),360)) 
  >> hlonPhi00", "circPol == 1", "colz");
  //Set the color axis to log scale
  cL.SetLogz();
  //Save plot as a .png (or other chosen format)
  cL.SaveAs("LonPhiPeak00CPol1.png");
  //Save plot as .root as well for quick changes as needed
  cL.SaveAs("LonPhiPeak00CPol1.root");
  
}
\end{verbbox}
\fbox{\theverbbox}\par

\par
\begin{verbbox}
/////////////////////FOR ANITA-2///////////////////////
#include "analysis_info_4pol.h" //struct holding data variables 
using namespace std;
class MyCorrelator;
int main()
{
  char filename90[10000];
  char filename10[10000];
  sprintf(filename90,"/data/anita/btdailey/final_filter/
  90sample/geom_4pol_partial_0301/output*000.root");
  sprintf(filename10,"/data/anita/btdailey/final_filter/
  10sample/geom_4pol_partial_0301/output*.root");
  TChain tchain("analysis_info_4pol");
  tchain.Add(filename90);
  tchain.Add(filename10);
  
  //Create a pointer to instantiate struct 
  analysis_info_4pol *pol4_Ptr = NULL;
  tchain.SetBranchAddress("pol4_Ptr",&pol4_Ptr);
  
  //Note: R & L are switched in ANITA-2
  TH2D LonPhiR("LonPhiR","ANITA-2 100% Data After Quality Cuts;
  phi (degrees); longitude (degrees)", 360,0,360,360,-180,180);
  
  TCanvas cRmap("cRmap","cRmap",1000,800);
  tchain.Draw("pol4_Ptr->anitaLon:
  (fmod((pol4_Ptr->phiMap[3]-pol4_Ptr->heading+360),360)) 
  >> LonPhiR","","colz"); //R & L are switched in ANITA-2
  
  cRmap.SetLogz();
  cRmap.SaveAs("LonPhiR100pc.png");
  cRmap.SaveAs("LonPhiR100pc.root");
}
\end{verbbox}
\fbox{\theverbbox}

\end{document}
