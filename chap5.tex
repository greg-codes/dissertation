\chapter{Conclusions}
\label{chap:conclusions}

In this thesis, we laid the groundwork for performing attosecond transient absorption spectroscopy (ATAS) measurements in the condensed phase using mid-infrared (MIR) lasers. We designed, built and tested an attosecond transient absorption beamline (TABLe) and a two-dimensional XUV spectrometer. The vacuum system was designed to accept different detector endstations, much like a user facility. The pump arm of the TABLe was designed to be outside the vacuum system so that nonlinear optics or other systems can be inserted into the beam path, which broadens the potential phase space of future studies. A high pressure cell (HPC) was designed and demonstrated to be nearly two orders of magnitude brighter than existing XUV sources in the DiMauro lab. This technical achievement partially counteracts the low quantum efficiency and difficult phase matching conditions of HHG at longer wavelengths. Finally, this equipment was demonstrated in a prototypical ATAS experiment in a technologically important indirect bandgap semiconductor (germanium) using a ${\lambda = 1430 \ \textrm{nm}}$ wavelength ultrafast excitation pulse. With a Keldysh parameter on the order of 1, the ionization channel is closer to the tunneling regime compared to previous reports in the literature. We observed electron and phonon dynamics in germanium that are consistent with what is reported in the literature. Our limited energy resolution, which is an artifact of our harmonic comb and detector nonlinearities, prevented us from resolving the energy dependence of the dynamics on the femtosecond timescale.

Future efforts should concentrate on making an XUV continuum and exciting the sample with longer wavelengths. An isolated attosecond pulse (IAP) would increase both spectral and temporal resolution. Increasing the pump wavelength would further reduce the Keldysh parameter $\gamma$, changing the nature of the initial ionization. For example, a below-bandgap $2 \ \mu \textrm{m}$ excitation would result in $\gamma\simeq 0.25$ while simultaneously suppressing single-photon ionization.