\begin{abstract}

In this thesis, we lay the groundwork for performing attosecond transient absorption spectroscopy (ATAS) measurements in the condensed phase using mid-infrared (MIR) lasers. This was accomplished by designing, building and testing several pieces of home-built experimental equipment, including a MIR / extreme ultraviolet (XUV) Mach–Zehnder interferometer and a two-dimensional XUV spectrometer. A home-made bright XUV light source was designed and demonstrated to be nearly two orders of magnitude brighter than existing sources. Finally, the equipment was used to study ultrafast dynamics in germanium, a technologically important indirect bandgap semiconductor.

This thesis is organized as follows. \cref{chap:intro} introduces the relevant background, including ultrafast dynamics and the tools required to observe them. \cref{chap:Experimental_Apparatus} details the commercial laser system, the home-built \textit{transient absorption beamline} (TABLe) and the XUV spectrometer. In \cref{chap:XUV_source_design_performance}, we design and optimize the XUV light source for high flux, which is a general requirement for ATAS measurements and especially needed at longer wavelengths with poor high harmonic generation (HHG) quantum efficiency. Also covered in \cref{chap:XUV_source_design_performance} are basic diagnostics of the XUV \& IR optics, as well as our XUV detector. In \cref{chap:ATAS_in_Ge}, we present the results of a MIR ATAS experiment in germanium, an experimental first. \cref{chap:conclusions} concludes the dissertation with a roadmap for future condensed matter studies. An appendix provides instructions on how to operate some aspects of the home-built experimental apparatus.

\end{abstract}
